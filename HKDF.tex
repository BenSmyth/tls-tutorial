\subsection{Key establishment}\label{sec:hkdf}

Once a \ServerHello\ message has been sent, a shared (handshake traffic) key can be 
established, and that key can be used to enable confidentiality 
and integrity for the remainder of the handshake protocol, which includes the 
subsequent \EncryptedExtensions\ message (\S\ref{sec:EE}). The initial shared key 
is derived by application of a key derivation function, 
known as a \emph{HMAC-based Extract-and-Expand Key Derivation Function} (HKDF),
which applies the negotiated hash function to the handshake protocol's 
transcript and either the negotiated pre-shared key, the negotiated (EC)DHE key,
%(Appendix~\ref{sec:dheKey}), 
or both. Further shared (application traffic) keys can be established 
similarly, to protect additional data, including application data.
Since transcripts include client- and server-generated nonces, shared (traffic) keys
are always unique, regardless of whether the pre-shared key (or for that matter
(EC)DHE key shares) are used for multiple connections.

\subsubsection{Transcript hash}\label{sec:hkdf:transcript}

A protocol's transcript concatenates each of the protocol's 
messages, in the order that they were sent, including message headers (namely, type 
and length fields, as introduced in Section~\ref{sec:record}), 
but excluding record-layer headers. The concatenation of messages
starts with \ClientHello, optionally followed by \HelloRetryRequest\ 
and \ClientHello\ if present, and proceeded by \ServerHello.
That transcript is used in computing transcript traffic keys 
(which protect the remaining handshake messages). Thereafter, 
the concatenation of messages is extended with \EncryptedExtensions\
and optionally \CertificateRequest, \Certificate, and 
\CertificateVerify\ if sent. A MAC over that transcript 
is included in a server's \Finished\ message, and a signature
over the transcript (excluding message \CertificateVerify) is 
included in any \CertificateVerify\ message. Once extended with
that \Finished\ message, the transcript is used in computing
%
\begin{comment}
the application traffic key. 
\end{comment}
%
%% Clarified in email from Eric Rescorla, dated 11 May 2020: 
%
the application traffic keys (which protect application traffic).
Finally, for a client's \Finished\ message, the transcript
is further extended with their \EndOfEarlyData, \Certificate,  
and \CertificateVerify\ messages (as relevant), before computing a 
MAC, wherein any \CertificateVerify\ message includes a signature over
that transcript (excluding itself). 
%
\begin{comment}
Finally, the client's application
traffic key is computed after extending the transcript with the 
client's \Finished\ message.
\end{comment}


To capture a transcript hash (i.e., a hash of a transcript), 
we introduce function \TranscriptHash\ such that
%
\[
  \TranscriptHash(M_1,\dots,M_n) = \Hash(M_1 \parallel \dots \parallel M_n)
\]
%
for handshake protocol messages $M_1,\dots,M_n$ (sent in that order), where \Hash\ is the 
negotiated hash function and $\parallel$ denotes concatenation, except when 
messages $M_1$ and $M_2$ are \ClientHello\ and \HelloRetryRequest\ messages, 
respectively. In that case, $M_1$ is replaced by $M'_1$ in the hash, i.e., 
%
\[
  \TranscriptHash(M_1,\dots,M_n) = \Hash(M'_1 \parallel M_2 \parallel \dots \parallel M_n),
\]
where $M'_1$ is the following special, synthetic handshake message, 
namely, \ifSpecNotes\marginpar{Truthify, is the 0x0000 padding or something else}\fi
%
\begin{align*}
  M'_1 = 
    & \mathbin{\phantom{\parallel}}  0\textrm{x}FE &\textrm{/* header type \TLSmessageHash */}     \\
    & \parallel                      0\textrm{x}0000 \parallel \HashLength &\textrm{/* (padded) header length */} \\
    & \parallel                      \Hash(M_1) &\textrm{/* hash of \ClientHello\ message */}
\end{align*}
%
where \HashLength\ is the output length in bytes of negotiated hash function \Hash.
This special case enables servers to construct transcripts without maintaining 
state, in particular, they need not store an initial \ClientHello\ message,
since it can be stored in extension \TLScookie\ (\S\ref{sec:HRR}).\footnote{%
  \HelloRetryRequest\ messages need not be maintained by the server either, since
  they can be reconstructed from \ClientHello\ messages and the special constant
  value that is used by field \HelloRetryRequest\code{.}\TLSrandom.}

\begin{tcolorbox}
Transcript hashing is implemented by class \code{HandshakeHash} (Listing~\ref{lst:HandshakeHash}).
Instances of that class form part of the active client and server contexts (instantiated by class 
\code{SSL\-Socket\-Impl} (or \code{SSL\-Engine\-Impl}), which is updated by classes 
\code{SSL\-Socket\-Input\-Record} and \code{SSL\-Socket\-Output\-Record} (respectively classes 
\code{SSL\-Engine\-Input\-Record} and \code{SSL\-Engine\-Output\-Record})). Moreover, in the case 
of a \HelloRetryRequest\ message, it is updated by class 
\code{Server\-Hello.T13\-Hello\-Retry\-Request\-Consumer} (Listing~\ref{lst:T13HelloRetryRequestConsumerB})
and by class \code{HelloCookieManager} (Listing~\ref{lst:T13HelloCookieManager}), when consuming any 
\TLScookie\ extension associated with a corresponding \ClientHello\ message.
\end{tcolorbox}

\lstinputlisting[
  float=tbp,
  widthgobble=0*0,
  linerange={
    37-39,
    %40-40    %%Omit field hasBeenUsed and use of that field. It used by SSLMasterKeyDerivation and HandshakeContext
    41-44,
    %45-45,
    46-46,
    85-87,
    106-109,
    116-119,
    164-168,
    %169-169,
    170-170,
    172-176,
    178-180,
    %181-181,
    182-182,
    645-645    
  },
  label=lst:HandshakeHash,
  caption={[\code{HandshakeHash} supports transcript hashes]
    Class \code{HandshakeHash} defines field \code{reserves} to maintain
    a list of protocol messages (Line~39), which can be extended (e.g., with incoming 
    messages) using method \code{receive} (Lines~85--87), moreover, the class defines 
    field \code{transcriptHash} 
    as a message digest algorithm (Line~38, see also Lines~58 \& 551--644), whose digest 
    can be updated to include the aforementioned messages using methods \code{deliver}
    (Lines~116--119) and \code{update} (Lines~164--170). (The former method is used when 
    the digest should also include an additional message, e.g., an outgoing message, whereas 
    the latter only updates the digest with messages listed by field \code{reserves}.) 
    Furthermore, method \code{digest} returns the hash over the current digest (Lines~172--176) 
    and method \code{finish} resets all fields (Lines~178--182).
}]{listings/HandshakeHash.java}


\lstinputlisting[
  float=tbp,
  linerange={
    1401-1401,
    1403-1408,
    1411-1412,
    1415-1446
  },
  label=lst:T13HelloRetryRequestConsumerB,
  caption={[\code{ServerHello.T13HelloRetryRequestConsumer} modifies transcript hashes]
    Class \code{ServerHello.T13HelloRetryRequestConsumer} (omitted from 
    Listing~\ref{lst:T13HelloRetryRequestConsumer}) modifies the transcript hash 
    in the special case of \HelloRetryRequest\ messages: A hash of the \ClientHello\
    message is computed (Lines 1401--1415), using variable \code{chc.initialClientHelloMsg}
    that was initialised in Listing~\ref{lst:ClientHelloKickstartProducer}; a special,
    synthetic handshake message $M'_1$ is computed, as the concatenation of $0\textrm{x}FE$
    (Line~1426), $0\textrm{x}0000$ (Lines~1427--1428), \HashLength\ (Line~1429), and 
    the hashed \ClientHello\ message (Lines~1430--1431); the transcript hash's digest
    is reset and the special message is added (Line~1433--1434); a further message is computed 
    as the concatenation of $0\textrm{x}02$ (Line~1438), the \HelloRetryRequest\ message 
    length (Lines~1439--1441), and the \HelloRetryRequest\ (Lines~1443--1444); and that 
    message is added to the transcript hash's digest too (Line~1446). Thus, the client's
    active context includes the expected digest.
}]{listings/ServerHello.java}

\lstinputlisting[
  float=tbp,
  linerange={
    200-201,
    %202-216,
    271-272,
    278-279,
    284-284,
    291-292,
    312-335,
    338-339
  },
  label=lst:T13HelloCookieManager,
  caption={[\code{HelloCookieManager.T13HelloCookieManager} modifies transcript hashes]
    Class \code{HelloCookieManager.T13HelloCookieManager} processes cookies, in particular, 
    method \code{isCookieValid} tests the validity of cookies. That method also updates
    the transcript hash in the special case of \HelloRetryRequest\ messages: A
    \HelloRetryRequest\ message is reconstructed and added to the front of the 
    transcript hash's digest (Lines~322--324). Moreover, a special, synthetic handshake 
    message $M'_1$ is computed as the concatenation of $0\textrm{x}FE0000 \parallel \HashLength$ 
    and the hash (of a \ClientHello\ message) stored in the cookie, and message 
    $M'_1$ is added to the front of the transcript hash's digest (Lines 327--335).
    \ifImplNotes\textcolor{red}{I don't understand why \code{HandshakeHash.push} is needed, it
    seems that the transcript hash could be reset and the digest reconstructed, similarly
    to Listing~\ref{lst:T13HelloRetryRequestConsumerB}. Presumably there's a side-case
    when this doesn't work (probably because the digest already contains something useful, hence, 
    cannot be reset).}\fi
}]{listings/HelloCookieManager.java}






\subsubsection{Key derivation}\label{sec:secrets}

The key derivation process combines the negotiated pre-shared key, the 
(EC)DHE key, or both, with the protocol's transcript. The process 
uses function \HKDFExtract, which is defined by RFC~5869
such that 
%
\[
  \HKDFExtract(\TLSHKDFSalt,\TLSHKDFSecret) = 
  \left\{
    \begin{matrix*}[l]
      \HMACHash(\zeros,\TLSHKDFSecret)             & \textrm{if \TLSHKDFSalt\ is null}\\
      \HMACHash(\TLSHKDFSalt,\TLSHKDFSecret)  & \textrm{otherwise},
    \end{matrix*}
  \right.
\]
%
%parameterised by salt value $\TLSHKDFSalt$ 
%%(and is a \HashLength-length string of zeros if not provided) 
%and secret $\TLSHKDFSecret$, 
where $\zeros$ denotes a \HashLength-length string of zeros
\ifSpecNotes
\textcolor{red}{Rather than \zeros, The spec uses 0, which isn't particularly intuitive, 
since it can be confused for, well, integer 0. We deviate for clarity.}
\fi
(and function \HMACHash\ is specified by RFC~2104 over keys and messages, hence, the above
definition treats $\TLSHKDFSalt$ as a key and $\TLSHKDFSecret$ 
as a message when applying \HMACHash).
In the context of key derivation, 
the salt is initially $\zeros$ 
and the secret is initially the pre-shared key or $\zeros$ if no such 
key was negotiated. The function's first output is known 
as \TLSEarlySecret. It follows that
\[
  \TLSEarlySecret = \HKDFExtract(\zeros,\textrm{PSK}),
\]
where PSK is $\zeros$ for (EC)DHE-only key exchange and
otherwise the pre-shared key, which provides raw 
entropy without context.
Context is added using function \DeriveSecret, defined 
such that 
%
\[
  \DeriveSecret(\TLSHKDFSecret,\TLSHKDFLabel,\TLSHandshakeMessage) 
    %= \HKDFExpandLabel(\TLSHKDFSecret, \TLSHKDFLabel, \TranscriptHash(\TLSHandshakeMessage), \HashLength)
      %= \HKDFExpand(\TLSHKDFSecret,\TLSHKDFLabelExt,\HashLength)
        %= \HMACHash(\TLSHKDFSecret, ``" \parallel \TLSHKDFLabelExt \parallel 0\textrm{x}01), \marginpar{Does $``" \parallel M = M$? If so, simplify left}
          = \HMACHash\left(\TLSHKDFSecret, \TLSHKDFLabelExt \parallel 0\textrm{x}01\right),
\]
%
%parameterised by secret \TLSHKDFSecret\ and label \TLSHKDFLabel, 
where \TLSHandshakeMessage\ is a concatenation of the protocol's messages (\S\ref{sec:hkdf:transcript})
and $\TLSHKDFLabelExt$ is defined as the following message,\footnote{
  RFC 8446 defines function \DeriveSecret\ in terms of functions \HKDFExpandLabel\
  and \HKDFExpand, namely,
  $\DeriveSecret(\allowbreak\TLSHKDFSecret,\allowbreak\TLSHKDFLabel,\allowbreak\TLSHandshakeMessage) 
    = \HKDFExpandLabel(\allowbreak\TLSHKDFSecret,\allowbreak \TLSHKDFLabel,\allowbreak \TranscriptHash(\TLSHandshakeMessage),\allowbreak \HashLength)
      = \HKDFExpand(\allowbreak\TLSHKDFSecret,\allowbreak \TLSHKDFLabelExt,\allowbreak \HashLength)
        = \HMACHash(\allowbreak\TLSHKDFSecret,\allowbreak %``" \parallel 
                                \TLSHKDFLabelExt \parallel 0\textrm{x}01)$, 
  where $\TLSHKDFLabelExt = \HashLength \parallel ``\textrm{tls13\textvisiblespace}" \parallel \TLSHKDFLabel \parallel \TranscriptHash(\TLSHandshakeMessage)$.
  By comparison, we define function \DeriveSecret\ more directly and defer the 
  additional functions to Section~\ref{sec:hkdf:trafficKeys}, where we consider
  functions \HKDFExpandLabel\ and \HKDFExpand\ more generally (in particular,
  the former may omit transcript hashes in favour of strings and the latter
  may consider lengths other than \HashLength).
}
namely,
\[
  \TLSHKDFLabelExt = \HashLength \parallel ``\textrm{tls13\textvisiblespace}" \parallel \TLSHKDFLabel \parallel \TranscriptHash(\TLSHandshakeMessage).
\]
Function \DeriveSecret\ is used (with the empty context) to derive salt for subsequent
applications of \HKDFExtract. Indeed, we have
\[
  \TLSHandshakeSecret = \HKDFExtract(\DeriveSecret(\TLSEarlySecret, ``\textrm{derived}", ``"),\textrm{K}),
\]
where \textrm{K} is \zeros\ for PSK-only key exchange and
otherwise the (EC)DHE key, moreover, 
\[
  \TLSMasterSecret = \HKDFExtract(\DeriveSecret(\TLSHandshakeSecret, ``\textrm{derived}", ``"),\zeros),
\]
%
noting that \TranscriptHash(``'') = \Hash(``''), that is, the hash of the empty string
(null ASCII character 0x00).
Traffic secrets are derived from \TLSEarlySecret, \TLSHandshakeSecret, and \TLSMasterSecret, 
as shown in Figure~\ref{fig:keyDerivation}, by adding context. Those secrets are used to 
derive traffic keys (\S\ref{sec:hkdf:trafficKeys}) to protect the data summarised in the 
following table:

\ifPresentationNotes
\marginpar{Table nor Listing are complete.}
\fi

\begin{table}[H]
\caption{Traffic secrets that underlie traffic keys used to protect data}
\label{table:secrets}
\centering
\begin{tabular}{l|l}
Underlying traffic secret & Protected data \\ \hline
%\TLSbinderKey & \textcolor{red}{to do}\\
\TLSclientEarlyTrafficSecret & 0-RTT \\
%\TLSearlyExporterMasterSecret & \textcolor{red}{omit exporterts?} \\
\TLShandshakeTrafficSecret & Handshake extensions %\EncryptedExtensions,  
  %\CertificateRequest, \Certificate, \CertificateVerify, and \Finished. 
  \\
\TLSapplicationTrafficSecretN & Application traffic %\\
%\TLSexporterMasterSecret & \textcolor{red}{omit exporterts?} \\
%\TLSresumptionMasterSecret& \textcolor{red}{to do}
\end{tabular}
\end{table}

\noindent
where \TLSField{[sender]} is either \TLSField{client} or \TLSField{server}, and
\TLSapplicationTrafficSecretN[N+1] is defined as follows when $N>0$, namely,
%
\begin{multline*}
  \TLSapplicationTrafficSecretN[N+1] \\
    = \DeriveSecret(\TLSapplicationTrafficSecretN, ``\textrm{traffic upd}", ``"),
\end{multline*}
%
which is used to update application-traffic secrets.


\begin{figure}
%%
%% Lines 5159 & 5189 of listings/rfc8446.txt is 0, whereas we're using 0s
%% 
\begin{lstlisting}[belowskip=0pt,numbers=none,frame=none,nolol=true]
             0s
\end{lstlisting}
%
\lstinputlisting[
  aboveskip=0pt,
  belowskip=0pt,
  numbers=none,
  frame=none,
  nolol=true,
  linerange={
    %5159-5162,
    %5166-5168,
    %5159-5168,
    5160-5168,
    5172-5188
  }
]{listings/rfc8446.txt}
%
\begin{lstlisting}[aboveskip=0pt,belowskip=0pt,numbers=none,frame=none,nolol=true]
  0s -> HKDF-Extract = Master Secret
\end{lstlisting}
%
\lstinputlisting[
  aboveskip=0pt,
  numbers=none,
  frame=none,
  nolol=true,
  linerange={
    5190-5197,
    5202-5205
  }
]{listings/rfc8446.txt}

\caption[Key derivation process]{
  Key derivation process, showing application of functions \HKDFExtract\ and \DeriveSecret\
  to derive working keys. Function \HKDFExtract\ is shown inputting salt from 
  the top and secrets from the left, and outputs to the bottom, where the output's name 
  is shown to the right. Moreover, function \DeriveSecret\ is shown inputting secrets from the 
  incoming arrow and the remaining inputs appear inline, and some outputs are named below 
  the function's application (e.g., \TLSEarlySecret\ is input as the secret to generate
  \textrm{client\_early\_traffic\_secret}) and others serve as salt for subsequent 
  applications of the former (e.g., \TLSEarlySecret\ is input as the secret to 
  generate salt for \TLSHandshakeSecret). Output \TLSbinderKey\ is derived
  by application of function \DeriveSecret\ to 
  $``\textrm{ext binder}" \mid ``\textrm{res binder}"$ which denotes either 
  ``\textrm{ext binder}'' or ``\textrm{res binder}.'' The former is used for external 
  PSKs (those established independently of TLS) and the latter is used for resumption 
  PSKs (those established by \NewSessionTicket\ messages, using the resumption master 
  secret of a previous handshake), hence, one type of PSK cannot be substituted for the 
  other.\\
  \emph{Source: This figure is excerpted from RFC~8446.}
}
\label{fig:keyDerivation}
\end{figure}

\begin{tcolorbox}
Function \HKDFExtract\ is implemented by class \code{HKDF} (Listing~\ref{lst:HKDF})
and function \DeriveSecret\ is implemented by class \code{SSL\-Secret\-Derivation} 
(Listing~\ref{lst:SSLSecretDerivation} \&~\ref{lst:SSLSecretDerivationB}).
\ifImplNotes
  \textcolor{red}{Note: Implementation of function \DeriveSecret\ 
  is reliant on the implementation of \HKDFExpand, 
  which we haven't introduced yet (beyond a footnote reference), since a more 
  direct definition of function \DeriveSecret\ suffices for this section
  (as per the footnote). This doesn't seem worth mentioning.}
\fi
Application of the former %function 
is dependent on the negotiated pre-shared 
key to derive \TLSEarlySecret, which is computed by static method 
\code{Server\-Hello.set\-Up\-Psk\-KD} (Listing~\ref{lst:setUpPskKD}), except
for (EC)DHE-only key exchange, which derives $\TLSEarlySecret$ as $\HKDFExtract(\zeros,\zeros)$
and is computed by class \code{DH\-Key\-Exchange.DHEKA\-Generator.DHEKA\-Key\-Derivation} or
class \code{ECDH\-Key\-Exchange.ECDHEKA\-Key\-Derivation}, which also compute 
\TLSHandshakeSecret. (PSK-only key exchange is unsupported and \TLSHandshakeSecret\ 
is only computed with an (EC)DHE key.) Those classes
are identical up to constructor names, strings \code{"DiffieHellman"}
and \code{"ECDH"}, and whitespace. (Refactoring has replaced those 
classes with \code{KAKeyDerivation} in JDK-13.) So, for brevity, we only present the 
former class (Listing~\ref{lst:DHEKAKeyDerivation}).
\end{tcolorbox}

\lstinputlisting[
  float=tbp,
  widthgobble=0*0,
  linerange={
    46-49,
    61-61,
    64-67,
    86-95,
    114-120,
    185-185
  },
  label=lst:HKDF,
  caption={[\code{HKDF} implements function \HKDFExtract]
    Class \code{HKDF} defines method \code{extract} to implement 
    function \HKDFExtract\ (RFC~5869), 
    over salt values of type \code{SecretKey} (Lines~86--95) and 
    \code{byte[]} (Lines~114--120), using a \HMACHash\ function 
    derived from the negotiated hash function (Lines~64--65), 
    where \code{JsseJce.getMac(hmacAlg)} computes \code{Mac.getInstance(hmacAlg)}
    or \code{Mac.getInstance(hmacAlg, cryptoProvider)}, depending 
    on whether \code{sun.security.ssl.SunJSSE.cryptoProvider} is 
    null.
    Both implementations allow the 
    salt to be \code{null} and will instantiate salt as a zero-filled 
    byte array of the same length as \HashLength\ (Lines~88--90 \& 116--118). 
    An HMAC is initialised with the salt as a key (Line~91) and a secret 
    as the message (Line~93), the resulting HMAC is returned as 
    a key of type \code{javax.crypto.spec.SecretKeySpec}.
}]{listings/HKDF.java}

\lstinputlisting[
  float=tbp,
  widthgobble=0*0,
  linerange={
    36-36,
    63-63,
    %64-64, %hkdfAlg
    65-67,
    69-73,
    %74-75, %init hkdfAlg (constructor)
    76-78,
    85-96,
    101-114
%    116-129,
  },
  label=lst:SSLSecretDerivation,
  caption={[\code{SSLSecretDerivation} implements function \DeriveSecret]
    Class \code{SSLSecretDerivation} implements function \DeriveSecret. The 
    class defines a constructor (Lines~69--78) that instantiates fields
    \code{context}, \code{transcriptHash} and \code{secret} 
    with data including the transcript hash, the hash of the corresponding digest 
    and a secret, respectively. Method \code{deriveKey} (Lines~85--114) is 
    instantiated with a string that references a label and returns an HMAC computed 
    by application of method \code{HKDF.expand} (Line~109) to inputs including 
    \TLSHKDFLabelExt, which is computed (Lines~105--106) over the 
    negotiated hash function's output length, the label prepended with 
    $``\textrm{tls13\textvisiblespace}"$, and a hash of 
    either the transcript's digest (when the resulting output will be used as a secret)
    or the empty digest (when the resulting output will be used as salt, i.e., when 
    \code{ks == SecretSchedule.TlsSaltSecret}), using static method 
    \code{SSLSecretDerivation.createHkdfInfo}  to handle concatenation.
    \ifImplNotes
    \textcolor{blue}{\code{SSLBasicKeyDerivation.createHkdfInfo} can be dropped in 
    favour of \code{SSLSecretDerivation.createHkdfInfo} (noting a slight
    difference regarding runtime exceptions). As can 
    \code{SSLTrafficKeyDerivation.T13TrafficKeyDerivation.createHkdfInfo}
    (noting the need to supply 0x00 as the context).}
    \textcolor{red}{Reported to security-dev@openjdk.java.net on 27 May 2020
    (\url{https://mail.openjdk.java.net/pipermail/security-dev/2020-May/021928.html}), 
    bug report added (\url{https://bugs.openjdk.java.net/browse/JDK-8245983})}
    \fi
}]{listings/SSLSecretDerivation.java}



\lstinputlisting[
  float=tbp,
  widthgobble=0*0,
  linerange={
    132-151,
    152-152    
  },
  label=lst:SSLSecretDerivationB,
  caption={[\code{SSLSecretDerivation.SecretSchedule} implements function \DeriveSecret\ (cont.)]
    Enum \code{SSLSecretDerivation.SecretSchedule} (omitted from Listing~\ref{lst:SSLSecretDerivation})
    maps strings to labels used by function \DeriveSecret, and prepends labels with 
    $``\textrm{tls13\textvisiblespace}"$.
}]{listings/SSLSecretDerivation.java}

\begin{comment}
\lstinputlisting[
  float=tbp,
  widthgobble=0*0,
  linerange={
    116-130    
  },
  label=lst:SSLSecretDerivationB,
  caption={[Class \code{SSLSecretDerivation} implements function \ExpandLabel]
    Class \code{SSLSecretDerivation} (omitted from Listing~\ref{lst:SSLSecretDerivationB})
    implements function \ExpandLabel...
}]{listings/SSLSecretDerivation.java}
\end{comment}

\lstinputlisting[
  float=tbp,
  widthgobble=0*0,
  linerange={
    1152-1153,
    1159-1170
  },
  label=lst:setUpPskKD,
  caption={[\code{ServerHello.setUpPskKD} derives \TLSEarlySecret\ over a pre-shared key]
    Static method \code{ServerHello.setUpPskKD} (omitted from Listings~\ref{lst:T13ServerHelloProducer}
    \&~\ref{lst:T13ServerHelloConsumer}) derives \TLSEarlySecret\ over a negotiated 
    pre-shared key. 
}]{listings/ServerHello.java}


\lstinputlisting[
  float=tbp,
  linerange={
    449-453,
    %454-463, %constructor
    464-465,    
    469-469,
    471-471,
    499-532
  },
  label=lst:DHEKAKeyDerivation,
  caption={[\code{DHKeyExchange.DHEKAGenerator.DHEKAKeyDerivation} derives keys]
  Class \code{DHKeyExchange.DHEKAGenerator.DHEKAKeyDerivation} defines method \code{t13DeriveKey}
  to derive the negotiated key (Lines~502--506); compute \TLSEarlySecret, for (EC)-DHE-only
  key exchange (Lines~511--520), i.e., when production or consumption of a \ServerHello\ 
  message did not call method \code{ServerHello.setUpPskKD}, which instantiates
  \code{context.handshakeKeyDerivation}; applies \DeriveSecret\ to \TLSEarlySecret\
  and label ``\textrm{derived}'' (Line~523); and uses the resulting output as salt when 
  applying \HKDFExtract\ to the negotiated key (Line~526), which produces 
  \TLSHandshakeSecret.
}]{listings/DHKeyExchange.java}


\subsubsection{Traffic keys}\label{sec:hkdf:trafficKeys}

Traffic keys are derived from traffic secrets listed in Table~\ref{table:secrets}, using 
function \HKDFExpandLabel, defined such that
%
\begin{multline*}
  \HKDFExpandLabel(\TLSHKDFSecret, \TLSHKDFLabel, \TLSContext, \TLSHKDFLength) \\
    = \HKDFExpand(\TLSHKDFSecret, \TLSHKDFLabelExt, \TLSHKDFLength),
\end{multline*}
%
where $\TLSHKDFLabelExt = \TLSHKDFLength \parallel ``\textrm{tls13\textvisiblespace}" \parallel \TLSHKDFLabel \parallel \TLSContext$
and function \HKDFExpand\ is defined by RFC 5869 such that
$
  \HKDFExpand(\TLSHKDFSecret,\allowbreak \TLSHKDFExpLabel,\allowbreak \TLSHKDFLength)
$
outputs the first \TLSHKDFLength-bytes of $T_1 \parallel \dots \parallel T_n$, where
$n = \lceil \frac{\TLSHKDFLength}{\HashLength} \rceil$ and 
%
\begin{align*}
  T_0 &= ``" \\
  T_1 &= \HMACHash(\TLSHKDFSecret, T_0 \parallel \TLSHKDFExpLabel \parallel 0\textrm{x}01 ) \\
  T_2 &= \HMACHash(\TLSHKDFSecret, T_1 \parallel \TLSHKDFExpLabel \parallel 0\textrm{x}02 ) \\
  &\vdots
\end{align*}

\noindent
Function \HKDFExpandLabel\ may input \TLSContext\ as the null ASCII character 0x00, denoted ``''.


\begin{tcolorbox}
Function \HKDFExpand\ is implemented by class \code{HKDF} (Listing~\ref{lst:HKDFB}) and
traffic keys are derived by class \code{SSLTrafficKeyDerivation} (Listing~\ref{lst:SSLTrafficKeyDerivation}). 
\ifImplNotes
\textcolor{red}{Note: Class \code{SSLTrafficKeyDerivation} shares similarities with 
\code{SSLSecretDerivation} and some refactoring may allow elimination of unnecessary code.}
\fi
\end{tcolorbox}

\lstinputlisting[
  float=tbp,
  widthgobble=0*0,
  linerange={
    137-139,
    150-150,
    %151-153
    154-159,
    162-184
  },
  label=lst:HKDFB,
  caption={[\code{HKDF} implements function \HKDFExpand]
    Class \code{HKDF} (omitted from Listing~\ref{lst:HKDF}) 
    defines method \code{expand} to implement function \HKDFExpand. 
    A buffer \code{kdfOutput} of length 
    $\HashLength\cdot\lceil \frac{\TLSHKDFLength}{\HashLength} \rceil$ is initialised
    (Lines~139 \& 154--155) and an HMAC is initialised with the input secret as a key (Line~150).
    The for-loop computes $T_1,T_2,...$ values as HMACs over messages that concatenate the 
    previous round's output (which is the empty string during the first round), label
    \code{info}, and the round number (Lines~167--170). Those values are stored in 
    buffer \code{kdfOutput} (Line~171), which is returned as a key of type 
    \code{javax.crypto.spec.SecretKeySpec} after truncating to length \code{outLen}
    (Line~183).
%
    \ifImplNotes\textcolor{red}{Lines~151-153 are omitted. As far as I can tell, info is never null}\fi
 %   
}]{listings/HKDF.java}

\lstinputlisting[
  float=tbp,
  %widthgobble=0*0,
  linerange={
    %43-43,   %%Outer class (SSLTrafficKeyDerivation)
    %47-47,
    %49-50,
    %52-56,
    %77-80,
    %121-122,
    %128-133,
    135-137,
    139-143,
    146-160,
%    162-176,
    177-177,
    179-202
  },
  label=lst:SSLTrafficKeyDerivation,
  caption={[\code{SSLTrafficKeyDerivation.T13TrafficKeyDerivation} derives traffic keys]
  Class \code{SSLTrafficKeyDerivation.T13TrafficKeyDerivation} derives traffic keys. 
  Method \code{deriveKey} is instantiated with a string that references a label and returns
  an HMAC computed by application of method \code{HKDR.expand} (Lines~153--155) to inputs including 
  \TLSHKDFLabelExt, which is computed (Lines~151--152) over the negotiated hash function's output length,
  the label prepended with $``\textrm{tls13\textvisiblespace}"$, and null ASCII character 0x00, using 
  static method \code{SSLTrafficKeyDerivation.T13TrafficKeyDerivation.createHkdfInfo}
  to handle concatenation and to introduce 0x00.
}]{listings/SSLTrafficKeyDerivation.java}


\noindent
Returning to key derivation, we derive the following traffic keys:
%
\begin{align*}
  \TLSwriteKey &= \HKDFExpandLabel(\TLSField{[secret]}, ``key", ``", key\_length) \\
  \TLSwriteIV  &= \HKDFExpandLabel(\TLSField{[secret]}, ``iv", ``", iv\_length)
\end{align*}
%
where \TLSField{[sender]} is either \TLSField{client} or \TLSField{server}, and
\TLSField{[secret]} is taken from the secrets listed in Table~\ref{table:secrets}.

\begin{tcolorbox}
Server- and client-side handshake-traffic key derivation is implemented by classes 
\code{Server\-Hello.T13\-Server\-Hello\-Producer} %(Listing~\ref{lst:T13ServerHelloProducerC})
and \code{Server\-Hello.T13\-Server\-Hello\-Consumer}, %(Listing~\ref{lst:T13ServerHelloConsumerB}), 
respectively.
The former class defines method \code{produce} to write a \ServerHello\ message to an 
output stream (Listings~\ref{lst:T13ServerHelloProducer} 
\&~\ref{lst:T13ServerHelloProducerB}), and that method derives handshake-traffic 
keys immediately after writing the \ServerHello\ message; the keys are used to encrypt subsequent 
outgoing handshake messages (including an \EncryptedExtensions\ message) and to decrypt subsequent
incoming handshake messages. Similarly, the latter class defines 
method \code{consume} to read a \ServerHello\ message from an input buffer 
(Listing~\ref{lst:T13ServerHelloConsumer}), and that method derives handshake-traffic 
keys immediately before reading an \EncryptedExtensions\ message (and prior to reading 
further extensions, including \Certificate\ and \CertificateVerify\ messages 
for (EC)DHE-only key exchange, and a \Finished\ message); the keys are used to decrypt subsequent
incoming handshake messages, including that \EncryptedExtensions\ message, and to encrypt subsequent
outgoing handshake messages.
The implementations 
are identical up to contexts (namely, \code{Server\-Handshake\-Context} and 
\code{Client\-Handshake\-Context}, that share parent \code{Handshake\-Context}), 
labels 
%hack for visible space:
{\code{s}\textvisiblespace{}\code{hs}\textvisiblespace{}\code{traffic}} and 
{\code{c}\textvisiblespace{}\code{hs}\textvisiblespace{}\code{traffic}}
(which are instantiated by enum \code{SSL\-Secret\-Derivation.Secret\-Schedule}
using strings \code{TlsServerHandshakeTrafficSecret} and 
\code{TlsClientHandshakeTrafficSecret}, respectively), treatment of null
in tricks to make the compiler happy (cf. \code{return null;} and \code{return;}
in catch-branches), $\alpha$-renaming of one variable, and whitespace
(and some obsolete, commented-out code). 
(Refactoring could eliminate unnecessary code.\footnote{%
  The OpenJDK team are aware of refactoring opportunities 
  (\url{https://mail.openjdk.java.net/pipermail/security-dev/2020-May/021928.html})
  and are tracking changes (\url{https://bugs.openjdk.java.net/browse/JDK-8245983}).
}) 
So, for brevity, we only present server-side handshake-traffic key derivation 
(Listings~\ref{lst:T13ServerHelloProducerC} \&~\ref{lst:T13ServerHelloProducerD}). 
%
%% The following has been moved to the previous section, where \TLSEarlySecret\
%% is introduced.
%
\begin{comment}
Derivation is dependent on \TLSEarlySecret\ and 
the negotiated (EC)DHE key (PSK-only key exchange is unsupported), 
which are computed by classes \code{DHKeyExchange.DHEKAGenerator.DHEKAKeyDerivation} and
\code{ECDHKeyExchange.ECDHEKAKeyDerivation}. Those classes
are identical up to constructor names, strings \code{"DiffieHellman"}
and \code{"ECDH"}, and whitespace. (Refactoring could elimate
unnecessary code.) So, for brevity, we only present the 
former class (Listing~\ref{lst:DHEKAKeyDerivation}).
\end{comment}

\end{tcolorbox}



\lstinputlisting[
  float=tbp,
  linerange={
    %587-587,
    588-589,
    591-592,
    593-598, %error handling for shc.handshakeKeyExchange == null 
    600-605,  
    613-615
  },
  label=lst:T13ServerHelloProducerC,
  caption={[\code{ServerHello.T13ServerHelloProducer} deriving keys]
  Class \code{ServerHello.T13ServerHelloProducer} (omitted from 
  Listing~\ref{lst:T13ServerHelloProducerB}) updates the transcript
  hash's digest to include all handshake protocol messages (Line~589), 
  derives an (EC)DHE key (Line~600), and establishes \TLSHandshakeSecret\
  (Lines~601--602). Variable \code{shc.handshakeKeyExchange} is assigned 
  by class \code{KeyShareExtension} (PSK-only key exchange is unsupported, 
    %%https://bugs.openjdk.java.net/browse/JDK-8049402
    %%https://bugs.openjdk.java.net/browse/JDK-8145252
    %%https://bugs.openjdk.java.net/browse/JDK-8209392
  hence, \code{ke} is not null) as an instance of class \code{SSLKeyExchange}
  parameterised with \code{SSLKeyExchange.T13KeyAgreement} (of type \code{SSLKeyAgreement})
  and \code{ke.createKeyDerivation(shc)} returns either 
    \code{ECDHKeyExchange.ecdheKAGenerator.createKeyDerivation(shc)} or
    \code{DHKeyExchange.kaGenerator.createKeyDerivation(shc)}, i.e., 
  an (EC)DHE key (Line~600). The class also initialises variables \code{kdg}
  (Line~604--605) and \code{kd} (Lines 614--615) which will be used to 
  derive traffic secrets and the corresponding traffic keys, respectively.
  The former is an instance of class \code{SSLTrafficKeyDerivation} that 
  overrides method \code{createKeyDerivation} such that it returns 
  an instance of class \code{SSLTrafficKeyDerivation.T13TrafficKeyDerivation}.
}]{listings/ServerHello.java}%


\begin{comment}
\marginpar{To aid readers of printed versions (and perhaps some digital readers), Listing~\ref{lst:T13ServerHelloProducerC} should appear on an even page and Listing~\ref{lst:T13ServerHelloConsumerB} should appear on the following (odd) page} 

\lstinputlisting[
  float=tbp,
  linerange={
    1242-1242,
    1249-1254,
    1262-1264,
    1266-1328
  },
  label=lst:T13ServerHelloConsumerB,
  caption={[\code{ServerHello.T13ServerHelloConsumer} derives keys]
    Class \code{ServerHello.T13ServerHelloConsumer} (omitted from 
  Listing~\ref{lst:T13ServerHelloConsumer})
}]{listings/ServerHello.java}%
\end{comment}


\lstinputlisting[
  float=tbp,
  linerange={
    617-636,
    640-664,
    668-675
  },
  label=lst:T13ServerHelloProducerD,
  caption={[\code{ServerHello.T13ServerHelloProducer} deriving keys (cont.)]
  Class \code{ServerHello.T13ServerHelloProducer} (continued from 
  Listing~\ref{lst:T13ServerHelloProducerC}) derives traffic secret
  \TLSclientHandshakeTrafficSecret\ (Lines~618--619), constructs an instance
  of \code{SSLTrafficKeyDerivation.T13TrafficKeyDerivation} from that secret
  (Lines~620--621), and uses that instance to derive the corresponding traffic 
  keys \TLSField{client\_write\_key} (Lines~622--623) and
  \TLSField{client\_write\_iv} (Lines~624--625), which will be used to decrypt (and read)
  incoming client traffic (Lines~626--643). Similarly, traffic secret   
  \TLSserverHandshakeTrafficSecret\ is derived (Lines~646--647), along with 
  traffic keys \TLSField{server\_write\_key} (Lines~650--651) and
  \TLSField{server\_write\_iv} (Lines~652--653), used to encrypt (and write) outgoing 
  traffic (Lines~654--672).
}]{listings/ServerHello.java}%


Traffic secrets \TLSclientHandshakeTrafficSecret\ and \TLSserverHandshakeTrafficSecret\
are used to derive handshake-traffic keys that protect handshake extensions (\S\ref{sec:EE} 
\&~\ref{sec:handshakeAuth}). After those extensions are processed,
application-traffic keys to protect application data can be derived (\S\ref{sec:FIN}).
