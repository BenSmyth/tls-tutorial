\section{Extensions}
\label{sec:extensions}

Extensions listed by an endpoint are generally 
followed by a corresponding extension from their peer. 
Corresponding extensions must not be sent without solicitation, and  
endpoints must abort with an \TLSunsupportedExtension\ alert upon 
receipt of such unsolicited extensions. For instance, a \ClientHello\ message listing 
extension \TLSsupportedGroups\ is followed by a \ServerHello\ message listing 
the same extension, whereas a \ServerHello\ message must not list that 
extension in response to a \ClientHello\ message that does not and a client
should abort in such cases. 

Table~\ref{table:extensions} formally specifies which extensions can be listed 
in the \TLSextensions\ field of handshake protocol messages. Endpoints 
must abort with an \TLSillegalParameter\ alert if an extension is received
in a handshake protocol message for which it is not specified. Support for 
the following extensions is mandatory (unless an implementation explicitly opts out): 
%
\TLScookie,
\TLSkeyShare,
\TLSserverName,
\TLSsignatureAlgorithms,
\TLSsignatureAlgorithmsCert,
\TLSsupportedGroups, and
\TLSsupportedVersions.
%
A client requesting a non-mandatory extension may abort if the extension is
not supported by the server. A server may require \ClientHello\ messages
to include extension \TLSserverName\ and should abort with an \TLSmissingExtension\ 
alert if the extension is missing. 


\begin{table}[H]
\caption{Extensions and the handshake protocol messages in which they
  may appear, where such messages are abbreviated as follows:
  CH (\ClientHello), SH (\ServerHello), EE (\EncryptedExtensions), 
  CT (\Certificate), CR (\CertificateRequest), NST (\NewSessionTicket), 
  and HRR (\HelloRetryRequest).}
\label{table:extensions}
\centering
\begin{tabular}{l|l|l}
Extension                                 &RFC & Handshake message \\ \hline
\TLSapplicationLayerProtocolNegotiation   &7301&      CH, EE    \\
\TLScertificateAuthorities                &8446&      CH, CR    \\
\TLSclientCertificateType                 &7250&      CH, EE    \\
\TLScookie                                &8446&     CH, HRR    \\
\TLSearlyData                             &8446& CH, EE, NST    \\
\TLSheartbeat                             &6520&      CH, EE    \\
\TLSkeyShare                              &8446& CH, SH, HRR    \\
\TLSmaxFragmentLength                     &6066&      CH, EE    \\
\TLSoidFilters                            &8446&          CR    \\
\TLSpadding                               &7685&          CH    \\
\TLSpostHandshakeAuth                     &8446&          CH    \\
\TLSpsk                                   &8446&      CH, SH    \\
\TLSpskModes                              &8446&          CH    \\
\TLSserverCertificateType                 &7250&      CH, EE    \\
\TLSserverName                            &6066&      CH, EE    \\
\TLSsignatureAlgorithms                   &8446&      CH, CR    \\
\TLSsignatureAlgorithmsCert               &8446&      CH, CR    \\
\TLSsignedCertificateTimestamp            &6962&  CH, CR, CT    \\
\TLSstatusRequest                         &6066&  CH, CR, CT    \\
\TLSsupportedGroups                       &7919&      CH, EE    \\
\TLSsupportedVersions                     &8446& CH, SH, HRR    \\
\TLSuseSrtp                               &5764&      CH, EE    
\end{tabular}
\end{table}


When designing new extensions, the following considerations should 
be taken into account:
%
First, a server that does not support a client-requested extension 
\ifSpecNotes
  \textcolor{red}{the spec states client requested feature, but my 
  narrowing seems correct}
\fi
  should indicate that the extension is unsupported by inclusion 
  of a suitable extension in their response, rather than aborting.
  By comparison, a server should abort when a client-supplied
  extension is erroneous. 
%
Secondly, prior to authentication, active attackers can remove and 
  inject messages, hence, they can modify handshake messages. 
  Since an HMAC is computed over the entire handshake, such 
  modifications can typically be detected and endpoints can 
  abort. However, to quote RFC 8446, ``extreme care is needed 
  when the extension changes the meaning of messages sent in the 
  handshake phase.'' Thus, extensions should be designed to 
  prevent an active adversary from unduly influencing parameter 
  negotiation, i.e., endpoints should negotiate their preferred
  parameters, even in the presence of an adversary.
%
In addition, any interactions with early data must be defined.




\begin{tcolorbox}
Extensions are enumerated and instantiated by enum \code{SSLExtension} (Listing~\ref{lst:SSLExtension}), 
and class \code{SSLExtensions} (Listings~\ref{lst:SSLExtensions}--\ref{lst:SSLExtensionsB}) represents
a list of extensions.
\end{tcolorbox}

\lstinputlisting[
  float=tbp,
  widthgobble=0*0,
  linerange={
    38-38,%class def
    489-505,%constructor
    529-530,532-532,537-537,%produce
    539-540,542-542,547-547,%consumeOnLoad
    549-550,552-552,557-557,%consumeOnTrade
    685-685%closing brace
  },
  label=lst:SSLExtension,
  caption={[\code{SSLExtension} enumerates and instantiates extensions]
  \code{SSLExtension} enumerates and instantiates extensions. 
  Each instantiation defines a hexadecimal value (Line~495) and a name (Line~497). 
  Moreover, they define variable \code{networkProducer} of (interface) type 
  \code{HandshakeProducer} which is instantiated by a constant 
  \code{ThisNameExtension.messageNetworkProducer}, where \code{ThisName} corresponds 
  to extension \TLSField{this\_name} and \code{message} is an abbreviation of
  the message type, e.g., \code{ch} abbreviates \ClientHello. For instance, constants 
  \code{SupportedVersionsExtension.chNetworkProducer} 
  and \code{PreSharedKeyExtension.chNetworkProducer} are used for extensions
  \TLSsupportedVersions\ and \TLSpsk, respectively, for \ClientHello\ messages. 
  Variable \code{networkProducer} is used by method \code{produce} to instantiate extensions
  (Lines~529--537).
  Variables \code{onLoadConsumer} and \code{onTradeConsumer} of (interface) type 
  \code{ExtensionConsumer} and \code{HandshakeConsumer}, respectively, are defined 
  similarly. The former is used by method \code{consumeOnLoad} to consume extensions 
  (Lines~539--547) and the latter is used by method \code{consumeOnTrade} to 
  update the active context to include extensions \ifImplNotes\textcolor{red}{accurate?}\fi
  (Lines~549--557).
  Hence, enum \code{SSLExtension} is reliant
  on classes implementing interfaces \code{HandshakeConsumer}, \code{HandshakeProducer}, and 
  \code{ExtensionConsumer}, e.g., inner-classes of class \code{PreSharedKeyExtension}.
}]{listings/SSLExtension.java}

\lstinputlisting[
  float=tbp,
  widthgobble=0*0,
  linerange={
    39-41,
    47-49,
    51-51,
    53-61,
    77-82,
    91-93,
    98-100,
    102-104,
    114-119,
    121-123,
    207-208,
    209-209,
    228-230,
    233-234,
    238-240
  },
  label=lst:SSLExtensions,
  caption={[\code{SSLExtensions} produces and consumes extensions]
  Class \code{SSLExtensions} defines a map of extensions and their 
  associated data (Line~41). That map can be instantiated by method \code{produce} 
  (Lines~207--240) or during construction from an input stream (Lines~53-123).
}]{listings/SSLExtensions.java}


\lstinputlisting[
  float=tbp,
  widthgobble=0*0,
  linerange={
    132-134,
    163-164,
    169-170,
    175-177,
    197-197,
    201-202,    
    293-307,
    362-362
  },
  label=lst:SSLExtensionsB,
  caption={[\code{SSLExtensions}  produces and consumes extensions (cont.)]
  Class \code{SSLExtensions} (continued from 
  Listing~\ref{lst:SSLExtensions}) defines method \code{consumeOnLoad} to 
  consume received extensions (Lines~132--170), using method 
  \code{SSLExtension.consumeOnLoad} (Listing~\ref{lst:SSLExtension});
  \code{consumeOnTrade} to update the active context to include extensions
  \ifImplNotes\textcolor{red}{accurate?}\fi
  (Lines~175--202), using method \code{SSLExtension.consumeOnTrade} (Listing~\ref{lst:SSLExtension});
  and method \code{send} to write extensions and associated data to an output stream 
  (Lines~293--307).
}]{listings/SSLExtensions.java}


