\subsection{Authentication}\label{sec:handshakeAuth}

The handshake protocol concludes with unilateral authentication of the server.
(Client authentication is also possible, as discussed in Appendix~\ref{sec:CR}.)
For (EC)DHE-only key exchange, the server must send a \Certificate\ message
followed by a \CertificateVerify\ message (\S\ref{sec:CT}), immediately after an 
\EncryptedExtensions\ message (except when client authentication is requested).
Those messages are followed by a \Finished\ message (\S\ref{sec:FIN}). 
For PSK-based key exchange, the pre-shared key serves to authenticate the
handshake (without certificates), hence, \Certificate\ and \CertificateVerify\ 
messages are not sent, and the server only sends a \Finished\ message.\footnote{%
  RFC 8446 does not permit PSK-based key exchange with \Certificate\ and 
  \CertificateVerify\ messages from the server; (direct) certificate-based server 
  authentication is unsupported for PSK-based key exchange. (The specification notes 
  that future documents may support such authentication.) Certificate-based
  client authentication is compatible with PSK-based key exchange  
  (Appendix~\ref{sec:CR}).
}

\subsubsection{\Certificate\ and \CertificateVerify}\label{sec:CT}\label{sec:CV}

\ifPresentationNotes
\textcolor{red}{This section defines \Certificate\ and \CertificateVerify\ messages
  originating from servers. Some remarks are made with regards such messages originating from
  clients, especially when the design looks peculiar. E.g., the existence of field 
  \TLScertificateRequestContext\ makes little sense for \Certificate\ messages  
  originating from servers, because the field contains a zero-length identifier.
  To avoid (over) complicating the discourse, such details are kept to a minimum.
  Is it worth sign-posting that \Certificate\ and \CertificateVerify\ messages 
  originating from the client differ? I think not, that detail can be made 
  explicit in the appendix, if at all.}
\fi  

A \Certificate\ message contains a certificate (along with its certificate chain)
for authentication, and a \CertificateVerify\ message contains a 
signature (constructed with the private key corresponding to the public key in the certificate)
over a hash of the protocol's transcript, thereby, proving 
possession of the private key used for signing, hence, identifying the server.
%The former comprises of the following fields:

A \Certificate\ message comprises of the following fields:

\begin{description}

\item \TLScertificateRequestContext: A zero-length identifier. (A
  \Certificate\ message may also be sent in response to a 
  \CertificateRequest\ message during post-handshake authentication, 
  as discussed in Appendix~\ref{sec:CR}, in which case this field 
  echos the identifier used by the \CertificateRequest\ message.)

\item \TLScertificateList: A (non-empty) list of certificates and any associated
  extensions.   %(The list may be empty for \Certificate\ messages sent by the client.) 
%
  (Any extensions must respond to ones listed in the \ClientHello\ message.
  Moreover, an extension that applies to the entire chain should appear 
  in the first extension listed.)
  Certificates must be DER-encoded X.509v3 certificates, unless
  an alternative certificate type was negotiated (using extension 
  \TLSserverCertificateType). 
  The server's certificate 
  \ifPresentationNotes
  \textcolor{red}{\emph{the server's certificate} seems 
  a little ambiguous (a server may have many), but perhaps that's okay}
  \fi
  must appear first and every subsequent certificate should certify the 
  previous one (i.e., every subsequent certificate should contain a 
  signature -- using the private key corresponding to the certificate's
  public key -- over the previous certificate's public key), hence, the 
  list is a certificate chain. That first 
  \begin{comment}
  \sout{certificate must be signed using 
  an algorithm amongst those offered by the client 
  (\ClientHello.\TLSsignatureAlgorithms) and the}
  \textcolor{red}{I can't find any evidence to support that -- my mistake?}
  \end{comment}
  certificate's public key 
  should be compatible with \begin{comment}\sout{that algorithm}\end{comment}
  an algorithm amongst those offered, by the client, for \CertificateVerify\
  messages (i.e., advertised by \ClientHello.\TLSsignatureAlgorithms).
  \ifSpecNotes
  \textcolor{red}{
    The spec requires that the certificate's public key be compatible
    with the selected authentication algorithm from the client's 
    "signature\_algorithms" extension, but does not require the server
    to select such an algorithm.
  }
  \fi
  Any remaining certificates' public keys should be compatible with an algorithm 
  offered %by the client 
  for \Certificate\ messages (i.e., those advertised 
  by extension \TLSsignatureAlgorithmsCert\ if present and extension 
  \TLSsignatureAlgorithms\ otherwise).
  (When a certificate chain cannot be constructed from compatible algorithms, 
  the chain may rely on algorithms not offered by the client, 
  except for SHA-1, which must not be used, unless offered.)
  All certificates must (explicitly) permit signature verification (whenever
  certificates include a Key Usage extension).
  (Self-signed certificates or trust anchors may be signed 
  with any algorithm, trust anchor certificates 
  may be omitted when they are known to be in the client's 
  possession, and, for raw public keys, the list must contain 
  \ifSpecNotes 
  \textcolor{red}{
  The spec requires \emph{no more than one certificate}.
  The spec also requires a non-empty list for servers. 
  So, we can infer \emph{exactly one} here [for the case of servers].
  }
  \fi
  exactly one certificate.)

\end{description}

\noindent
A server's \Certificate\ message is consumed by the client, which aborts
with a \TLSdecodeError\ alert if the \Certificate\ message is empty
and with a \TLSbadCertificate\ alert if a certificate relies on MD5, moreover,
it is recommended that a client also aborts with a \TLSbadCertificate\ alert if
a certificate relies on SHA-1. The client may validate certificates using 
procedures beyond the scope of TLS. (The TLS 1.3 specification
cites RFC~5280 as a reference for validation procedures.)

A \CertificateVerify\ message comprises of the following fields:

\begin{description}

\begin{sloppypar}
\item \TLSalgorithm: A signing algorithm, which must be amongst those
  offered by the client (\ClientHello.\TLSsignatureAlgorithms), unless 
  unless a certificate chain cannot be constructed from compatible algorithms.
\end{sloppypar}
  
\item \TLSsignature: A signature, produced by the aforementioned algorithm, over the concatenation of: 
  0x20 repeated 64 times, string ``TLS 1.3, server CertificateVerify'',
  0x00, and the transcript hash (\S\ref{sec:hkdf:transcript}).

\end{description}

\noindent
A server's \Certificate\ message is consumed by the client, which aborts
with a  \TLSbadCertificate\ alert if the signature does not verify.

\begin{tcolorbox}
\Certificate\ and \CertificateVerify\ messages are implemented, produced, and consumed by 
inner-classes of class \code{CertificateMessage} (Listings~\ref{lst:CertificateMessage}--\ref{lst:CertificateMessageD}) 
and \code{CertificateVerify} (Listings~\ref{lst:CertificateVerify}--\ref{lst:CertificateVerifyD}), respectively.
\end{tcolorbox}

\lstinputlisting[
  float=tbp,
  linerange={
    732-734, %%CertificateEntry
    736-739,
    777-777,
    782-784, %%T13CertificateMessage
    786-798,
    913-913
  },
  label=lst:CertificateMessage,
  caption={[\code{CertificateMessage.T13CertificateMessage} defines \Certificate]
  Class \code{CertificateMessage.T13CertificateMessage} defines the two fields of a 
  \Certificate\ message (Lines~783--784) and a constructor to instantiate them
  (Lines~786--798), where the latter field is defined over a list of pairs, comprising a 
  certificate and any associated extensions (Lines~732--777).
  A further (omitted) constructor is defined to instantiate a \Certificate\ message 
  from an input buffer.
}]{listings/CertificateMessage.java}%


\lstinputlisting[
  float=tbp,
  linerange={
    73-74,
    %732-739, %%CertificateEntry
    %777-777,
    %782-784, %%T13CertificateMessage
    %786-798,
    %800-806,
    %913-913,
    918-919, %%T13CertificateProducer
    926-927,
    %929-929,
    %934-935,
    %937-937,
    929-937,
    939-941,   %%onProduceCertificate
    943-943,
    955-956,
    963-971,
    974-975,
    %977-981,       %%stapling
    983-995,
    1001-1003,
    1005-1007,
    1126-1126
  },
  label=lst:CertificateMessageB,
  caption={[\code{CertificateMessage.T13CertificateProducer} produces \Certificate]
  Class \code{CertificateMessage.T13CertificateProducer} defines method \code{produce} to write 
  (to an output stream) a \Certificate\ message, originating from a client (Lines~931--932) 
  or server (Lines~934--935). 
  For the latter, a private key and authenticating certificates are wrapped inside an instance 
  of class \code{X509Authentication.X509Possession} (Lines~943--955), using method \code{choosePossession} 
  (Listing~\ref{lst:CertificateMessageC}); the server's active context is updated to include that 
  private key and associated certificates (Lines~964--967); a \Certificate\ message is constructed
  from the certificates (Lines~968--975); and the message is written to an output stream 
  (Lines~1002--1003).
}]{listings/CertificateMessage.java}%


\lstinputlisting[
  float=tbp,
  linerange={
    1009-1011,
    %1021-1021,
    1022-1022,
    1031-1035,
    1043-1044,
    1046-1047,
    1053-1054,
    1056-1057,
    1062-1063,
    1065-1066,
    1071-1072
  },
  label=lst:CertificateMessageC,
  caption={[\code{CertificateMessage.T13CertificateProducer} produces \Certificate\ (cont.)]
  Class \code{CertificateMessage.T13CertificateProducer} (omitted from Listing~\ref{lst:CertificateMessageB}) 
  defines method \code{choosePossession} to iterate over the client offered signature 
  algorithms for certificates (defined by extension \TLSsignatureAlgorithmsCert, 
  or \TLSsignatureAlgorithms\ if the former is absent), which class 
  \code{CertSignAlgsExtension.CHCertSignatureSchemesUpdate} 
  (respectively \code{SignatureAlgorithmsExtension.CHSignatureSchemesUpdate})
  assigns to variable \code{hc.peerRequestedCertSignSchemes}; disregard
  algorithms not offered for signing \CertificateVerify\ requests
  (Lines~1033--1044), unsupported algorithms (Lines~1046--1054), 
  or algorithms for which no suitable private key is available (1056--1063); and return a private key 
  for the first suitable algorithm (Line~1065), or null if no such key
  exists (Line~1071).
}]{listings/CertificateMessage.java}%


\lstinputlisting[
  float=tbp,
  linerange={
    71-72,
    %732-739, %%CertificateEntry
    %777-777,
    %782-784, %%T13CertificateMessage
    %786-798,
    %800-806,
    1131-1131, %%T13CertificateConsumer
    1138-1139,
    1141-1141,
    1144-1146,
    1151-1152,
    1157-1159,
    1186-1187,
    1194-1200,
    1202-1204,
    1207-1207,
    1209-1212,
    %%
    1369-1369
  },
  label=lst:CertificateMessageD,
  caption={[\code{CertificateMessage.T13CertificateConsumer} consumes \Certificate]
  Class \code{CertificateMessage.T13CertificateConsumer} 
  defines method \code{consume} to instantiate a \Certificate\ message from an 
  input buffer (Line~1145) and consume the message as originating from a server (Line~1151)
  or client (Lines~1157). For the former, certificates are checked (Lines~1203--1204) and
  the active context is updated (Lines~1209--1211).
}]{listings/CertificateMessage.java}%


\lstinputlisting[
  float=tbp,
  linerange={
    793-795,
    813-813,
    823-823,
    825-826,
    844-844,
    854-854,
    857-858,
    860-861,
    863-870,
    878-910,
    1030-1030
  },
  label=lst:CertificateVerify,
  caption={[\code{CertificateVerify.T13CertificateVerifyMessage} defines \CertificateVerify]
  Class \code{CertificateVerify.T13CertificateVerifyMessage} defines the two fields of a 
  \CertificateVerify\ message (Lines~858 \&~861) and constructors to instantiate them from parameters
  (Lines~863--910) or an input buffer (Listing~\ref{lst:CertificateVerifyB}). The former instantiates 
  the first field with the 
  chosen signature algorithm (Lines~867--870); derives the string over which to compute the signature
  (Lines~878--890), using constant \code{serverSignHead} (Lines~764--823) for messages originating  
  from a server, and constant \code{clientSignHead} (Lines~825--854) for messages originating from
  a client, where bytes used to construct those contents are omitted for brevity; and instantiates
  the second field as a signature over that string (Lines~892--909).
}]{listings/CertificateVerify.java}%

\lstinputlisting[
  float=tbp,
  linerange={
    912-914,
    925-927,
    941-948,
    957-991
  },
  label=lst:CertificateVerifyB,
  caption={[\code{CertificateVerify.T13CertificateVerifyMessage} defines \CertificateVerify\ (cont.)]
  Class \code{CertificateVerify.T13CertificateVerifyMessage} (omitted from Listing~\ref{lst:CertificateVerify})
  defines a constructor which instantiates a \CertificateVerify\ message from an input buffer, parametrising 
  the first field with the chosen signature algorithm (Lines~926--927) and the second with the signature
  (Line~957), if the signature verifies (Lines~974--980) with respect to the expected string
  (Lines~959--971).
}]{listings/CertificateVerify.java}%




\lstinputlisting[
  float=tbp,
  linerange={
    60-61,
    1035-1036,
    1043-1046,
    1048-1053,
    1066-1073,
    1075-1078,  
    1084-1086,
    1088-1090,
    1108-1108
  },
  label=lst:CertificateVerifyC,
  caption={[\code{CertificateVerify.T13CertificateVerifyProducer} produces \CertificateVerify]
  Class \code{CertificateVerify.T13CertificateVerifyProducer} defines method \code{produce} to
  write (to an output stream) a \CertificateVerify\ message, originating from a client (Lines~1067--1068) or server (Lines~1070--1071).  
  \ifImplNotes
  \textcolor{red}{The implementations of method \code{onProduceCertificateVerify} parameterised 
  on \code{ServerHandshakeContext} and \code{ClientHandshakeContext} are identical up to 
  variables \code{shc} and \code{chc}, and string \code{"server"} and \code{"client"}, refactoring
  could eliminate unnecessary code.}
  \textcolor{red}{Reported to security-dev@openjdk.java.net on 27 May 2020}
  \fi
  For the latter, a \CertificateVerify\ message is constructed (Lines~1077-1078)
  and written to an output stream (Lines~1085-1086).
}]{listings/CertificateVerify.java}%


\lstinputlisting[
  float=tbp,
  linerange={
    58-59, 
    1113-1114,
    1121-1122,
    1124-1126,
    1141-1142    
  },
  label=lst:CertificateVerifyD,
  caption={[\code{CertificateVerify.T13CertificateVerifyConsumer} consumes \CertificateVerify]
  Class \code{CertificateVerify.T13CertificateVerifyConsumer} defines method \code{consume} to 
  instantiate a \CertificateVerify\ message from an input buffer (Line~1125--1126), checking 
  validity of the message's signature as a side effect.
}]{listings/CertificateVerify.java}%



%%%%%%%%%%%%%%%%%%%%%%%%%%%%%%%%%%%%%%%%%%%
%%%%%%%%%%%%%%%%%%%%%%%%%%%%%%%%%%%%%%%%%%%
%%%%%%%%%%%%%%%%%%%%%%%%%%%%%%%%%%%%%%%%%%%



\subsubsection{\Finished}\label{sec:FIN}

The handshake protocol concludes with a \Finished\ message, which provides key confirmation, 
binds the server's identity to the exchanged keys (and the client's identity, if client 
authentication is used), and, for PSK-based key exchange, authenticates the handshake. 
A \Finished\ message comprises of the following field:

\begin{description}
\item \TLSverifyData: An HMAC over the entire handshake.
\end{description}

\noindent
The HMAC is computed as  
\[
  \HMACHash(\TLSfinishedKey,\TranscriptHash(\TLSHandshakeMessage))
\]
where \TLSHandshakeMessage\ is a concatenation of the protocol's messages (\S\ref{sec:hkdf:transcript}),
\begin{multline*}
  \TLSfinishedKey =   \HKDFExpandLabel(\TLShandshakeTrafficSecret,\\ ``finished", ``", \HashLength),
\end{multline*}
%
%traffic secret $S$ is \TLShandshakeTrafficSecret\ when the \Finished\ message concludes an initial 
%handshake and \TLSapplicationTrafficSecretN\ when concluding post-handshake authentication, and 
%\TLSField{[sender]} is either \TLSField{client} or \TLSField{server}.
%\ifSpecNotes
%\textcolor{red}{
%  Q: In the final instance, shouldn't \TLSapplicationTrafficSecret[client]{N}
%  be \TLSapplicationTrafficSecretN? A: No, it should not. During post-handshake authentication, only the client sends a finished message (as far
%  as I can tell).
%}
%\fi
%and traffic secret $S$ is \TLSserverHandshakeTrafficSecret\ when the \Finished\ message originates from
%a server to conclude an initial handshake, \TLSclientHandshakeTrafficSecret\ when originating 
%from a client to conclude an initial handshake, and \TLSapplicationTrafficSecret[client]{N} when 
%concluding post-handshake authentication. 
%(Post-handshake authentication is only concerned with updating the client's application-traffic key, 
%for the purposes of blinding the client's identity to that key. Hence, secret \TLSfinishedKey\ is not 
%concerned with traffic secret \TLSapplicationTrafficSecret[server]{N}. Beyond traffic keys, a key 
%established by a \NewSessionTicket\ message, sent after post-handshake authentication, will also be 
%bound to the client's identity.)
%
%% Defer post-handshake aspects to the appendix.
%
and \TLSField{[sender]} is \TLSField{server} when the \Finished\ message originates from
a server to conclude a handshake and \TLSField{client} when originating from a client. %to conclude a handshake. 


A \Finished\ message is first sent by the server (immediately after a \CertificateVerify\ message 
for (EC)DHE-only key exchange and immediately after an \EncryptedExtensions\ message for PSK-based 
key exchange). That message is consumed by the client, which recomputes the HMAC (using secret
\TLSserverHandshakeTrafficSecret) and checks that it matches the \Finished\ message's HMAC 
(\Finished.\TLSverifyData), terminating the connection with a \TLSdecryptError\ alert if the check 
fails. A client that successfully consumes a server's \Finished\ message responds with its own \Finished\
message, which is similarly consumed by the server (albeit using secret \TLSclientHandshakeTrafficSecret).
(That message is preceded by client generated \Certificate\ and \CertificateVerify\ messages, if 
client authentication is used.) Once endpoints have successfully consumed \Finished\ messages, 
(encrypted) application data may be exchanged. Moreover, a server may send (encrypted) application 
data immediately after sending its \Finished\ message (i.e., without consuming a \Finished\ message), 
albeit, since \ClientHello\ messages may be replayed, any such data is sent without assurance of the 
client's liveness (nor identity). 

\begin{tcolorbox}
\Finished\ messages are implemented, produced, and consumed by inner-classes of class \code{Finished} 
(Listings~\ref{lst:Finished}--\ref{lst:T13FinishedConsumer:ServerSide}).
    \ifImplNotes
\textcolor{blue}{Classes \code{Finished.T13FinishedProducer} and \code{Finished.T13FinishedConsumer}
  seem to contain some obsolete secure renegotiation code.} 
    \fi
\ifPresentationNotes
\textcolor{red}{Perhaps separate the listings into production/consumption, and establishing keys, as we 
did earlier. Or perhaps not, the reader should be able to consume it all (no pun intended) at this stage.}
\fi
\end{tcolorbox}

\lstinputlisting[
  float=tbp,
  linerange={
    69-70,
    72-73,
    75-76,
    78-84,
    86-87,
    89-98,
    106-107,
    109-114,
    117-123, 
    156-156
  },
  label=lst:Finished,
  caption={[\code{Finished.FinishedMessage} defines \Finished]
  Class \code{Finished.FinishedMessage} defines the one field of a \Finished\ message (Line~70)
  and two constructors to instantiate it. The first constructor parameterises the field with an HMAC
  it constructs (Lines 72--87) and the second parses an HMAC from an input buffer (Lines 92--107),
  recomputes the expected HMAC itself (Lines~109--118), and checks that the HMACs match
  (Lines~119--122). The HMACs are (indirectly) computed using method 
  \code{T13VerifyDataGenerator.createVerifyData} (Listing~\ref{lst:FinishedB}).
}]{listings/Finished.java}%

\lstinputlisting[
  float=tbp,
  linerange={
    326-329,
    332-333,
    335-358
  },
  label=lst:FinishedB,
  caption={[\code{Finished.T13VerifyDataGenerator} defines \Finished\ (cont.)]
  Class \code{Finished.T13VerifyDataGenerator} defines method \code{createVerifyData}
  to compute HMACs for \Finished\ messages. 
  \ifImplNotes
  \textcolor{red}{Comment on Line 325 doesn't match the class}
  \fi
  That method computes variable \code{finishedSecret} by indirect application
  of method \code{HKDF.expand} to inputs including secret \code{context.baseReadSecret} or
  \code{context.baseWriteSecret}, and \TLSHKDFLabelExt, which is computed over the 
  negotiated hash function's output length, 
  label $``\textrm{tls13\textvisiblespace{}finished}"$, and null ASCII character 0x00,
  using class \code{SSLBasicKeyDerivation} to apply method \code{HKDF.expand}.
  (Class \code{SSLBasicKeyDerivation} uses method \code{createHkdfInfo}
  to handle concatenation and is reliant on method \code{Record.putBytes8} to introduce 
  0x00. This differs from a similar application of method \code{HKDF.expand} by 
  class \code{SSLTrafficKeyDerivation.T13TrafficKeyDerivation}, which introduces 
  0x00 itself.)
  \ifImplNotes
  \textcolor{red}{Classes \code{SSLBasicKeyDerivation} and 
  \code{SSLTrafficKeyDerivation.T13TrafficKeyDerivation} are almost identical, 
  and some refactoring could eliminate one of the classes.}
  \fi
}]{listings/Finished.java}%

\begin{comment}
\lstinputlisting[
  float=tbp,
  widthgobble=0*0,
  linerange={
    35-38,
    40-45,
    48-58,
    81-81
  },
  label=lst:SSLBasicKeyDerivation,
  caption={[\code{SSLBasicKeyDerivation} defines \Finished\ (cont.)]
  Class \code{SSLBasicKeyDerivation} defines three fields (Lines~36--38),
  a constructor to instantiate them (Lines~40-45), and method \code{deriveKey}
  to return an HMAC computed by application of method \code{HKDR.expand}
  (Lines~52--53) to inputs including \TLSHKDFLabelExt, which is computed 
  (Line~44) over some label, context, and length, using static method 
  \code{SSLBasicKeyDerivation.createHkdfInfo} to handle concatenation.
    \textcolor{red}{Drop this figure, pushing details into the previous figure}
}]{listings/SSLBasicKeyDerivation.java}%
\end{comment}

\lstinputlisting[
  float=tbp,
  widthgobble=0*0,
  linerange={
    63-64,
    630-631,
    638-639,  
    641-649,
    738-741,
    743-743,
    749-751,
    753-754,
    762-763,
    772-783
  },
  label=lst:T13FinishedProducer,
  caption={[\code{Finished.T13FinishedProducer} produces server-side \Finished]
  Class \code{Finished.T13FinishedProducer} defines method \code{produce}
  to write (to an output stream) a \Finished\ message, originating from a 
  client or a server, and to establish \TLSMasterSecret\ when the message 
  originates from such a server. For messages originating 
  from servers, processing  
  proceeds with method \code{onProduceFinished}, parameterised by the 
  server's active context. That method updates the transcript hash's 
  digest to include all handshake protocol messages (Line~741), 
  instantiates and outputs a \Finished\ message (Lines~743--751), and 
  establishes \TLSMasterSecret\ (Lines~782--783). Variable \code{shc.handshakeKeyDerivation} 
  (Line~754) is assigned by class \code{ServerHello.T13ServerHelloProducer} (Listing 27)
  as an instance of class \code{SSLSecretDerivation}, parameterised by 
  \TLSHandshakeSecret, hence, the salt necessary to establish 
  \TLSMasterSecret\ is correctly derived (Line~774), as-is the necessary 
  \HashLength-length string of zeros (Lines~780--781).
}]{listings/Finished.java}%

\lstinputlisting[
  float=tbp,
  widthgobble=0*0,
  linerange={
    785-786,
    788-811,
    814-815,
    824-831
  },
  label=lst:T13FinishedProducerB,
  caption={[\code{Finished.T13FinishedProducer} produces server-side \Finished\ (cont.)]
  Class \code{Finished.T13FinishedProducer} defines method \code{onProduceFinished}
  (continued from Listing~\ref{lst:T13FinishedProducer}) to derive traffic   
  secret \TLSserverApplicationTrafficSecret\ from an instance of  
  class \code{SSLSecretDerivation}, parameterised by \TLSMasterSecret\ (Lines~785--790); 
  constructs an instance of \code{SSLTrafficKeyDerivation.T13TrafficKeyDerivation}
  from that secret (Lines~791--792); and uses that instance to derive the corresponding 
  traffic keys \TLSField{server\_write\_key} (Lines~793--794) and
  \TLSField{server\_write\_iv} (Lines~795--796), used to encrypt (and write) outgoing 
  traffic (Lines~797--807). Moreover, the method prepares the server's active context
  for the client's response (Lines~825--826).
}]{listings/Finished.java}%

\lstinputlisting[
  float=tbp,
  widthgobble=0*0,
  linerange={
    61-62,
    836-836,
    843-844,
    846-854,
    856-858,
    874-884,
    892-893,
    909-920
  },
  label=lst:T13FinishedConsumer,
  caption={[\code{Finished.T13FinishedConsumer} consumes server-generated \Finished]
  Class \code{Finished.T13FinishedConsumer} defines method \code{consume}
  to read (from an input buffer) a \Finished\ message, originating from a 
  client or a server, and to establish \TLSMasterSecret\ when the message 
  originates from such a server. For messages 
  originating from servers, processing proceeds with method \code{onConsumeFinished}, 
  parameterised by the client's active context. That method instantiates 
  a \Finished\ message from the input buffer (Line~858), updates the transcript hash's 
  digest to include all handshake protocol messages (Line~883), and 
  establishes \TLSMasterSecret\ (Lines~919--920). (Computations are similar to 
  Listing~\ref{lst:T13FinishedProducer} and refactoring could eliminate unnecessary 
  code.) 
}]{listings/Finished.java}%

\lstinputlisting[
  float=tbp,
  widthgobble=0*0,
  linerange={
    922-947,
    950-951,
    953-963,
    965-972
  },
  label=lst:T13FinishedConsumerB,
  caption={[\code{Finished.T13FinishedConsumer} consumes server-generated \Finished\ (cont.)]
  Class \code{Finished.T13FinishedConsumer} defines method \code{consume}
  (continued from Listing~\ref{lst:T13FinishedConsumer}) to derive traffic   
  secret \TLSserverApplicationTrafficSecret\ (Lines~922--927) and corresponding 
  traffic keys \TLSField{server\_write\_key} (Lines~930--931) and
  \TLSField{server\_write\_iv} (Lines~932--933), used to decrypt (and read) incoming 
  traffic (Lines~934--943). (Computations are similar to Listing~\ref{lst:T13FinishedProducerB} 
  and refactoring could eliminate unnecessary code.) Moreover, the method updates the 
  client's active context to include a producer for \Finished\ messages (Lines~956--957);
  constructs an array of producers clients might use during the remainder of the 
  handshake protocol, namely, produces for messages \Certificate, \CertificateVerify, 
  and \Finished, in the order that they might be used (Lines~958--963); and uses 
  those producers to produce messages when the active context includes the producer
  (Lines~965--971). Since a \Finished\ message producer is included, a \Finished\
  message is always produced, using class \code{Finished.T13FinishedProducer}
  (Listing~\ref{lst:T13FinishedProducer} \&~\ref{lst:T13FinishedProducer:ClientSide}).
}]{listings/Finished.java}%


\lstinputlisting[
  float=tbp,
  widthgobble=0*0,
  linerange={
    651-654,
    656-656,
    662-664,
    672-673,
    681-682,
    691-712,
    714-714,
    717-718,
    %727-736
    734-736
  },
  label=lst:T13FinishedProducer:ClientSide,
  caption={[\code{Finished.T13FinishedProducer} produces client-side \Finished\ (cont.)]
  Class \code{Finished.T13FinishedProducer} defines method \code{onProduceFinished} 
  parameterised by a client's active context (omitted from Listing~\ref{lst:T13FinishedProducer})
  to write (to an output stream) a \Finished\ message originating from a 
  client, and to derive traffic secret \TLSclientApplicationTrafficSecret\ (Lines~692--693) 
  and corresponding traffic keys \TLSField{client\_write\_key} (Lines~698--699) and
  \TLSField{client\_write\_iv} (Lines~700--701), used to encrypt (and write) outgoing
  traffic (Lines~702--712). %Moreover, the method \textcolor{red}{...XXX...}
}]{listings/Finished.java}%

\lstinputlisting[
  float=tbp,
  widthgobble=0*0,
  linerange={
    974-976,
    991-995,
    1003-1004,
    %1013-1019,
    1020-1040,
    %1041-1048,
    1049-1049,
    1052-1053,
    %1073-1073,
    1075-1075
  },
  label=lst:T13FinishedConsumer:ServerSide,
  caption={[\code{Finished.T13FinishedConsumer} consumes client-generated \Finished\ (cont.)]
  Class \code{Finished.T13FinishedConsumer} defines method \code{onConsumeFinished} 
  parameterised by a server's active context (omitted from Listing~\ref{lst:T13FinishedConsumer})
  to read (from an input buffer) a \Finished\ message originating from a 
  client, and to derive traffic secret \TLSclientApplicationTrafficSecret\ (Lines~1022--1023) 
  and corresponding traffic keys \TLSField{client\_write\_key} (Lines~1027--1028) and
  \TLSField{client\_write\_iv} (Lines~1029--1030), used to decrypt (and read) incoming
  traffic (Lines~1031--1040). (Computations are similar to Listing~\ref{lst:T13FinishedProducer:ClientSide} 
  and refactoring could eliminate unnecessary code.)
  \ifImplNotes
  \textcolor{blue}{Not only computation similar to Listing~\ref{lst:T13FinishedProducer:ClientSide}, 
    but computation is (unsurprisingly) similar between other traffic key derivations. This could be 
    resolved by refactoring.}
  \textcolor{red}{Reported to security-dev@openjdk.java.net on 27 May 2020}
  \fi
}]{listings/Finished.java}%

Traffic secrets \TLSserverApplicationTrafficSecret\ and \TLSclientApplicationTrafficSecret\
are used to derive application-traffic keys to protect application data. 

\begin{comment}
\sout{Both secrets are 
computed from the protocol's transcript, but at different stages of the transcripts
evolution. Indeed, the former traffic secret is computed from the transcript after the server
generates its \Finished\ message (and when the transcript concludes with that server-generated 
\Finished\ message), whereas the latter traffic secret is computed after the client generates 
its \Finished\ message (and when the transcript concludes with that client-generated \Finished\
message).} \textcolor{red}{That's false according to the spec: Figure~\ref{fig:keyDerivation} 
shows that both secrets are computed from transcripts ending with a server-generated \Finished\ 
message. But, the code suggests the crossed-out text is true. Truthify.}
%%
%% I have confirmed that both secrets are computed from transcripts ending with a 
%% server-generated \Finished\ message. I haven't dug into the code too deeply. My 
%% guess is that 

\end{comment}

\subsubsection*{Application data}

TLS protects application-layer communication independently of specific applications.
%How applications should use TLS is unspecified. 
Independence %(between applications and TLS) 
is readily apparent from the specification: There 
is no mention of interaction between applications and TLS. Designers and implementors
must decide for themselves how to use TLS within their applications. For instance,
when to initiate a handshake and how to validate certificates.



