\section{Compatibility mode}\label{sec:compatibilityMode}

%%There's a nice write-up in https://blog.cloudflare.com/why-tls-1-3-isnt-in-browsers-yet/

\begin{sloppypar}
TLS 1.3 removed \TLSlegacySessionId\ and \ChangeCipherSpec, which have been present since 
SSLv3. Their removal caused an intolerable increase in connection failures. To reduce 
failures, implementations are permitted to make TLS 1.3 look more like TLS 1.2: A client 
generates a 32-byte \TLSlegacySessionId. A server sends a dummy \ChangeCipherSpec\ 
immediately after its \ServerHello\ or \HelloRetryRequest\ message. A client also sends 
a dummy \ChangeCipherSpec, either immediately before their second flight (e.g., after 
consuming a \ServerHello\ or \HelloRetryRequest\ message) when there's no early data, or 
immediately after their (first) \ClientHello\ otherwise.
\end{sloppypar}

\begin{tcolorbox}
Class \code{ClientHello.ClientHelloKickstartProducer} generates a 32-byte \TLSlegacySessionId\ 
(Lines 557--572, omitted from Listing~\ref{lst:ClientHelloKickstartProducer}). 
Class \code{ServerHello.T13ServerHelloProducer} calls method \code{changeWriteCiphers}
(Lines 671~\& 672, Listing~\ref{lst:T13ServerHelloProducerD}) to produce a dummy \ChangeCipherSpec, 
which is queued for transmission before further data.
Class \code{ServerHello.T13ServerHelloConsumer} similarly calls \code{changeWriteCiphers}. 
That class also makes the active context ready to receive and discard the server's dummy 
\ChangeCipherSpec\ (Lines 1335--1337, omitted from Listing~\ref{lst:T13ServerHelloConsumer}).
Class \code{ClientHello.T13ClientHelloConsumer} does the same for the client's dummy \ChangeCipherSpec.
Compatibility mode can be disabled with \code{System.setProperty("jdk.tls.client.useCompatibilityMode", "false")}.
\end{tcolorbox}

\begin{sloppypar}
Compatibility mode is partially negotiable: A client decides whether to provide a non-empty
\TLSlegacySessionId\ (the server echo's the client's value). A client may send a dummy 
\ChangeCipherSpec\ in-between sending a \ClientHello\ message and receiving a \Finished\ 
message. For an empty \TLSlegacySessionId, a server may similarly send a dummy 
\ChangeCipherSpec\ in-between receiving \ClientHello\ and \Finished\ messages. 
(For non-empty \TLSlegacySessionId, the server must send a dummy \ChangeCipherSpec\
as described above.) \ChangeCipherSpec\ must be dropped without processing.
\end{sloppypar}

\begin{comment}
TL;DR. A complete trace differs from our opening figure (as described above):

        -----CH----->
        <-----SH-----
        <----CCS-----
        <--EE...FIN-- 
        -----CCS---->
        -----FIN---->
        <----NST-----

Verbose. Digging in:

javax.net.ssl|DEBUG|0B|Thread-0|2022-05-25 11:25:47.362 CEST|SSLEngineOutputRecord.java:529|WRITE: TLSv1.3 handshake, length = 378              -----CH----->
javax.net.ssl|DEBUG|0C|Thread-1|2022-05-25 11:25:47.390 CEST|SSLEngineInputRecord.java:214|READ: TLSv1.2 handshake, length = 378                             -----CH----->
javax.net.ssl|DEBUG|0D|Thread-2|2022-05-25 11:25:47.468 CEST|EncryptedExtensions.java:137|Produced EncryptedExtensions message 
javax.net.ssl|DEBUG|0D|Thread-2|2022-05-25 11:25:47.582 CEST|CertificateMessage.java:1022|Produced server Certificate message 
javax.net.ssl|DEBUG|0D|Thread-2|2022-05-25 11:25:47.636 CEST|CertificateVerify.java:1111|Produced server CertificateVerify handshake message 
javax.net.ssl|DEBUG|0D|Thread-2|2022-05-25 11:25:47.639 CEST|Finished.java:767|Produced server Finished handshake message 
javax.net.ssl|DEBUG|0C|Thread-1|2022-05-25 11:25:47.646 CEST|SSLEngineOutputRecord.java:529|WRITE: TLSv1.3 handshake, length = 122                           <-----SH-----
javax.net.ssl|DEBUG|0C|Thread-1|2022-05-25 11:25:47.649 CEST|SSLEngineOutputRecord.java:529|WRITE: TLSv1.3 change_cipher_spec, length = 1                    <----CCS-----
javax.net.ssl|DEBUG|0B|Thread-0|2022-05-25 11:25:47.650 CEST|SSLEngineInputRecord.java:214|READ: TLSv1.2 handshake, length = 122                <-----SH-----
javax.net.ssl|DEBUG|0C|Thread-1|2022-05-25 11:25:47.650 CEST|SSLEngineOutputRecord.java:529|WRITE: TLSv1.3 handshake, length = 1256                          <--EE...FIN--
javax.net.ssl|DEBUG|0B|Thread-0|2022-05-25 11:25:47.668 CEST|SSLEngineOutputRecord.java:529|WRITE: TLSv1.3 change_cipher_spec, length = 1       -----CCS---->
javax.net.ssl|DEBUG|0B|Thread-0|2022-05-25 11:25:47.670 CEST|SSLEngineInputRecord.java:214|READ: TLSv1.2 change_cipher_spec, length = 1         <----CCS-----    
javax.net.ssl|DEBUG|0C|Thread-1|2022-05-25 11:25:47.705 CEST|SSLEngineInputRecord.java:214|READ: TLSv1.2 change_cipher_spec, length = 1                      -----CCS---->
javax.net.ssl|DEBUG|0B|Thread-0|2022-05-25 11:25:47.711 CEST|SSLEngineInputRecord.java:214|READ: TLSv1.2 application_data, length = 1289        <--EE...FIN--
javax.net.ssl|DEBUG|0B|Thread-0|2022-05-25 11:25:47.758 CEST|SSLEngineOutputRecord.java:529|WRITE: TLSv1.3 handshake, length = 52               -----FIN---->
CLIENT, POST-WRAP: Status = OK HandshakeStatus = FINISHED
javax.net.ssl|DEBUG|0C|Thread-1|2022-05-25 11:25:47.760 CEST|SSLEngineInputRecord.java:214|READ: TLSv1.2 application_data, length = 85                       -----FIN---->
javax.net.ssl|DEBUG|0C|Thread-1|2022-05-25 11:25:47.767 CEST|SSLEngineOutputRecord.java:529|WRITE: TLSv1.3 handshake, length = 50                            <----NST-----
SERVER, POST-WRAP: Status = OK HandshakeStatus = FINISHED
javax.net.ssl|DEBUG|0B|Thread-0|2022-05-25 11:25:47.770 CEST|SSLEngineInputRecord.java:214|READ: TLSv1.2 application_data, length = 83          <----NST-----
CLIENT, POST-UNWRAP: Status = OK HandshakeStatus = FINISHED
\end{comment}
