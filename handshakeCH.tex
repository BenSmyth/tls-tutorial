\subsection{\ClientHello}\label{sec:handshakeCH}\label{sec:CH}

The handshake protocol is initiated by a \ClientHello\ message, %(specified in Appendix~\ref{sec:specification})
comprising the following fields:

\begin{description}

\item \TLSlegacyVersion: Constant 0x0303.
%
%\footnote{Previous versions of TLS used this field for the highest 
%  offered protocol version of the client, but experience has shown that
%  servers implement version negotiation rather poorly, in particular, 
%  some servers reject \ClientHello\ messages with a version number
%  higher than it supports. In TLS 1.3, the client offers protocol 
%  versions in an extension.}
%
(Previous versions of TLS used this field for the client's highest offered 
  protocol version. In TLS 1.3, protocol versions are offered in an 
  extension, as explained below.)
  % The field remains for backwards compatibility.)

\item \TLSrandom: A 32 byte nonce.

\item \TLSlegacySessionId: A zero-length vector, except for 
  compatibility mode (Appendix~\ref{sec:compatibilityMode}) 
  or to resume an earlier pre-TLS 1.3 session. 
  (Previous versions of TLS used this field for 
  ``session resumption.'' In TLS 1.3, that feature has been 
  merged with pre-shared keys.)
  %Again, the field remains for backwards compatibility.)

\item \TLScipherSuites: A list of offered \emph{symmetric cipher suites} in
  descending order of client preference, where a suite defines a value identifying an Authenticated Encryption 
  with Associated Data (AEAD) algorithm and a hash function (Table~\ref{table:suites}).\footnote{
    Support for cipher suite TLS\_AES\_128\_GCM\_SHA256 is mandatory (unless an implementation 
    explicitly opts out), and cipher suites TLS\_AES\_256\_GCM\_SHA384 and 
    TLS\_CHACHA20\_POLY1305\_SHA256 should also be supported.
  }


\begin{table}[tbp]
\caption{Symmetric cipher suites defined by a value identifying
  an AEAD algorithm and a hash function. Suites are named in the 
  format TLS\_AEAD\_HASH, where AEAD and HASH are replaced by the 
  corresponding algorithm and function names.}
\label{table:suites}

\centering

\begin{tabular}{l|l}
  Name                             & Value  \\ \hline
  TLS\_AES\_128\_GCM\_SHA256       & 0x1301 \\
  TLS\_AES\_256\_GCM\_SHA384       & 0x1302 \\
  TLS\_CHACHA20\_POLY1305\_SHA256  & 0x1303 \\
  TLS\_AES\_128\_CCM\_SHA256       & 0x1304\\
  TLS\_AES\_128\_CCM\_8\_SHA256    & 0x1305
\end{tabular}
\end{table}

\item \TLSlegacyCompressionModes: Constant 0x00. %byte set to zero. 
  (Previous versions of TLS used this field to list supported compression methods.
  In TLS 1.3, this feature has been removed.)
  % The field remains for backwards compatibility.)

\ifPresentationNotes\marginpar{Maybe mention that hex values are used in place of (extension) names}\fi
\item \TLSextensions: A list of \emph{extensions}, where an extension comprises 
  a name along with associated data. The list must contain at least extension
  \TLSsupportedVersions\ %(0x002B) \marginpar{drop hex names?}
  associated with a list of offered protocol versions 
  in descending order of client preference, minimally including constant 0x0304,
  denoting TLS 1.3.

\end{description}

\noindent
Legacy fields \TLSlegacyVersion, \TLSlegacySessionId, and \TLSlegacyCompressionModes\
are included for backwards compatibility.

\ifPresentationNotes
\marginpar{Perhaps elaborate on the produce/consume design}

\marginpar{I've overridden \texttt{listings.sty} (in \texttt{main-tls-intro.tex}) 
  to add vertical space (controlled by \texttt{tinyskip}) at the end of a 
  \texttt{linerange}, e.g., see Listing~\ref{lst:ClientHelloMessage}, between 
  Lines~71~\&~74, for example. I'm unsure whether it helps.}
\fi

\begin{tcolorbox}
The \ClientHello\ message is implemented by class 
\code{ClientHello.ClientHelloMessage} 
%(Listings~\ref{lst:ClientHelloMessage} \&~\ref{lst:ClientHelloMessageB}). 
(Listing~\ref{lst:ClientHelloMessage}).
Instances of that class are produced by class 
\code{ClientHello.Client\-Hello\-Kickstart\-Producer} 
(Listing~\ref{lst:ClientHelloKickstartProducer}), %\footnotemark\ 
which is instantiated as static constant \code{ClientHello.kickstart\-Producer}. That constant is used by method 
\code{SSLHandshake.kickstart}. 
\begin{comment}
Class \code{ClientHello} is reliant on classes 
\code{SSLExtensions} %(Listings~\ref{lst:SSLExtensions} \& \ref{lst:SSLExtensionsB}) 
and \code{SSLExtension} (Appendix~\ref{sec:extensions}) 
for extensions.
\end{comment}
\end{tcolorbox}

\begin{comment}
\footnotetext{Class \code{ClientHello.ClientHelloKickstartProducer} implements
  interface \code{SSLProducer}, which defines a single method --
    namely, \code{byte[] produce(ConnectionContext context) throws IOException}
  -- common to \ClientHello\ messages originating from the client, %\HelloRequest\ and 
  \NewSessionTicket\ messages originating from the server, and 
  \KeyUpdate\ messages originating from either endpoint. Parameter \code{context} defines the active context and may 
  be cast to children \code{ClientHandshakeContext} and \code{ServerHandshakeContext}, 
  as seen in Line~398 of Listing~\ref{lst:ClientHelloKickstartProducer}, for instance.
  %That context is updated during production of \ClientHello\ %, \HelloRequest,
  %and \NewSessionTicket\ messages, whereas production of \KeyUpdate\ messages
  %concludes by returning \textcolor{red}{``the encoded producing'' (reword)}.
  That context is updated during production of \ClientHello, \KeyUpdate\
  and \NewSessionTicket\ messages.
  \textcolor{red}{Are these details worth knowing?}
}
\end{comment}

\lstinputlisting[
  float=tbp,
  linerange={
    71-71,%class declaration
    74-76,79-83,%variable declaration
    84-88,90-93,100-100,102-102,105-107,%constructor
    160-162,165-167,185-190,192-192,194-200,%constructor
    312-316,318-322,326-329,%ClientHello output stream
    382-383%closing brace
  },
  label=lst:ClientHelloMessage,
  caption={[\code{ClientHello.ClientHelloMessage} defines \ClientHello]
  Class \code{ClientHello.ClientHelloMessage} defines the six fields 
  of a \ClientHello\ message (Lines~74--81) and constructors to instantiate them 
  from parameters (Lines~85--106) or an input buffer (Lines~160--200). The former 
  constructor does not populate the extensions field (and a call to method \code{SSLExtensions.produce},
  Listing~\ref{lst:SSLExtensions}, is required), whereas the latter may (Line~195--196).
  Method \code{send} (Lines~312--316) writes those fields to an output stream, using method 
  \code{sendCore} (Lines~318--328) to write all fields except the extensions 
  field, which is written by method \code{SSLExtensions.send} 
  (Listing~\ref{lst:SSLExtensions}, Lines~293--307). 
}]{listings/ClientHello.java}

\begin{comment}
\lstinputlisting[
  float=tbp,
  linerange={
%    281-289,%getEncodedCipherSuites()
    312-322,326-329,%ClientHello output stream
    382-383%closing brace
  },
  label=lst:ClientHelloMessageB,
  caption={Class \code{ClientHello.ClientHelloMessage} (continued from 
  Listing~\ref{lst:ClientHelloMessage}) defines method \code{send} (Lines~312--316) 
  to write \ClientHello\ message fields to an output stream, using method 
  \code{sendCore} (Lines~318--327) to write all fields except the extensions 
  field, which is written by method \code{SSLExtensions}.\code{send} 
  (Listing~\ref{lst:SSLExtensions}, Lines~293--307).
}]{listings/ClientHello.java}
\end{comment}

\lstinputlisting[
  float=tbp,
  linerange={
    50-51,%kickstartProducer
    387-388,%class declaration
    396-396,%method declaration
    398-398,%clientHandshakeContext
%    404-404,
    407-407,%sessionId
    410-410,%cipherSuites
%%%%%%%
    415-419,
%    420-426,
%    428-433, %chc.reservedServerCerts
%    435-436,
%    442-443,
    445-447, %sessionSuite
%    455-456,
%    458-462,
%    468-469,
%    513-513,
 %   522-529,
%    534-534,
    542-542,
    553-555,
%%%%%%
%    607-608,
    615-615,%clientVersion
%    617-617,
    618-621,%Construct ClientHelloMessage
    622-630,%cache clientRandom & clientHelloVersion
    635-638,%output
    639-646,%manage clientHandshakeContext
    653-657%closing braces etc.
  },
  label=lst:ClientHelloKickstartProducer,
  caption={[\code{ClientHello.ClientHelloKickstartProducer} produces \ClientHello]
  Class \code{ClientHello.ClientHelloKickstartProducer} defines 
  method \code{produce} which instantiates a \ClientHello\ message 
  (Lines~619--621), populates the extension field for the active context 
  (Lines~628--630), writes the \ClientHello\ message to an output 
  stream (Lines~637--638), and prepares the client's active context for the server's 
  response (Lines~624--625, 642, \& 645--646).
  The \ClientHello\ message parameterises \TLSlegacySessionId\ as a zero-length 
  byte array (Line~407); \ifImplNotes\textcolor{red}{truthify the following (EC)DHE
  vs. PSK}\fi \TLScipherSuites\ as the list of available cipher suites, for 
  (EC)DHE-only key exchange (Line~410), or as a list containing the cipher suite 
  %used by the previous session,
  associated with the pre-shared key, 
  for PSK-based key exchange (Line~542); and \TLSlegacyVersion\ as constant 0x0303 (Line~615). 
  (Prior versions of TLS are supported by the class and constants other than 
  0x0303 may be assigned to \TLSlegacyVersion. We omit those details for brevity.)
  The output stream is written-to using method \code{ClientHello.ClientHelloMessage.write}, 
  defined by parent class \code{SSLHandshake.HandshakeMessage}, which in 
  turn uses method \code{ClientHello.ClientHelloMessage.send}
  (Listing~\ref{lst:ClientHelloMessage}).
}]{listings/ClientHello.java}


The primary goal of the handshake protocol is to establish a channel that protects 
communication using one of the symmetric cipher suites offered by the client
and a key shared between the endpoints. That key is derived
from (secret) client and server key shares for 
%Ephemeral Diffie-Hellman key exchange over finite fields (DHE) or elliptic curves (ECDHE), 
(EC)DHE key exchange, from a (secret) pre-shared key for PSK-only key exchange, or by a combination of 
key shares and a pre-shared key for PSK with (EC)DHE key exchange. The 
desired key exchange mode determines which extensions to include: 
%
  For (EC)DHE, extensions \TLSsupportedGroups\ and \TLSkeyShare\ are included;
%
  for PSK-only, extensions \TLSpsk\ and \TLSpskModes\ must be included, and
  extensions \TLSsupportedGroups\ and \TLSkeyShare\ may be included to allow 
  the server to decline resumption and fall back to a full handshake; and 
%
  for PSK with (EC)DHE, all four of the aforementioned extensions are included.
%
Those extensions are associated with data:

\begin{description}



\item \TLSsupportedGroups\ and \TLSkeyShare: A list of offered %\sout{ephemeral}
  Diffie-Hellman groups for key exchange (\TLSsupportedGroups) 
  and key shares for some or all of those groups (\TLSkeyShare), in  descending 
  order of client preference. Groups may be selected over finite fields or
  elliptic curves.\footnotemark\ A key share for a particular group must 
  be listed in the same order that the group is listed. However, a key share 
  for a particular group may be omitted, even when a key share for a less
  preferred group is present. This situation could arise when a group is 
  new or lacking support, making key shares for such groups redundant
  and wasteful. An empty vector of key shares can be used to request 
  group selection from the server. (Servers respond with \HelloRetryRequest\
  messages when no key share is offered for the server selected group.)

\ifSpecNotes
\textcolor{red}{Why include both extensions? They overlap. In particular, \TLSsupportedGroups\
  is associated with a list of type \texttt{NamedGroup} and \TLSkeyShare\ is associated with
  list of pairs, where the first element of each pair is also of type \texttt{NamedGroup}
  (the second element contains the key share and a zero-length key share could be used
  when a key share isn't offered for the group), hence, the group is repeated.}
\fi


\footnotetext{Supported groups include: 
  Finite field groups defined in RFC~7919, namely, 
  %
    \TLSConstant{ffdhe2048} (0x0100), 
    \TLSConstant{ffdhe3072} (0x0101), 
    \TLSConstant{ffdhe4096} (0x0102),
    \TLSConstant{ffdhe6144} (0x0103), and
    \TLSConstant{ffdhe8192} (0x0104), and
  elliptic curve groups defined in either FIPS 186-4 
  or RFC~7748, namely, 
  %
    \TLSConstant{secp256r1} (0x0017), 
    \TLSConstant{secp384r1} (0x0018), 
    \TLSConstant{secp521r1} (0x0019),
    \TLSConstant{x25519} (0x001D), and
    \TLSConstant{x448} (0x001E).
  Supporting group \TLSConstant{secp256r1} is mandatory (unless an implementation explicitly 
  opts out), and group \TLSConstant{x25519} should also be supported.
}


\item \TLSpsk\ and \TLSpskModes: A list of offered pre-shared
  key identifiers (\TLSpsk) and a key exchange mode for each 
  (\TLSpskModes). (Further details on extension \TLSpsk\
  appear in Section~\ref{sec:NST}, after \NewSessionTicket\ messages -- which 
  establish pre-shared keys for subsequent connections -- are introduced.)
  At least one offered cipher suite should define a hash function associated 
  with at least one of the identifiers. 
  \ifSpecNotes
  \textcolor{red}{A stronger requirement seems desirable here, e.g., a hash 
  function associated with each identifier, but that's not what's required by 
  the spec (p28)}
  \fi
  Key exchange modes include PSK-only (\TLSPskKe) and PSK with (EC)DHE (\TLSPskDheKe).
  Extension \TLSpsk\ must be the last extension in the \ClientHello\ message.
  (Other extensions may appear in any order.)


\end{description}

\noindent
A further goal of the handshake protocol is unilateral authentication
of the server, which for (EC)DHE key exchange mode is achieved by inclusion of extensions 
\TLSsignatureAlgorithms\ and \TLSsignatureAlgorithmsCert\ (for PSK-only and
PSK with (EC)DHE, authentication is derived from the \Finished\ message), 
and associated data:

\begin{description}

\begin{sloppypar}
\item \TLSsignatureAlgorithms\ and \TLSsignatureAlgorithmsCert:
  A list of accepted signature algorithms in descending order of client 
  preference for \CertificateVerify\ messages (\TLSsignatureAlgorithms)
  and \Certificate\ messages (\TLSsignatureAlgorithmsCert).\footnotemark\
  (Extension \TLSsignatureAlgorithmsCert\ may be omitted in favour of extension 
  \TLSsignatureAlgorithms, when accepted algorithms for \Certificate\ and 
  \CertificateVerify\ messages coincide. In such cases,
  algorithms listed by extension \TLSsignatureAlgorithms\ apply to certificates too.)
\end{sloppypar}

\footnotetext{Supported signature algorithms include: 
  RSASSA-PKCS1-v1_5 (RFC8017) or RSASSA-PSS (RFC8017) with 
  a corresponding hash function, namely, SHA256, SHA384, or 
  SHA512;
  ECDSA (American National Standards Institute, 2005) with 
  a corresponding curve \& hash function, namely, 
  secp256r1 \& SHA256, secp384r1 \& SHA384, or 
  secp521r1 \& SHA512; and
  EdDSA (RFC8032). (RSASSA-PKCS1-v1_5 is only supported for 
  \Certificate\ messages.) 
  Supporting RSA-based signatures with SHA256 (for certificates)
  and ECDSA signatures with secp256r1 \& SHA256 is mandatory (unless an implementation 
  explicitly opts out). (RSASSA-PSS must also be supported for \CertificateVerify\ messages.)
}


%\item certificate\_authorities (47)

\end{description}

\noindent
Additional extensions exist and may be included in \ClientHello\ messages.
(Appendix~\ref{sec:extensions} lists all extensions.)

A \ClientHello\ message is consumed by the server: The 
server first checks that the message is a TLS 1.3 \ClientHello\ 
message, which is achieved by checking that extension \TLSsupportedVersions\
is present and that constant 0x0304 is the first listed preference.
(The \ClientHello\ message format is backward compatible with 
previous versions of TLS, hence, the message might need to be processed
by a prior version of TLS. Those details are beyond the scope of this 
manuscript.) The server may 
also check that field \TLSlegacyVersion\
is set to constant 0x0303 and field \TLSlegacySessionId\ is set to a zero-length vector.
Moreover, the server checks field \TLSlegacyCompressionModes\ is set to constant 0x00
and aborts with an \TLSillegalParameter\ alert if the check fails.\footnote{%
  RFC 8446 does not explicitly require servers to check fields 
  \TLSlegacyVersion\ and \TLSlegacySessionId, it merely requires clients to
  set those fields correctly. Accordingly, we assume servers \emph{may} 
  perform these checks, rather than mandating them. By comparison,
  RFC 8446 explicitly requires field \TLSlegacyCompressionModes\ to be
  correctly set.
}
Next,\label{comp:CH:cons:cipher} the server selects an acceptable cipher suite from field \TLScipherSuites, 
disregarding suites that are not recognised, unsupported, or otherwise unacceptable, 
and aborting with a \TLShandshakeFailure\ or an \TLSinsufficientSecurity\ alert if no 
mutually acceptable cipher suite exists. Finally, the server processes any remaining
extensions:


\begin{description}

\item \TLSsupportedGroups\ and \TLSkeyShare: The server selects an acceptable group 
  from the list; aborting with a \TLSmissingExtension\ alert if extension \TLSsupportedGroups\ 
  is present and extension \TLSkeyShare\ is absent, or vice versa; aborting with a 
  \TLShandshakeFailure\ or an \TLSinsufficientSecurity\ alert if no mutually acceptable 
  group exists; and \label{comp:CH:cons:HRR} responding with a \HelloRetryRequest\ 
  message if extension \TLSkeyShare\ does not offer a key share for the selected group.

\item \TLSpsk\ and \TLSpskModes: The server selects an acceptable key identifier
  from the list (\ifSpecNotes\textcolor{red}{the spec requires that identifier to be compatible 
  with the server-selected cipher suite. That's rather vague. I think compatibility varies 
  between out-of-band and NST provisioned keys. I think the following captures the meaning 
  of compatibility:}\fi{}that identifier must be associated with a hash function, 
  AEAD algorithm, or both, which are defined by the server-selected cipher suite), 
  disregarding unknown identifiers, aborting with an \TLSillegalParameter\ 
  alert if extension \TLSpsk\ is not the last extension in the \ClientHello\ message,
  and aborting if extension \TLSpsk\ is present without \TLSpskModes.
  The server also selects a key exchange mode.\label{comp:CH:cons:psk}
  If no mutually acceptable key identifier exists and extensions \TLSsupportedGroups\ 
  and \TLSkeyShare\ are present, then the server should perform a non-PSK handshake.
  

\item \TLSsignatureAlgorithms\ and \TLSsignatureAlgorithmsCert:
  The server selects acceptable signature algorithms for \CertificateVerify\ 
  and \Certificate\ messages.

\end{description}

\noindent
Any unrecognised extensions are ignored and the server aborts with a 
\TLSmissingExtension\ alert if extension \TLSpsk\ is absent as-is
either extension \TLSsupportedGroups, \TLSsignatureAlgorithms, or 
both. (Alerts are formally defined by RFC 8446, as 
discussed in Appendix~\ref{sec:alerts}.)

\begin{tcolorbox}
Consumption is implemented by class 
\code{ClientHello.ClientHelloConsumer} (Listing~\ref{lst:ClientHelloConsumer}).
That class checks the presence of extension \TLSsupportedVersions, to determine
whether the message is a TLS 1.3 \ClientHello\ message, and the remainder of the 
message is processed by class \code{ClientHello.T13ClientHelloConsumer} 
(Listings~\ref{lst:T13ClientHelloConsumer} \&~\ref{lst:T13ClientHelloConsumerC}), 
if it is a TLS 1.3 message. %\footnotemark\
\begin{comment}
which is reliant on classes \code{SSLExtensions} 
%(Listings~\ref{lst:SSLExtensions} \&~\ref{lst:SSLExtensionsB}) 
and \code{SSLExtension} 
%(Listing~\ref{lst:SSLExtension}).\footnotemark\
(Appendix~\ref{sec:extensions}).
\end{comment}
%The latter is in turn reliant on a variable \code{onLoadConsumer} of (interface) type 
%\code{SSLExtension.ExtensionConsumer} which is instantiated by a constant in 
%the form \code{ThisNameExtension.chOnLoadConsumer}, where \code{ThisName}
%corresponds to extension \TLSField{this\_name}. 
Consumption of the \ClientHello\ message may result in the server aborting or 
responding with either a \ServerHello\ or \HelloRetryRequest\ message.
\end{tcolorbox}

\begin{comment}
\footnotetext{Class \code{ClientHello.ClientHelloConsumer} implements
  interface \code{SSLConsumer}, which defines a single method, namely, 
  \code{void consume(ConnectionContext context, ByteBuffer message) throws IOException}.
  Similarly, class \code{ClientHello.T13ClientHelloConsumer} implements
  interface \code{HandshakeConsumer}, which defines method
  \code{void consume(ConnectionContext context, HandshakeMessage message) throws IOException}.
  Parameter \code{context} defines the active context in both interfaces 
  and may be cast to children \code{ClientHandshakeContext} and \code{ServerHandshakeContext}, 
  as seen in Line~770 of Listing~\ref{lst:ClientHelloConsumer}, for instance.
  \textcolor{red}{Are these details worth knowing?}
}
\end{comment}

\lstinputlisting[
  float=tbp,
  linerange={
    52-53,%handshakeConsumer
    760-760,%ClientHelloConsumer class declaration
    767-768,%consume method delaration
    770-770,
    781-783,
    785-786,
    791-793,
    795-796,
%    800-816,
    800-808,
    %810-811,
    809-815,
    816-816,
    %831-831,
    830-834,
    836-836,
%    840-842,%negotiateProtocol according to ClientHello.client_version
%    844-846,
%    852-854,
%    857-859,
%    868-868,
    873-875,
    877-880,
    884-884,
    888-892,
    902-903
  },
  label=lst:ClientHelloConsumer,
  caption={[\code{ClientHello.ClientHelloConsumer} consumes generic \ClientHello]
  Class \code{ClientHello.ClientHelloConsumer} defines method \code{consume} to 
  instantiate a (generic) \ClientHello\ message from an input buffer (Lines~785--786); update the 
  server's active context to include the client's offered versions (Lines~800--803), 
  indirectly using method 
  \code{SupportedVersionsExtension.CHSupportedVersionsConsumer.consume}, which calls
  \code{context.handshakeExtensions.put(CH\_SUPPORTED\_VERSIONS, spec)}, where 
  parameter \code{spec} is a byte array encoding of extension \TLSsupportedVersions;
  %(variable \code{context} is named \code{shc} in the method);
  select the first server preferred version that the client offered (Lines~810--811 \& 880--892); 
  and update the active context to include that selected version preference as the negotiated 
  protocol (Line~816). Further processing is deferred (Line 831) to class 
  \code{ClientHello.T13ClientHelloConsumer} (Listing~\ref{lst:T13ClientHelloConsumer}).
  \ifPresentationNotes\textcolor{red}{Maybe mention consumption of TLS 1.2 \ClientHello\ messages}\fi
}]{listings/ClientHello.java}

\lstinputlisting[
  float=tbp,
  linerange={
    59-60,
    1075-1076,
    1083-1084,
    1086-1087,
    1097-1119,
    1120-1127
  },
  label=lst:T13ClientHelloConsumer,
  caption={[\code{ClientHello.T13ClientHelloConsumer} consumes \ClientHello]
  Class \code{ClientHello.T13ClientHelloConsumer} defines method \code{consume}
  to process incoming (TLS 1.3) \ClientHello\ messages (further to processing 
  shown in Listing~\ref{lst:ClientHelloConsumer}). The method updates the server's 
  active context to include
  any pre-shared key identifiers and key exchange modes offered by the client (Lines~1101--1105),
  indirectly using the \code{consume} method of classes 
  \code{PskKeyExchangeModesExtension.PskKeyExchangeModesConsumer} 
  and \code{PreSharedKeyExtension.CHPreSharedKeyConsumer}; updates the active 
  context to include any further (enabled) extensions 
  (Lines~1113--1119), excluding those that have already been added to the active context,
  namely, extensions \TLSsupportedVersions, \TLSpsk, and \TLSpskModes; and proceeds by 
  producing either a \HelloRetryRequest\ message if extension \TLSkeyShare\ does not
  offer a key share for the server selected group (method  
  \code{KeyShareExtension.CHKeyShareConsumer.consume} may add a producer 
  for \HelloRetryRequest\ messages which ensures \code{!shc.handshakeProducers.isEmpty()} 
  holds) or a \ServerHello\ message otherwise (Lines~1121--1126).
  \ifImplNotes
  \textcolor{blue}{Contrary to comments (Lines~1107--1109) extension key\_share doesn't 
  appear to be ignored in Lines 1113--1119, extensions \TLSpskModes, \TLSpsk, and 
  \TLSsupportedVersions\ are. (If it were ignored, then \HelloRetryRequest\ messages 
  would never be produced.) }
  \fi
}]{listings/ClientHello.java}

\lstinputlisting[
  float=tbp,
  linerange={
    1129-1133,
    1135-1135,
    1147-1147,
    1149-1150,
    1154-1154,
    1159-1180,
    1185-1193
  },
  label=lst:T13ClientHelloConsumerC,
  caption={[\code{ClientHello.T13ClientHelloConsumer} consumes \ClientHello\ (cont.)]
  Class \code{ClientHello.T13ClientHelloConsumer} (continued from 
  Listing~\ref{lst:T13ClientHelloConsumer}) defines methods \code{goHelloRetryRequest} 
  to produce a \HelloRetryRequest\ message and \code{goServerHello} 
  to produce a \ServerHello\ message. The latter method prepares the server's active 
  context for the client's response (Lines~1154 \& 1159--1162); updates the active context to 
  include a producer for \ServerHello\ messages (Lines~1168--1169); constructs
  an array of producers that servers might use during the handshake protocol, 
  namely, producers for messages \ServerHello, \EncryptedExtensions, 
  \CertificateRequest, \Certificate, \CertificateVerify, and \Finished,
  in the order that they might be used (Lines~1171--1180); and uses those producers 
  to produce messages when the active context includes the producer
  (Lines~1185--1191). Since a \ServerHello\ message producer is 
  included, a \ServerHello\ message is always produced, using 
  method \code{ServerHello.T13ServerHelloProducer.produce} 
  (Listing~\ref{lst:T13ServerHelloProducer}). That method 
  adds producers for \EncryptedExtensions\ and \Finished\ messages
  (Listing~\ref{lst:T13ServerHelloProducer}, Lines~560--563), 
  since those messages must be sent. Other producers 
  may also be added.
}]{listings/ClientHello.java}

\begin{comment}

%Pushed to Listing~\ref{lst:SSLExtensions}
\lstinputlisting[
  float=tbp,
  linerange={
    132-134,
    163-164,
    169-170    
  },
  label=lst:SSLExtensionsB,
  caption={Class \code{SSLExtensions}}?
}]{listings/SSLExtensions.java}

%Pushed to Listing~\ref{lst:SSLExtension}
\lstinputlisting[
  float=tbp,
  linerange={
    539-547
  },
  label=lst:SSLExtensionB,
  caption={Class \code{SSLExtension} 
}]{listings/SSLExtension.java}

\end{comment}




