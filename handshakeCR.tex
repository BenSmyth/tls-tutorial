\section{Client authentication: \CertificateRequest}\label{sec:CR}

A server may request client authentication by sending a \CertificateRequest\ 
message, comprising the following fields:

\begin{description}

\item \TLScertificateRequestContext: A zero-length identifier. (A 
  \CertificateRequest\ message may also be sent to initiate post-handshake 
  authentication, as explained below, in which case a nonce may be used as
  an identifier.)

\item \TLSextensions: A list of extensions describing authentication
  properties. The list must contain at least extension \TLSsignatureAlgorithms.
  (Table~\ref{table:extensions}, Appendix~\ref{sec:extensions}, lists other
  permissible extensions.)

    
\end{description}

\begin{sloppypar}
\noindent
A client %consuming a \CertificateRequest\ message
may decline to authenticate by responding with a \Certificate\ message
that does not contain a certificate, followed by a \Finished\ message. 
(The server may continue %\sout{the handshake} 
without client authentication or abort with a \TLScertificateRequired\ alert.)
\ifSpecNotes
\textcolor{red}{
  The spec only defines the bracketed remark for handshakes. Presumably 
  it applies to post-handshake authentication too.
}
\fi
 Alternatively,
a client may authenticate by responding with \Certificate\ 
and \CertificateVerify\ messages 
(such that $\CertificateRequest.\TLScertificateRequestContext = \Certificate.\TLScertificateRequestContext$),
followed by a \Finished\ message.
The \CertificateVerify\ message includes a signature over string ``TLS 1.3, client CertificateVerify'',
rather than ``TLS 1.3, server CertificateVerify'', to distinguish client-
and server-generated \CertificateVerify\ messages, and to help defend 
against potential cross-protocol attacks. The signature
algorithm must be one of those listed in field
\TLSsupportedSignatureAlgorithms\ of extension \TLSsignatureAlgorithms\
in the \CertificateRequest\ message.
(The server may abort %\sout{the handshake} 
if the client's certificate chain is 
unacceptable, e.g., when the chain contains a signature from an unknown or 
untrusted certificate authority. Alternatively, the server may proceed, 
considering the client unauthenticated.)
\ifSpecNotes
\textcolor{red}{
  The spec only defines the bracketed remark for handshakes. Presumably 
  it applies to post-handshake authentication too.
}
\fi
Any extensions listed by the \Certificate\ message must respond to ones 
listed in the \CertificateRequest\ message.
\end{sloppypar}

For (EC)DHE-only key exchange, client authentication is possible during 
a handshake: a server includes a \CertificateRequest\ message immediately 
after their \EncryptedExtensions\ message (and before \Certificate, 
\CertificateVerify, and \Finished\ messages), and a client responds 
with \Certificate, (optionally) \CertificateVerify, and \Finished\ messages. 
For PSK-based key exchange, a server must only request client authentication
if their peer's \ClientHello\ message included extension \TLSpostHandshakeAuth. 
Such a request can be made by sending a \CertificateRequest\ message (with a 
non-zero length identifier) after the handshake protocol completes. A client 
responds with \Certificate, (optionally) \CertificateVerify, and \Finished\ 
messages, computing the HMAC with 
%
\begin{multline*}
  \TLSfinishedKey =   \HKDFExpandLabel( \TLSapplicationTrafficSecret[client]{N},\\
       ``finished", ``", \HashLength)
\end{multline*}
%
(Post-handshake authentication is only concerned with updating the client's 
application-traffic key, for the purposes of blinding the client's identity 
to that key. Hence, secret \TLSfinishedKey\ is not concerned with traffic 
secret \TLSapplicationTrafficSecret[server]{N}. Beyond traffic keys, a key 
established by a \NewSessionTicket\ message, sent after post-handshake 
authentication, will also be bound to the client's identity.)
A client receiving an unsolicited post-handshake authentication request 
(i.e., message \ClientHello\ did not include extension \TLSpostHandshakeAuth) 
must abort with an \TLSunexpectedMessage\ alert.                                          


