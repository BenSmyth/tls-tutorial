\subsection{Early data}\label{sec:handshakeEarlyData}

For PSK-based key exchange, clients may exceptionally start sending encrypted application data 
immediately after \ClientHello\ messages (before receiving a \ServerHello\ message),\footnote{%
  RFC 8446 only permits clients to send early data when the pre-shared key is associated with 
  data permitting them to do so. For resumption PSKs, permission is granted by inclusion of 
  extension \TLSearlyData\ in \NewSessionTicket\ messages (\S\ref{sec:NST}).
}
enabling 
a zero round-trip time (0-RTT), at the cost of forward secrecy 
(since application data is solely encrypted by the 
pre-shared key, %rather than a combination of that key and Diffie-Hellman
%  key shares, hence, forward secret cannot be obtained
which does not afford forward secrecy, as per PSK-only key exchange) and replay protection
(since such protection is derived from the server's nonce, which is constructed
after encrypted application data is sent).\footnote{
  RFC 8446 discusses anti-replay defences and notes that 
  single-use PSKs enjoy forward secrecy.
}
Such early data requires the
\ClientHello\ message to include extensions \TLSearlyData\ and \TLSpsk,
and 
application data must be encrypted using 
%the pre-shared key identified by the first entry in the client's list of pre-shared key identifier .
the client's first identified pre-shared key. (Extension \TLSearlyData\
is not associated with data, encrypted application data is sent separately.)

To consume early data, a server must select the client's first pre-shared key
identifier and an offered cipher suite associated with that identifier. The
server must check the identifier is associated with the server-selected 
protocol version and (if extension \TLSapplicationLayerProtocolNegotiation\ is present) 
application protocol. (These checks are a superset of those for PSK-based key 
exchange without early data.) Additionally, for resumption PSKs, the server 
must check that the PSK is not beyond its lifetime. 
If checks succeed (and the server is willing to consume early data), 
then the server will include a corresponding \TLSearlyData\ extension
in their \EncryptedExtensions\ message.
(When consuming that extension, the client must check that the server selected
the client's first pre-shared key identifier, aborting with an \TLSillegalParameter\
alert, if the check fails.)
Otherwise, no such extension will 
be sent and no early data will be consumed (extension \TLSearlyData\ is
ignored), and the server will proceed in one of the following two ways (which 
must also be followed by servers not supporting early data): Respond 
with a \ServerHello\ message, excluding extension \TLSearlyData,
or respond with a \HelloRetryRequest\ message, forcing the client to send 
a second \ClientHello\ message without extension \TLSearlyData.
In both cases, the server must skip past early data. For the former, 
given that all messages will be encrypted, the server must decrypt 
messages with the handshake traffic key, discard messages when 
decryption fails, and treat the first successfully decrypted
message as the client's next handshake message, thereafter 
proceeding as if no early data were sent. 
For the latter, the second \ClientHello\ message will be unencrypted
and the server can discard all encrypted messages (identified by
record type \TLSapplicationData\ (0x23), rather than type \TLShandshake\
(0x22), as introduced in Section~\ref{sec:record}), before proceeding as 
if no early data were sent when the second \ClientHello\ message
is identified.
(When the pre-shared key is associated with a maximum amount of early 
data, the server should abort with an \TLSunexpectedMessage\ alert if 
the maximum is exceeded when skipping past early data.)



\begin{tcolorbox}
Early data is not supported by JDK~11 (\url{https://openjdk.java.net/jeps/332}),
nor subsequent versions: When extension \TLSearlyData\ is included in message
\ClientHello, that extension will be processed (Line~1119, Listing~\ref{lst:T13ClientHelloConsumer})
and runtime exception \code{UnsupportedOperationException} will be thrown 
(omitted from Listing~\ref{lst:SSLExtension}).
\ifImplNotes
\textcolor{red}{
Since JDK throws a runtime exception rather than skipping early data, 
is JDK non-compliant with the spec, or can runtime exceptions be used
to avoid non-compliance?
}
\fi
\end{tcolorbox}


\subsubsection*{Data associated with pre-shared keys}

An external PSK (established independently of TLS) must minimally be associated 
with a hash function and an identity. (The hash function may default to SHA-256, 
if no function is explicitly associated.)
Such a PSK grants freedom over AEAD algorithms, whilst 
fixing the hash function. By comparison, a resumption PSK (established by 
\NewSessionTicket\ messages) is associated with values negotiated in the 
connection that provisioned the PSK, which fixes a cipher suite, hence, 
an AEAD algorithm and a hash function.\footnote{%
  Although resumption PSKs are associated with cipher suites, they need 
  not be used with defined AEAD algorithms, except for compatibility 
  with early data.}
(It follows that a connection established using a resumption PSK will 
inherit security from the connection in which the resumption PSK was 
established.)
Resumption PSKs are compatible with 
early data by default (assuming suitable provisioning with extension 
\TLSapplicationLayerProtocolNegotiation, if relevant), whereas (minimally 
associated) external PSKs are not. They must be associated with a cipher suite 
(rather than just a hash function), a protocol version, and (if relevant)  an 
application protocol, for compatibility with early data.


\subsubsection{\EndOfEarlyData}

A client can transmit early data until they receive a server's \Finished\
message. After which, the client transmits an \EndOfEarlyData\ message
(encrypted using a key derived from secret \TLSclientEarlyTrafficSecret), 
if the server's \EncryptedExtensions\ message included extension
\TLSearlyData. Otherwise, early data has not and will not be consumed 
by the server, and no \EndOfEarlyData\ message is sent. The \EndOfEarlyData\ 
message indicates that all early data has been transmitted and subsequent
handshake messages will be encrypted with the client's handshake-traffic key.
(Servers must not send \EndOfEarlyData\ messages and clients receiving 
such messages must abort with an \TLSunexpectedMessage\ alert.)




