\subsubsection{\HelloRetryRequest}\label{sec:HelloRetryRequest}\label{sec:HRR}

A server that consumes a \ClientHello\ message, without a share for the 
server-selected group, responds with a \HelloRetryRequest\ message.
That message is an instance of a \ServerHello\ message, with field \TLSrandom\
set to a special constant value.\footnote{
  For convenience, \HelloRetryRequest\ and \ServerHello\ messages are distinctly named 
  (in the specification), despite \HelloRetryRequest\ messages being instances 
  of \ServerHello\ messages. It follows that a \ServerHello\ message might be 
  confused for a \HelloRetryRequest\ message, but this only occurs with probability 
  $\frac{1}{2^{128}}$, hence, confusion will not occur in practice.
}
In addition to mandatory extension \TLSsupportedVersions, message \HelloRetryRequest\ 
\ifSpecNotes
\marginpar{The specification states ``[a \HelloRetryRequest] SHOULD 
contain the minimal set of extensions necessary for the client to 
generate a correct ClientHello pair," which I presume means the server
should include \TLSkeyShare, but surely that is mandatory?} 
\fi
should include extension \TLSkeyShare\ to indicate the server-selected 
group.\footnote{Extension \TLSkeyShare\ is associated with key shares
  for \ClientHello\ messages and a single key share for \ServerHello\
  messages, whereas the extension is associated with the server-selected 
  group for \HelloRetryRequest\ messages. Hence, data structures associated
  with extension \TLSkeyShare\ vary between messages.}
(The server should defer producing a key share for this group
until the client's response is received.) 
The server may also include extension \TLScookie\ associated 
with some data:

\begin{description}

\item \TLScookie: Some server-specific data for purposes including, but not limited to,
  first, offloading state (required to construct transcripts) to the client, by 
  storing the hash of the \ClientHello\ message in the cookie (with suitable integrity
  protection); and, secondly, DoS protection, by forcing the client to demonstrate
  reachability of their network address.

\end{description}

A \HelloRetryRequest\ message is consumed by the client, which performs
the checks specified for \ServerHello\ messages (above), 
additionally aborting with an \TLSillegalParameter\ alert if  
the server-selected group is not amongst those offered (\ClientHello.\TLSsupportedGroups)
or a key share for that group was already offered,
or aborting with an an \TLSunexpectedMessage\ alert if a \HelloRetryRequest\
message was already received in the same connection.
%
\begin{comment}
Additional abort [due to \emph{a key share for that group was already offered}] inferred 
(possibly incorrectly) from: 
\begin{quote}
  Clients MUST abort the handshake with an "illegal_parameter" 
  alert if the HelloRetryRequest would not result in any change 
  in the ClientHello.
\end{quote}
A change is required if key shares (plural) were offered in the Client Hello Message. (Those key 
shares must be replaced by a single key share.) I suspect the spec meant  "would not result in a
meaningful change," e.g., when a key share for the server-selected group was already offered.

Actually, this is made explicit later in the specification:
\begin{quote}
  the client MUST verify that...the selected_group field does not
  correspond to a group which was provided in the "key_share" extension
  in the original ClientHello. If [the check] fails, then the client MUST 
  abort the handshake with an "illegal_parameter" alert.
\end{quote}
\end{comment}
%
\ifSpecNotes
\begin{color}{red}\\
The specification seems to contain conflicting requirements: \marginpar{conflict reported 1 May 2020}
\begin{quote}
  Clients MUST check for this [supported_versions] extension \emph{prior to
  processing} the rest of the ServerHello (although they will have to           
  parse the ServerHello in order to read the extension)
\end{quote}
and
\begin{quote}
   the client \emph{MUST check} the
   legacy_version, legacy_session_id_echo, cipher_suite, and
   legacy_compression_method as specified in Section 4.1.3 and \emph{then}
   process \emph{the extensions}, starting with determining the version using          
   "supported_versions".
\end{quote}
which conflict on the emphasised text. 
\end{color}
\fi
A client that is able to successfully consume a \HelloRetryRequest\ message
responds with their original \ClientHello\ message, replacing the key shares 
in extension \TLSkeyShare\ with a single key share from the server-selected group,
removing extension \TLSearlyData\ if present,
including a copy \label{comp:HRR:cons:cookie} of extension \TLScookie\ and 
associated data if the extension appeared in the 
\HelloRetryRequest\ message, and updating extension \TLSpsk\ by recomputing its 
obfuscated age and binder values (\S\ref{sec:NST}).
Moreover, the client should remove any pre-shared
key identifiers that are incompatible with the server-selected cipher suite
\ifSpecNotes
\textcolor{red}{similarly to earlier issue regarding ``compatibility,''
  in relation to \ClientHello, the spec is rather vague}
\fi
(i.e., remove identifiers associated with a hash function, AEAD algorithm, 
  or both that are not defined by the server-selected cipher suite).
That \ClientHello\ message is consumed by the server (\S\ref{sec:CH}) and 
the server responds with a \ServerHello\ message, which must contain the
previously selected cipher suite, namely, \HelloRetryRequest.\TLSCipherSuite.
The \ServerHello\ message is consumed by the client as described above, 
additionally aborting with an \TLSillegalParameter\ alert if the server-selected
cipher suite differs from the previous server-selected cipher suite 
(\HelloRetryRequest.\TLSCipherSuite), if extension \TLSsupportedVersions\
is associated with a list of offered protocol versions that differ from 
the previous list (\HelloRetryRequest.\TLSsupportedVersions), or if the 
server's key share does not belong to the previous server-selected group 
(\HelloRetryRequest.\TLSkeyShare).

Beyond the above two instances of \ClientHello\ messages, a server that 
receives a \ClientHello\ message at any other time must abort with an 
\TLSunexpectedMessage\ alert.


 
\begin{tcolorbox}
The \HelloRetryRequest\ message is implemented by class \code{Server\-Hello.Server\-Hello\-Message}
(Listings~\ref{lst:ServerHelloMessage}--\ref{lst:ServerHelloMessageB}). Instances 
of that class are produced by class \code{Server\-Hello.T13\-Server\-Hello\-Producer} 
(Listings~\ref{lst:T13HelloRetryRequestProducer}), which is instantiated as static 
constant \code{Server\-Hello.t13\-Handshake\-Producer}. That constant is used indirectly
-- via class \code{SSL\-Handshake.HELLO_RETRY_REQUEST} -- to produce \HelloRetryRequest\
messages in class \code{Client\-Hello.T13\-Client\-Hello\-Consumer}
(Listing~\ref{lst:T13ClientHelloConsumerC}). Consumption is implemented by 
class \code{Server\-Hello.Server\-Hello\-Consumer} (Listing~\ref{lst:ServerHelloConsumer}
\&~\ref{lst:ServerHelloConsumerB}). That class checks the presence of extension
\TLSsupportedVersions, to determine whether the message is a TLS 1.3 \HelloRetryRequest\
message, and the remainder of the message is processed by class
\code{Server\-Hello.T13\-Hello\-Retry\-Request\-Consumer} 
(Listing~\ref{lst:T13HelloRetryRequestConsumer}), if it is a TLS 1.3 message. 
Successful consumption 
results in transmission of a further \ClientHello\ message (which is 
consumed by class \code{Client\-Hello.T13\-Client\-Hello\-Consumer}, 
Listings~\ref{lst:T13ClientHelloConsumer} \&~\ref{lst:T13ClientHelloConsumerC}), 
with any \TLScookie\ extension being indirectly processed -- via class \code{CookieExtension}
-- by class \code{HelloCookieManager}.
\end{tcolorbox}


\lstinputlisting[
  float=tbp,
  linerange={
    61-62,
    732-733,
    740-743,
    745-747,
    754-760,
    762-762,
    766-770,
    776-778,
    780-782,
    784-786,
    788-791  
  },
  label=lst:T13HelloRetryRequestProducer,
  caption={[\code{ServerHello.T13HelloRetryRequestProducer} produces \HelloRetryRequest]
  Class \code{ServerHello.T13HelloRetryRequestProducer} defines method 
  \code{produce} to instantiate a \HelloRetryRequest\ message, i.e., 
  a \ServerHello\ message with field \TLSrandom\ set to a special constant
  value~(Lines 754--760), populate 
  the extension field for the active context (Lines~767--770), write 
  the %\ServerHello\ 
  message to an output stream (Lines~777--778),
  and prepare the server's active context for the client's response (Lines~762 \& 785--786).  
}]{listings/ServerHello.java}


\lstinputlisting[
  float=tbp,
  linerange={
    876-877,
    881-889,
    891-892,
    897-902,
    909-909,
%    914-919,
    924-924,
    926-926
  },
  label=lst:ServerHelloConsumerB,
  caption={[\code{ServerHello.ServerHelloConsumer} consumes generic \HelloRetryRequest\ (cont.)]
  Class \code{ServerHello.ServerHelloConsumer} (omitted from 
  Listing~\ref{lst:ServerHelloConsumer}) defines method 
  \code{onHelloRetryRequest} to consume a (generic) \HelloRetryRequest\ message.
  Similarly to method \code{ServerHello.ServerHelloConsumer.onHelloServer}
  (Listing~\ref{lst:ServerHelloConsumer}), the client's active context is updated to include
  the server's selected version (Lines~886--892), aborting if that version was not
  offered by the client (Lines~897--902). 
  Further processing is deferred (Line~924) to class \code{ServerHello.T13HelloRetryRequestConsumer}
  (Listing~\ref{lst:T13HelloRetryRequestConsumer}).
}]{listings/ServerHello.java}


\lstinputlisting[
  float=tbp,
  linerange={
    77-78,
    1373-1374,  
    1376-1377,
    1383-1383,
    1389-1392,
    1397-1397,
%    1448-1454,
    1459-1461
  },
  label=lst:T13HelloRetryRequestConsumer,
  caption={[\code{ServerHello.T13HelloRetryRequestConsumer} consumes \HelloRetryRequest]
  Class \code{ServerHello.T13HelloRetryRequestConsumer} defines method 
  \code{consume} to process incoming (TLS 1.3) \HelloRetryRequest\ messages (further to 
  processing shown in Listing~\ref{lst:ServerHelloConsumerB}). The method 
  updates the active context to include the server's selected
  cipher suite (Line~1383) and extensions (Line~1390--1397), and produces a 
  \ClientHello\ message (Line~1459).
}]{listings/ServerHello.java}        

