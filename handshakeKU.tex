\subsubsection{\KeyUpdate}

After sending a \Finished\ message, an endpoint may send a \KeyUpdate\ message
to notify their peer that they are updating their cryptographic key. The message 
comprises of the following field:

\begin{description}

\item \TLSrequestUpdate: A bit indicating whether the recipient should 
  respond with their own \KeyUpdate\ message and update their own cryptographic
  key.

\end{description}

\noindent
After sending a \KeyUpdate\ message, the sender must update their application-traffic 
secret and corresponding application-traffic keys (\S\ref{sec:secrets}--\ref{sec:hkdf:trafficKeys}).

A peer that receives a \KeyUpdate\ message prior to receiving a \Finished\ 
message aborts with an \TLSunexpectedMessage\ alert, moreover, the peer 
aborts with an \TLSillegalParameter\ alert if field \TLSrequestUpdate\ does not
contain a bit. Otherwise, the peer updates its receiving keys
(\S\ref{sec:secrets}--\ref{sec:hkdf:trafficKeys}). Moreover, when the sender
requests that the peer updates their sending keys, the peer must send a
\KeyUpdate\ message of its own (without requesting that the sender update 
its cryptographic key), prior to sending any further application data.

\begin{tcolorbox}
\KeyUpdate\ messages are implemented, produced, and consumed by inner-classes of 
class \code{KeyUpdate}. Those classes define the one field of a \NewSessionTicket\ 
message and constructors to instantiate it from parameters or an input buffer, moreover,
they define methods to write such a message to an output stream and to read such a message 
from an input buffer, those methods also update application-traffic secrets and 
corresponding application-traffic keys, and the latter produces a \KeyUpdate\ message
of the receiver when requested by the sender.
\end{tcolorbox}


