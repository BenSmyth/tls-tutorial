\subsubsection{\NewSessionTicket}\label{sec:NST}

After receiving a client's \Finished\ message, a server can initiate establishment 
of a new pre-shared key, which will be derived from the resumption master secret
\TLSresumptionMasterSecret\ (Figure~\ref{fig:keyDerivation}). 
Such a pre-shared key may be used to establish subsequent channels. Establishment 
is initiated with a \NewSessionTicket\ message, comprising the following fields:

\begin{description}
\item \TLSticketLifetime: A 32-bit unsigned integer indicating the lifetime 
      in seconds %(from the time of issuance) 
      of the pre-shared key, which must not exceed seven days (604800 seconds).

\item \TLSticketAgeAdd: A 32-bit nonce to obscure the lifetime.

\item \TLSticketNonce: A nonce for key derivation.

\item \TLSticket: A key identifier. 

\item \TLSextensions: A list of extensions, currently limited to extension \TLSearlyData, 
  which indicates that early data is permitted and defines a maximum amount 
  of early data.

\end{description}

\begin{sloppypar}
\noindent
The associated pre-shared key is computed as follows:
\[
  \HKDFExpandLabel(\TLSresumptionMasterSecret, ``resumption", \TLSticketNonce, \HashLength).
\]
Since the pre-shared key is computed from nonce \TLSticketNonce, it follows 
that each \NewSessionTicket\ message creates a distinct pre-shared key.
\end{sloppypar}

A \NewSessionTicket\ is consumed by the client, which derives and stores 
the pre-shared key along with associated data (including the negotiated 
hash function). That data may be stored by client and used in 
extension \TLSpsk\ of subsequent \ClientHello\ messages.
Data must not be used 
\ifSpecNotes
\textcolor{red}{the spec refers to storage rather than use}
\fi
longer than seven days or beyond its lifetime (specified by \TLSticketLifetime), 
whichever is shorter, and endpoints may retire data early.


   The sole extension currently defined for NewSessionTicket is
   "early_data", indicating that the ticket may be used to send 0-RTT
   data (Section 4.2.10).  It contains the following value:

   max_early_data_size:  The maximum amount of 0-RTT data that the
      client is allowed to send when using this ticket, in bytes.  Only
      Application Data payload (i.e., plaintext but not padding or the
      inner content type byte) is counted.  A server receiving more than
      max_early_data_size bytes of 0-RTT data SHOULD terminate the
      connection with an "unexpected_message" alert.  Note that servers
      that reject early data due to lack of cryptographic material will
      be unable to differentiate padding from content, so clients
      SHOULD NOT depend on being able to send large quantities of
      padding in early data records.




\begin{tcolorbox}
\NewSessionTicket\ messages are implemented, produced, and consumed by inner-classes of 
class \code{NewSessionTicket}. Those classes define the five fields of a \NewSessionTicket\ 
message and constructors to instantiate them from parameters or an input buffer, moreover,
they define methods to write such a message to an output stream and to read such a message 
from an input buffer.
\end{tcolorbox}

\subsubsection*{Extension \TLSpsk}

The pre-shared key identifiers listed by extension \TLSpsk\ (\S\ref{sec:CH}) may include 
identifiers established by \NewSessionTicket\ messages, identifiers established %out-of-band,
externally,
or both. Each identifier is coupled with an obfuscated age, which is derived by addition of 
\NewSessionTicket.\TLSticketLifetime\ and \NewSessionTicket.\TLSticketAgeAdd\ (modulo $2^{32}$)
for identifiers established by \NewSessionTicket\ messages, and $0$ for identifiers established
%out-of-band. 
externally.
Moreover, extension \TLSpsk\ associates each identifier with a PSK binder, 
which binds the pre-shared key with the current handshake, and to the session in which the 
pre-shared key was generated for pre-shared keys established by \NewSessionTicket\ messages.
The PSK binder is computed as an HMAC over a partial transcript, which excludes binders, 
namely,
\[
  \HMACHash(\TLSbinderKey,\TranscriptHash(\Truncate(CH)))
\]
for transcripts which include only a single \ClientHello\ message $CH$, and
\[
  \HMACHash(\TLSbinderKey,\TranscriptHash(CH,HRR,\Truncate(CH')))
\]
for transcripts that include an initial \ClientHello\ message $CH$, followed 
by a \HelloRetryRequest\ message $HRR$ and a subsequent \ClientHello\ message $CH'$,
where function \Truncate\ removes binders. 
%
When consuming a \ClientHello\ message that includes extension \TLSpsk, a server recomputes 
the HMAC for their selected pre-shared key and checks that it matches the corresponding binder 
listed by the extension, aborting if the check fails or the binder is not present.

\begin{tcolorbox}
Extension \TLSpsk\ is implemented, produced, and consumed by inner-classes of 
class \code{PreSharedKeyExtension}. Those classes define fields of extension \TLSpsk\ 
 and constructors to instantiate them from parameters or an input buffer, moreover,
they define methods to write such an extension to an output stream and to read such an extension 
from an input buffer, the latter is reliant on static methods \code{checkBinder}, 
\code{computeBinder} and \code{deriveBinderKey} to recompute HMACs for pre-shared keys 
and to check whether they match the corresponding binder listed by the extension.
\end{tcolorbox}


