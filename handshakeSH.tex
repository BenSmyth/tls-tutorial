\subsection{\ServerHello}\label{sec:handshakeSH}\label{sec:SH}

A server that is able to successfully consume a \ClientHello\ message responds
with a \ServerHello\ message, %(specified in Appendix~\ref{sec:specification}), 
comprising fields \TLSlegacyVersion, \TLSrandom, and \TLSextensions\ as per the 
\ClientHello\ message and the following fields:

\begin{description}

\item \TLSlegacySessionIdEcho: The contents of \ClientHello.\TLSlegacySessionId.

\item \TLScipherSuite:  The cipher suite selected by the server from \ClientHello.\TLScipherSuites.

\item \TLSlegacyCompressionMode: Constant 0x00.

\end{description}

\noindent
Legacy fields are included for backwards compatibility.

\begin{tcolorbox}
The \ServerHello\ message is implemented by class \code{ServerHello.ServerHelloMessage}
(Listings~\ref{lst:ServerHelloMessage} \&~\ref{lst:ServerHelloMessageB}). Instances 
of that class are produced by class \code{Server\-Hello.T13\-Server\-Hello\-Producer}
(Listings~\ref{lst:T13ServerHelloProducer} \& \ref{lst:T13ServerHelloProducerB}), %\footnotemark\
which is instantiated as static constant \code{Server\-Hello.t13\-Handshake\-Producer}. That 
constant is used indirectly -- via class \code{SSL\-Handshake.SERVER_HELLO} -- to produce
\ServerHello\ messages in class \code{Client\-Hello.T13\-Client\-Hello\-Consumer}
(Listing~\ref{lst:T13ClientHelloConsumerC}).
\end{tcolorbox}

\begin{comment}
\footnotetext{Class \code{ServerHello.T13ServerHelloProducer} implements
  interface \code{HandshakeProducer}, which defines a single method, namely, 
  \code{byte[] produce(ConnectionContext context, HandshakeMessage message) throws IOException}, 
  where parameter \code{context} defines the active context in both interfaces 
  and may be cast to children \code{ClientHandshakeContext} and \code{ServerHandshakeContext}.
  That context may be updated during production of messages or production may conclude
  by returning \textcolor{red}{``the encoded producing'' (reword)}.
  \textcolor{red}{Are these details worth knowing?}
}
\end{comment}


\lstinputlisting[
  float=tbp,
  linerange={
    85-122
  },
  label=lst:ServerHelloMessage,
  caption={[\code{ServerHello.ServerHelloMessage} defines \ServerHello/\HelloRetryRequest]
  Class \code{ServerHello.ServerHelloMessage} defines the six fields 
  of a \ServerHello\ message (Lines~86--91), two additional fields 
  for production and consumption of a \HelloRetryRequest\ message (Lines~96 \&~100), 
  and constructors to instantiate those fields from parameters (Lines~102--122) 
  or an input buffer (Listing~\ref{lst:ServerHelloMessageB}). The former constructor does 
  not populate the extensions field,
  %\sout{ (and a call to method \code{SSLExtensions.produce}, Listing~\ref{lst:SSLExtensions}, is required)}
  whereas the latter may (Listing~\ref{lst:ServerHelloMessageB}, Line~173--174).
}]{listings/ServerHello.java}

\lstinputlisting[
  float=tbp,
  linerange={
    124-126,
    128-129,
    131-133,
    141-142,
    149-155,
    157-157,
    163-170,
    172-177,
    179-183,
    205-216,
    246-246
  },
  label=lst:ServerHelloMessageB,
  caption={[\code{ServerHello.ServerHelloMessage} defines \ServerHello/\HelloRetryRequest\ 
    (cont.)]
  Class \code{ServerHello.ServerHelloMessage} (continued from 
  Listing~\ref{lst:ServerHelloMessage}) defines a constructor 
  which instantiates \ServerHello\ or \HelloRetryRequest\ messages
  from an input buffer (Lines~124-183), checking that the server-selected cipher suite 
  (Lines~149--150) is amongst those offered by the client (Lines~151--155), and method 
  \code{send} to write %\ServerHello\ or \HelloRetryRequest\ fields 
  such messages 
  to an output stream, using  
  method \code{SSLExtensions.send} to write the extensions field
  (Listing~\ref{lst:SSLExtensions}, Lines~293--307).
}]{listings/ServerHello.java}


\lstinputlisting[
  float=tbp,
  linerange={
    59-60,
    484-485,
    492-493,
    495-501,
%    495-496,
%    501-501,
    513-514,
    516-516,
    518-525,
%    519-522,
%    525-525,
    531-532,
    535-536,
    538-542,
%    539-542,
    546-546,
    549-551,
%    550-551,
    557-563
  },
  label=lst:T13ServerHelloProducer,
  caption={[\code{ServerHello.T13ServerHelloProducer} produces \ServerHello]
  Class \code{ServerHello.T13ServerHelloProducer} defines method 
  \code{produce} to write a \ServerHello\ message to an output stream. 
  Prior to instantiating such a message, the server's active context is updated to 
  include extensions -- in particular,
  \TLSsignatureAlgorithms, \TLSsignatureAlgorithmsCert, and \TLSpsk\ -- that 
  may impact the \ServerHello\ message (Lines~519--522 or~539--542). Moreover, 
  the active context is updated to include a producer for \EncryptedExtensions\ 
  and \Finished\ messages (Lines~560--563). Code for writing the \ServerHello\ 
  message appears in Listing~\ref{lst:T13ServerHelloProducerB}.
}]{listings/ServerHello.java}


\lstinputlisting[
  float=tbp,
  linerange={
    565-578,
    582-585,
    677-692,
    694-694,
    697-701,
    713-714,
    725-727
  },
  label=lst:T13ServerHelloProducerB,
  caption={[\code{ServerHello.T13ServerHelloProducer} produces \ServerHello\
    (cont.)]
  Class \code{ServerHello.T13ServerHelloProducer} defines method 
  \code{produce} (continued from Listing~\ref{lst:T13ServerHelloProducer}) 
  which instantiates a \ServerHello\ message~(Lines 566--571), populates 
  the extension field for the server's active context (Lines~575--578), and writes 
  the \ServerHello\ message to an output stream (Lines~584--585). The 
  \ServerHello\ message parameterises \TLSlegacyVersion\ as constant 0x0303 
  (Line~567), \TLSlegacySessionIdEcho\ as \ClientHello.\TLScipherSuites\ 
  (Line~568), and \TLScipherSuite\ as the negotiated cipher suite (Line~569), 
  which is the server selected cipher suite for (EC)DHE-only key exchange 
  (Lines~525 \&~531, Listing~\ref{lst:T13ServerHelloProducer}), selected using 
  method \code{chooseCipherSuite}, or the cipher suite %used by the previous session 
  associated with the pre-shared key for PSK-based key exchange (Line~546, Listing~\ref{lst:T13ServerHelloProducer}). 
  \ifImplNotes\textcolor{red}{Truthify regarding DHE/PSK/both}\fi
  The output stream is written to using method \code{ServerHello.ServerHelloMessage.write}, 
  defined by parent class \code{SSLHandshake.HandshakeMessage}, which in 
  turn uses method \code{ServerHello.ServerHelloMessage.send} (Listing~\ref{lst:ServerHelloMessageB}). (After outputting the message, the server 
  updates the active context to include new keying material in preparation 
  for the server's response, Section~\ref{sec:hkdf}.)
  Method \code{chooseCipherSuite} instantiates lists of 
  preferred and proposed cipher suites as the list of available 
  cipher suites and the list of offered cipher suites, respectively,
  or vice-versa, depending on the active context (Lines~684--692),
  and returns the first preferred cipher suite that is amongst those 
  proposed (Lines~694--714) or \code{null} if no such suite exists
  (Line~725).
}]{listings/ServerHello.java}


In addition to mandatory extension \TLSsupportedVersions, message \ServerHello\ 
must include additional extensions depending on the key exchange mode: For ECDHE/DHE,
\label{comp:SH:prof:keyShare} extension
\TLSkeyShare\ is included in association with the server's key share, which must 
be in the group selected by the server from \ClientHello.\TLSsupportedGroups; 
for PSK-only, extension \TLSpsk\ is included in association with 
the pre-shared key identifier selected by the server from \ClientHello.\TLSpsk; 
and for PSK with (EC)DHE, both of those extensions are included.
Additional extensions are sent separately 
in the \EncryptedExtensions\ message.

\begin{sloppypar}
A \ServerHello\ message is consumed by the client: The client first checks that
the message is a TLS 1.3 \ServerHello\ message, which is achieved by checking 
that extension \TLSsupportedVersions\ is present and that constant 0x0304 is
the first listed preference. 
\ifSpecNotes
\begin{color}{red}
The specification seems to contain conflicting requirements:\marginpar{conflict reported 1 May 2020}
\begin{quote}
  Clients MUST check for this [supported_versions] extension \emph{prior to
  processing} the rest of the ServerHello (although they will have to           
  parse the ServerHello in order to read the extension)
\end{quote}
and 
\begin{quote}
  Upon receiving a message with type server_hello, implementations MUST      
  \emph{first examine} the Random value
\end{quote}
which conflict on the emphasised text. Presumably, these must be the 
first two checks, but their ordering is unclear.
\end{color}
\fi
Next, the client checks whether the server's nonce 
(\TLSrandom) is a special value (defined by constant 
\code{RandomCookie.hrrRandomBytes}) indicating that the \ServerHello\ message
is a \HelloRetryRequest\ message and should be processed as such 
(\S\ref{sec:HelloRetryRequest}). \ifPresentationNotes\marginpar{Maybe drop inline reference to \code{RandomCookie}
  if including OpenSSL, or include a reference to OpenSSL's handling.}\fi
The client also checks whether \label{comp:SH:cons:version} the server selected protocol version
(\TLSsupportedVersions) is amongst those offered (\ClientHello.\TLSsupportedVersions)
and is at least version 1.3, whether the server selected cipher suite
(\TLScipherSuite) is amongst those offered (\ClientHello.\TLScipherSuites),
and whether field \TLSlegacySessionIdEcho\ matches 
\ClientHello.\TLSlegacySessionId, aborting with an \TLSillegalParameter\
alert if any check fails. Finally, the client processes any remaining 
extensions:
\end{sloppypar}


\begin{description}
\ifSpecNotes
\item   \textcolor{red}{\TLSkeyShare: 
    I expected the client to check whether the server's key share is 
    in a group [selected by the server] from \ClientHello.\TLSsupportedGroups, 
    but that doesn't seem to be required.}
\fi

\item \TLSpsk: The client checks whether the server-selected key 
  identifier is amongst those offered by the client, 
  the server-selected cipher suite defines a hash function associated 
  with that identifier, and 
  extension \TLSkeyShare\ is present if the offered key exchange 
  mode for that identifier is PSK with (EC)DHE, aborting with an 
  \TLSillegalParameter\ alert if either check fails.
\end{description}

%\noindent 
%\textcolor{red}{...XXX...}

\begin{tcolorbox}
Consumption is implemented by class \code{ServerHello.ServerHelloConsumer}
(Listing~\ref{lst:ServerHelloConsumer}). That class checks the presence of extension 
\TLSsupportedVersions, to determine whether the message is a TLS~1.3 \ServerHello\ message, and 
the remainder of the message is processed by \code{ServerHello.T13ServerHelloConsumer}
(Listing~\ref{lst:T13ServerHelloConsumer}), if it is a TLS 1.3 message. 
\end{tcolorbox}

\lstinputlisting[
  float=tbp,
  linerange={
    55-56,%handshakeConsumer
    839-840,
    847-848,
    850-850,
    864-864,
    869-874,
    928-929,
    933-936,
%    937-960, svs != null for TLS 1.3
    937-941,
    943-944,
    948-954,
    956-956,
%    957-960,
    984-984,
    993-994
  },
  label=lst:ServerHelloConsumer,
  caption={[\code{ServerHello.ServerHelloConsumer} consumes generic 
            \ServerHello/\HelloRetryRequest]
  Class \code{ServerHello.ServerHelloConsumer} defines method \code{consume} to 
  instantiate a (generic) \ServerHello\ message from an input buffer (Line~864) and processes the message
  as a \HelloRetryRequest\ (Line~870) or a \ServerHello\ message (Line~872). The latter
  updates the client's active context to include the server's selected version (Lines~933--936), 
  using method \code{SupportedVersionsExtension.SHSupportedVersionsConsumer.consume}, which 
  calls \code{chc.handshakeExtensions.put(SH\_SUPPORTED\_VERSIONS, spec)},
  and checks whether that version was offered by the client (Lines~949), aborting if it
  was not (Lines~950--953) and, otherwise, updating the active context to include that version as the   
  negotiated protocol (Lines~956--960). 
  (Variable \code{serverVersion}, Lines~943--944, cannot be null for (TLS 1.3) \HelloRetryRequest\ 
  nor \ServerHello\ messages.)
  Further processing is deferred (Line~984) to class \code{ServerHello.T13ServerHelloConsumer}
  (Listing~\ref{lst:T13ServerHelloConsumer}).
  %\textcolor{red}{Maybe explicitly mention that Lines~958 \&~959 update \code{TransportContext}.}
}]{listings/ServerHello.java}




\lstinputlisting[
  float=tbp,
  linerange={
    69-70,
    1172-1173,
    1180-1181,
    1183-1184,
    1189-1190,
    1193-1193,
    1200-1203,
    1214-1216,
    1219-1219,
    1221-1222,
    1228-1228,
    1230-1231,
%    1236-1236, %Calls MaxFragExtension.shOnTradeConsumer (uninteresting?)
    1330-1330,
    1339-1356,
    1362-1363
  },
  label=lst:T13ServerHelloConsumer,
  caption={[\code{ServerHello.T13ServerHelloConsumer} consumes \ServerHello/\HelloRetryRequest]
  Class \code{ServerHello.T13ServerHelloConsumer} defines method \code{consume}
  to process incoming (TLS 1.3) \ServerHello\ or \HelloRetryRequest\ messages
  (further to processing shown in Listing~\ref{lst:ServerHelloConsumer}). The method
  updates the client's active context to include the server's selected cipher suite as the 
  negotiated suite (Line~1190), extensions, including \TLSpsk\ (Lines~1200--1202), and 
  additional session information  (Lines~1203--1231). (The client also updates the active 
  context to include new keying material, Section~\ref{sec:hkdf}.) Moreover, the  
  active context is made ready to received further server messages (Lines~1330--1356).
}]{listings/ServerHello.java}










