\section{Introduction}\label{sec:intro}

\ifPresentationNotes
\textcolor{red}{
 It would be nice to open on a very broad note that illustrates the importance
 of TLS to society, rather than diving straight into the technical details.
 Perhaps start from e-commerce, e.g., by building upon the following:
}
%
%
\begin{comment}
The e-commerce market was valued at over twenty trillion US dollars in 2016
and is forecast to double in value by 2022, with business-to-business
sales accounting for over eighty percent of the market share. 
%%
%% Source: https://www.researchandmarkets.com/research/swhq24/global_ecommerce
%%
That market is reliant on secure communication, which can be achieved using
TLS.
\end{comment}
%
% Jumping straight into the tech seems to miss an opportunity.
%
%
The Internet delivered in excess of forty terabytes per second in 2017 
and is expected to deliver more than three times that by 2022, % (Cisco, 2018),
%%
%% Cisco (2018) "Cisco Visual Networking Index: Forecast and Trends, 2017–2022," 
%% White Paper 1543280537836565, https://www.cisco.com/c/en/us/solutions/collateral/service-provider/visual-networking-index-vni/white-paper-c11-741490.html (Table 7)
%%  
%and over half of today's Internet traffic is encrypted (Sandvine, 2018);
%%
%% Sandvine (2018) "The Global Internet Phenomena Report," White Paper,  
%% https://www.sandvine.com/hubfs/downloads/phenomena/2018-phenomena-report.pdf
%%
%%with sources suggesting we are nearing an all-encrypted Internet; 
%-- with nearly all of today's Internet traffic being encrypted -- 
%%
%% Multiple sources have tracked, but they have seemingly stopped tracking, 
%% perhaps because we're so close to all-encrypted (e.g., figures around 90%).
%%
enabling trade worth trillions of dollars. %(Statista, 2017). 
%%
%% Statista (2017) "Retail e-commerce sales worldwide from 2014 to 2021,"
%% https://www.statista.com/statistics/379046/worldwide-retail-e-commerce-sales/
%%
\fi
%
%Sources suggest we 
We are nearing an all-encrypted Internet; 
yet, the underlying encryption technology %used to secure communication channels
is understood by only a select few. 
%% "a select few" is idiomatic
This manuscript broadens understanding by exploring TLS, an encryption technology 
used to protect application layer communication (including HTTP, FTP and SMTP traffic), 
and by examining OpenJDK's Java implementation. 
%We focus on the most recent version of which is defined by RFC8446,
We focus on the most recent TLS release, namely, version 1.3, which is defined by RFC~8446.
\ifPresentationNotes
\textcolor{red}{needs extending/revising or just plain rewriting}
\fi

\ifPresentationNotes
\marginpar{The history of TLS appears in many manuscripts on TLS. 
  It has been done; it can probably be omitted here. (If not, then push to a sidebar 
  or an appendix.)}
\fi

TLS is a protocol that establishes a channel between an initiating \emph{client} and a 
interlocutory \emph{server} (also known as \emph{endpoints} and \emph{peers}),
%. The protocol is designed to enable:
for the purpose of enabling:


\begin{description}

\item Authentication. 
  %The client's belief of the server's identity
  %is correct, and similarly for the server's belief.
  An endpoint's belief of their peer's identity is correct.

\item Confidentiality. Communication over an established channel is only
  visible to endpoints.

\item Integrity. Communication over an established channel is received-as-sent, 
  or tampering is detected.

\end{description}

\noindent
These properties should hold even in the presence of an adversary that has 
complete control of the underlying network, i.e., an adversary that may read, 
modify, drop, and inject messages. 

\ifPresentationNotes
\marginpar{\emph{cryptographic primitives} vs. \emph{cryptographic schemes} vs. ...}
\fi

The TLS protocol commences with a \emph{handshake}, wherein cryptographic primitives 
and parameters are negotiated, and shared (traffic) keys are established. 
Moreover, the handshake %\sout{typically} 
includes unilateral authentication of the server. (Mutual authentication of both 
the client and the server is also possible.) %\sout{, as-is unauthenticated communication}.) 
The handshake results in a channel which uses the negotiated cryptography and 
parameters, along with a shared key, to protect communication.

%\marginpar{I'd like to use \emph{key} only for traffic keys, and \emph{secrets} 
%  for the inputs used to derive them. But, the term \emph{pre-shared keys} makes
%  that impossible and (EC)DHE key share \& (EC)DHE key makes it awkward.}

\begin{comment}
Shared keys are established using one of the three supported key exchange 
modes: Ephemeral Diffie-Hellman over finite fields (DHE) or elliptic curves (ECDHE), 
pre-shared key (PSK), or PSK with (EC)DHE. (EC)DHE key exchange requires no prior knowledge, whereas PSK-based key exchanges requires knowledge of a pre-shared key, 
which may have been established out-of-band or during a previous connection. Such a 
pre-shared key also serves to authenticate endpoints, whereas (EC)DHE-only key 
exchange is reliant on asymmetric cryptography for authentication.
\end{comment}
%
% The above implicitly assumes knowledge of key exchange, let's try to focus
% on the high-level functional requirement instead:
%
%\begin{comment}
The handshake does not require any prior knowledge: A shared key may be derived 
from secrets established using Diffie-Hellman key exchange over finite fields (DHE) 
or elliptic curves (ECDHE). Alternatively, %endpoints may derive a shared (traffic) key 
such a shared key may be derived from a secret pre-shared key (PSK), which %they 
endpoints
establish %out-of-band 
externally or during a previous connection. (Shared keys 
%are derived from underlying secrets 
are combined with nonces to ensure they are always unique, regardless of whether
secrets have been previously used.)
The former achieves \emph{forward secrecy} -- i.e., confidentiality is 
preserved even if long-term keying material is compromised after the handshake, 
as long as (EC)DHE secrets are erased  -- whereas the latter does not.
The two key exchange modes can be combined, using PSK with (EC)DHE key exchange,
to achieve forward secrecy with pre-shared keys. 
%\sout{Pre-shared keys serve to authenticate endpoints, whereas (EC)DHE-only key 
%exchange is reliant on asymmetric cryptography for authentication. }
%\end{comment}
%
% If using the above, mention PSK with (EC)DHE
%

The handshake is itself a protocol (summarised in Figure~\ref{fig:handshake}). 
It is commenced by the client sending a 
\emph{\ClientHello} message, comprising: a nonce; offered protocol versions,  
symmetric ciphers, and hash functions; offered Diffie-Hellman key 
shares, pre-shared key labels, or both; 
and %any additional extensions.
details of any extended functionality.
The protocol proceeds with the server receiving the client's 
message, establishing mutually acceptable cryptographic primitives and parameters, 
and responding with a \emph{\ServerHello} message, containing: a nonce; 
selected protocol version, symmetric cipher, and hash function; and
a Diffie-Hellman key share, a selected pre-shared key label, or both. (The server 
may respond with a \emph{\HelloRetryRequest} message, if the offered key shares
are unsuitable.) Once the client receives the server's message, a shared
(handshake traffic) key can be established to enable confidentiality and integrity for the 
remainder of the handshake protocol. In particular, that shared key
is used to protect an \emph{\EncryptedExtensions} message, sent by the server to the 
client, which may detail extended functionality.

The handshake protocol concludes with unilateral authentication of the 
server. (Client authentication is also possible.) For (EC)DHE-only key 
exchange, after sending the \EncryptedExtensions\ message, the server sends 
a \emph{\Certificate} message, containing a certificate (or some other 
suitable material corresponding to the server's long-term, private key), 
and a \emph{\CertificateVerify} message, containing a signature 
(using the private key corresponding to the public key in the certificate)
over a hash of the handshake protocol's \emph{transcript} (i.e., a 
concatenation of each handshake message, e.g., \ClientHello, \ServerHello,
\EncryptedExtensions, and \Certificate, in this instance).
Finally, the server 
sends a \emph{\Finished} message, containing a Message Authentication Code (MAC) 
over the protocol's transcript, which provides key confirmation, binds 
the server's identity to the exchanged keys, and, for PSK-based
key exchange, authenticates the handshake. Moreover,
the client responds with a \Finished\ message of its own. A shared (application
traffic) key can then be established to protect communication of application data. 

\begin{figure}
\caption[Handshake protocol]{
  A client initiates the handshake protocol by sending a \ClientHello\ (CH) message. 
  After sending that message, the client waits for a \ServerHello\ (SH) message 
  followed by an \EncryptedExtensions\ (EE) message, or a \HelloRetryRequest\ (HRR) 
  message. An \EncryptedExtensions\ message might be followed by a \CertificateRequest\
  (CR) message (requesting client authentication). Moreover, for certificate-based
  server authentication, the client waits for a \Certificate\ (CT) message followed 
  by a \CertificateVerify\ (CV) message. The handshake protocol concludes upon 
  an exchange of  \Finished\ (FIN) messages from each of the client and server. 
  (We omit the client's \Finished\ message for brevity.) The client's \Finished\ 
  message may be preceded by client generated \Certificate\ and \CertificateVerify\ 
  messages, when client authentication is requested. (We omit those messages for
  brevity.) 
}
\label{fig:handshake}
\begin{tikzpicture}
  \node[state, initial] (1) {Begin};
  \node[state, right=of 1] (2) {Wait};
  \node[state, right=of 2] (3) {Wait};
  \node[state, right=of 3] (4) {Wait};
  \node[state, right=of 4] (5) {Wait};
  \node[state, right=of 5] (6) {Wait};
  \node[state, right=of 6, accepting] (7) {End};
  
  \draw (1) edge[bend left, above] node{CH} (2)
        (2) edge[bend left, below] node{HRR} (1)
        (2) edge[above] node{SH} (3)
        (3) edge[above] node{EE} (4)
        (4) edge[looseness=4,above] node{CR} (4)
        (4) edge[bend right, above] node{FIN} (7)  
        (4) edge[above] node{CT} (5)  
        (5) edge[above] node{CV} (6)
        (6) edge[above] node{FIN} (7);
\end{tikzpicture}
\end{figure}

Beyond the handshake protocol, TLS defines a \emph{record protocol} which 
writes handshake protocol messages (and application data, as well as 
error messages) to the transport layer, after adding headers and, where
necessary, protecting messages. 

\paragraph{Contribution and structure.}

We explore the TLS handshake (\S\ref{sec:handshake}) and record (\S\ref{sec:record})
protocols, as defined by RFC~8446,\footnote{\url{https://tools.ietf.org/html/rfc8446}.}
moreover, we examine OpenJDK's Java 
implementation,\footnote{\url{https://hg.openjdk.java.net/jdk/jdk11/file/1ddf9a99e4ad/}, 
 that repository may be discontinued, a GitHub repository is now used:
 \url{https://github.com/openjdk/jdk}.
} 
namely, JDK~11 package \code{sun.security.ssl}.

