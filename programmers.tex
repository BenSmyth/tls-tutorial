\section{Java Secure Socket Extension (JSSE)}\label{sec:JSSE}

Java programmers need not concern themselves with the intricacies of TLS 
nor OpenJDK's implementation: They can 
use the Java Secure Socket Extension (JSSE), which provides an abstract, high-level
API to establish a TLS channel. Doing otherwise is outright dangerous! TLS 1.3 
was developed over four years by a team of almost one hundred security experts 
from more than forty institutions, including tech behemoths Amazon, Apple, Google, 
IBM, and Microsoft. \ifPresentationNotes \textcolor{red}{Part of this is probably 
better placed elsewhere.} \fi
Their work involved iterating over the subtle 
details to ensure that security objectives were achieved. 
JSSE abstracts away those subtleties to provide programmers with a high-level, 
low-risk means to establish secure communication, without the complexities of underlying 
OpenJDK algorithms.
We present toy applications that demonstrate the use of JSSE (Section~\ref{sec:monkeys}), 
before delving into the details (Section~\ref{sec:SunJSSE}).

\subsection{Examples for code monkeys: Toy client and server}\label{sec:monkeys}

\ifPresentationNotes
\marginpar{``channel'' (throughout) is perhaps too academic}
\fi

JSSE trivialises the development of toy applications. For instance, the following code 
snippet establishes a TLS socket:\footnote{%
  Prepending the snippet with 
    \code{System.setProperty("javax.net.debug", "ssl handshake verbose")}
  prints additional information, which can be useful.}

\lstinputlisting[
  numbers=none,
  frame=none,
  nolol=true,
  linerange={
    16-22
  }
]{code/JSSEClient.java}

\noindent 
The established TLS socket protects communication, for example, the following HTTP
request and response is protected:

\lstinputlisting[
  numbers=none,
  frame=none,
  nolol=true,
  linerange={
    24-39
  }
]{code/JSSEClient.java}

\noindent
JSSE uses a ``provider''-based architecture, whereby services (e.g., \code{SSLSocket} and \code{SSLSocketFactory})
and implementations (e.g., \code{SSLSocketImpl} and \code{SSLSocketFactoryImpl}) are defined independently, 
and are (typically) instantiated by factory methods (e.g., \code{SSLSocketFactory.getDefault()}). Hence, 
programmers need not concern themselves with the inner-workings of implementations, such as those provided 
by \emph{SunJSSE} (we will nonetheless take a brief look in Section~\ref{sec:SunJSSE}). 
Let us now consider a toy server application, to compliment our (above) toy client.

Our client uses the default client-side context, whereas our server cannot, because 
server-side authentication is mandatory and a certification must be initialised, 
hence, we start by initialising a suitable context:

\lstinputlisting[
  numbers=none,
  frame=none,
  nolol=true,
  linerange={
    23-31
  }
]{code/JSSEServer.java}

\noindent
That context can be used to establish a TLS socket:

\lstinputlisting[
  numbers=none,
  frame=none,
  nolol=true,
  linerange={
    33-41
  }
]{code/JSSEServer.java}

\noindent
Communication over the TLS socket is protected, for example, any incoming character is
protected as is any subsequent response:

\lstinputlisting[
  numbers=none,
  frame=none,
  nolol=true,
  linerange={
    43-51
  }
]{code/JSSEServer.java}

\noindent 
Our client and server can communicate by assigning \code{InetAddress.getLocalHost()} 
to variable \code{host} and \code{8443} to variable \code{port},\footnote{Unix systems
  protect ports under 1024, hence, we use port 8443, rather than port 443.} 
rather than \code{example.com} and \code{443}, respectively. The key 
store necessary for this example can be constructed using \code{keytool} as follows:

\begin{verbatim}
bas $ keytool -genkey -keyalg RSA -keystore store
Enter keystore password:  
Re-enter new password: 
What is your first and last name?
  [Unknown]:  127.0.1.1
What is the name of your organizational unit?
  [Unknown]:   
What is the name of your organization?
  [Unknown]:  
What is the name of your City or Locality?
  [Unknown]:  
What is the name of your State or Province?
  [Unknown]:  
What is the two-letter country code for this unit?
  [Unknown]:  
Is CN=127.0.1.1, OU=Unknown, O=Unknown, L=Unknown, ST=Unknown, C=Unknown correct?
  [no]:  yes

bas $
\end{verbatim}


\noindent
where \code{InetAddress.getLocalHost()} is 127.0.1.1 and filename \code{store} can 
be replaced with alternatives. Since the above key store is 
self-signed, it must be added to the Java virtual machine's trust store, which can 
be achieved by prepending client code with the following:
\code{System.setProperty( "javax.net.ssl.trustStore", "store" ); System.setProperty( "javax.net.ssl.trustStorePassword", <<pwd>> )}.









\subsection{SunJSSE provider for architects, researchers, and the curious}\label{sec:SunJSSE}

Our toy client uses statement \code{SSLSocketFactory.getDefault().createSocket(host, port)} 
to instantiate an instance of 
\code{SSLSocketFactoryImpl} parameterised with an initial context, and uses that 
context along with variables \code{host} and \code{port} to instantiate and return an 
instance of \code{SSLSocketImpl}. Method \code{SSLSocketImpl.startHandshake()} proceeds as follows:

\lstinputlisting[
  frame=none,
  nolol=true,
  linerange={
    377-377,
    395-398,
    401-403,
    411-411
  }
]{listings/SSLSocketImpl.java}

\noindent
Line~395 indirectly calls method \code{ClientHello.kickstartProducer.produce} (\S\ref{sec:CH}) and processes 
responses (Line~401--403) as follows:

\lstinputlisting[
  frame=none,
  nolol=true,
  linerange={
    1060-1061,
    1063-1067,
    1077-1077,
    1079-1080,
    1148-1149,
    1152-1153,
    1170-1171
  }
]{listings/SSLSocketImpl.java}

\noindent
Hence, until a connection is negotiated (Lines 1064--1067), responses are processed by 
method \code{SSLTransport.decode} (Line~1152--1153) as follows:

\lstinputlisting[
  frame=none,
  nolol=true,
  linerange={
    101-105,
    107-108,
    145-149,
    162-165,
    %167-191,
    %197-199,
    200-204
 }
]{listings/SSLTransport.java}

\noindent
Line~108 parses a handshake (or an alert) record header and calls method 
\code{SSLSocketInputRecord.decodeInputRecord}, providing the header as 
input. That method parses and decodes the complete record, which is then 
processed by method \code{TransportContext.dispatch} (Line~164).
%
Method \code{SSLSocketInputRecord.decodeInputRecord} proceeds as follows:

\lstinputlisting[
  frame=none,
  nolol=true,
  linerange={
    204-210,
    230-230,
    232-246,
    257-257,
    259-259,
    261-264,
    280-280,
    282-344,
    346-353
 }
]{listings/SSLSocketInputRecord.java}

\noindent
Finally, method \code{TransportContext.dispatch} proceeds as follows:

\lstinputlisting[
  frame=none,
  nolol=true,
  linerange={
    143-144,
    149-149,
    156-179,
    191-192
 }
]{listings/TransportContext.java}

\noindent
Line~178 indirectly calls \code{HandshakeContext.dispatch} on variables
\code{handshakeType} and \code{plaintext.fragment}, which calls the 
relevant consumer and updates the handshake hash.





%\subsection{Here be dragons: Hacking OpenJDK's TLS}

%TLS is a subtle, complex beast which has proven rather difficult to secure.
%JSSE strips away those subtleties to provide programmers with a high-level, 
%low-risk means to establish secure communication: Programmers need not
%expose themselves to the complexities of underlying OpenJDK algorithms, 
%Doing otherwise poses a threat to user security. Nonetheless, hackers like 
%to play. Proceed at your own risk. \emph{Hic sunt dracones}.


