\section{Java Secure Socket Extension (JSSE)}\label{sec:JSSE}

Java programmers need not concern themselves with the intricacies of TLS 
nor OpenJDK's implementation: They can 
use the Java Secure Socket Extension (JSSE), which provides an abstract, high-level
API to establish a TLS channel. Doing otherwise is outright dangerous! TLS 1.3 
was developed over four years by a team of almost one hundred security experts 
from more than forty institutions, including tech behemoths Amazon, Apple, Google, 
IBM, and Microsoft. \ifPresentationNotes \textcolor{red}{Part of this is probably 
better placed elsewhere.} \fi
They iterated over subtle details to ensure security objectives are achieved.
JSSE abstracts away those subtleties to provide programmers a high-level, 
low-risk means to secure communication, without the complexities of underlying 
OpenJDK algorithms.
We present toy applications that demonstrate the use of JSSE (Section~\ref{sec:monkeys}), 
before delving into the details (Section~\ref{sec:SunJSSE}).

%%%%%%%%%%%%%%%%%%%%%%%%%%%%%%%%%%%%%%%%%%%%%%%%%%%%%%%%%%%%%%%%
%%%%%%%%%%%%%%%%%%%%%%%%%%%%%%%%%%%%%%%%%%%%%%%%%%%%%%%%%%%%%%%%
%%%%%%%%%%%%%%%%%%%%%%%%%%%%%%%%%%%%%%%%%%%%%%%%%%%%%%%%%%%%%%%%

\subsection{Examples for code monkeys: Toy client and server}\label{sec:monkeys}

\ifPresentationNotes
\marginpar{``channel'' (throughout) is perhaps too academic}
\fi

JSSE trivialises development. For instance, the following code 
snippet establishes a TLS socket:\footnote{%
  Prepending the snippet with 
    \code{System.setProperty("javax.net.debug", "ssl handshake verbose")}
  or 
    \code{System.setProperty("javax.net.debug", "all")}
  prints additional information, which can be useful. 
  (For further details, see \url{https://docs.oracle.com/javase/8/docs/technotes/guides/security/jsse/ReadDebug.html}.)
}

\lstinputlisting[
  numbers=none,
  frame=none,
  nolol=true,
  linerange={
    16-22
  }
]{code/JSSEClient.java}

\noindent 
The established TLS socket protects communication, for example, the following HTTP
request and response is protected:

\lstinputlisting[
  numbers=none,
  frame=none,
  nolol=true,
  linerange={
    24-39
  }
]{code/JSSEClient.java}

\noindent
JSSE uses a provider-based architecture, whereby services (e.g., \code{SSLSocket} and \code{SSLSocketFactory})
and implementations (e.g., \code{SSLSocketImpl} and \code{SSLSocketFactoryImpl}) are defined independently, 
and are (typically) instantiated by factory methods (e.g., \code{SSLSocketFactory.getDefault()}). Hence, 
programmers need not concern themselves with the inner-workings of implementations, such as those provided 
by \emph{SunJSSE} (we will nonetheless take a brief look in Section~\ref{sec:SunJSSE}). 
Let us now consider a toy server application, to compliment our (above) toy client.

Our client uses the default client-side context, whereas our server cannot, because 
server-side authentication is mandatory and a certification must be initialised, 
hence, we start by initialising a suitable context:

\lstinputlisting[
  numbers=none,
  frame=none,
  nolol=true,
  linerange={
    23-31
  }
]{code/JSSEServer.java}

\noindent
That context can be used to establish a TLS socket:

\lstinputlisting[
  numbers=none,
  frame=none,
  nolol=true,
  linerange={
    33-41
  }
]{code/JSSEServer.java}

\noindent
Communication over the TLS socket is protected, for example, any incoming character is
protected, as is any subsequent response:

\lstinputlisting[
  numbers=none,
  frame=none,
  nolol=true,
  linerange={
    43-51
  }
]{code/JSSEServer.java}

\noindent 
Our client and server can communicate by assigning \code{InetAddress.getLocalHost()} 
to variable \code{host} and \code{8443} to variable \code{port},\footnote{Unix systems
  protect ports under 1024, hence, we use port 8443, rather than port 443.} 
rather than \code{example.com} and \code{443}, respectively. The key 
store necessary for this example can be constructed using \code{keytool} as follows:

\begin{verbatim}
bas $ keytool -genkey -keyalg RSA -keystore store
Enter keystore password:  
Re-enter new password: 
What is your first and last name?
  [Unknown]:  127.0.1.1
What is the name of your organizational unit?
  [Unknown]:   
What is the name of your organization?
  [Unknown]:  
What is the name of your City or Locality?
  [Unknown]:  
What is the name of your State or Province?
  [Unknown]:  
What is the two-letter country code for this unit?
  [Unknown]:  
Is CN=127.0.1.1, OU=Unknown, O=Unknown, L=Unknown, ST=Unknown, C=Unknown correct?
  [no]:  yes

bas $
\end{verbatim}


\noindent
where \code{InetAddress.getLocalHost()} is 127.0.1.1 and filename \code{store} can 
be replaced with alternatives. Since the above key store is 
self-signed, it must be added to the Java virtual machine's trust store, which can 
be achieved by prepending client code with the following:
\code{System.setProperty( "javax.net.ssl.trustStore", "store"); System.setProperty("javax.net.ssl.trustStorePassword", "pwd")}.


%%%%%%%%%%%%%%%%%%%%%%%%%%%%%%%%%%%%%%%%%%%%%%%%%%%%%%%%%%%%%%%%
%%%%%%%%%%%%%%%%%%%%%%%%%%%%%%%%%%%%%%%%%%%%%%%%%%%%%%%%%%%%%%%%
%%%%%%%%%%%%%%%%%%%%%%%%%%%%%%%%%%%%%%%%%%%%%%%%%%%%%%%%%%%%%%%%

\subsection{SunJSSE provider for architects, researchers, and the curious}\label{sec:SunJSSE}

Our toy client uses statement \code{SSLSocketFactory.getDefault().createSocket(host, port)} 
to instantiate an instance of 
\code{SSLSocketFactoryImpl} parameterised with an initial context, and uses that 
context along with variables \code{host} and \code{port} to instantiate and return an 
instance of \code{SSLSocketImpl}. Method \code{SSLSocketImpl.startHandshake()} proceeds as follows:

\lstinputlisting[
  frame=none,
  nolol=true,
  linerange={
    377-377,
    395-398,
    401-403,
    411-411
  }
]{listings/SSLSocketImpl.java}

\noindent
Line~395 indirectly calls method \code{ClientHello.kickstartProducer.produce()} (\S\ref{sec:CH}) and processes 
responses (Line~401--403) as follows:

\lstinputlisting[
  frame=none,
  nolol=true,
  linerange={
    1060-1061,
    1063-1067,
    1077-1077,
    1079-1080,
    1148-1149,
    1152-1153,
    1170-1171
  }
]{listings/SSLSocketImpl.java}

\noindent
Hence, until a connection is negotiated (Lines 1064--1067), responses are processed by 
method \code{SSLTransport.decode} (Line~1152--1153) as follows:

\lstinputlisting[
  frame=none,
  nolol=true,
  linerange={
    101-105,
    107-108,
    145-149,
    162-165,
    %167-191,
    %197-199,
    200-204
 }
]{listings/SSLTransport.java}

\begin{sloppypar}
\noindent
Line~108 parses a handshake (or an alert) record header and calls method 
\code{SSLSocketInputRecord.decodeInputRecord}, providing the header as 
input. That method parses and decodes the complete record, which is then 
processed by method \code{TransportContext.dispatch} (Line~164).
%
Method \code{SSLSocketInputRecord.decodeInputRecord} proceeds as follows:
\end{sloppypar}

\lstinputlisting[
  frame=none,
  nolol=true,
  linerange={
    204-210,
    230-230,
    232-246,
    257-257,
    259-259,
    261-264,
    280-280,
    282-344,
    346-353
 }
]{listings/SSLSocketInputRecord.java}

\noindent
Finally, method \code{TransportContext.dispatch} proceeds as follows:

\lstinputlisting[
  frame=none,
  nolol=true,
  linerange={
    143-144,
    149-149,
    156-179,
    191-192
 }
]{listings/TransportContext.java}

\begin{sloppypar}
\noindent
Line~178 indirectly calls \code{HandshakeContext.dispatch} on variables
\code{handshakeType} and \code{plaintext.fragment}, which calls the 
relevant consumer and updates the handshake hash.
\end{sloppypar}

%%%%%%%%%%%%%%%%%%%%%%%%%%%%%%%%%%%%%%%%%%%%%%%%%%%%%%%%%%%%%%%%
%%%%%%%%%%%%%%%%%%%%%%%%%%%%%%%%%%%%%%%%%%%%%%%%%%%%%%%%%%%%%%%%
%%%%%%%%%%%%%%%%%%%%%%%%%%%%%%%%%%%%%%%%%%%%%%%%%%%%%%%%%%%%%%%%

\subsection{Here be dragons: Non-blocking I/O}\label{sec:monkeys:SSLEngine}

J2SE 1.4 added non-blocking I/O and Java SE 5 added non-blocking TLS (\code{SSLEngine}), 
separating TLS functionality from I/O---Oracle advises ``[this] is an advanced API, 
and is not appropriate for casual use,''\footnote{\url{https://docs.oracle.com/en/java/javase/17/security/java-secure-socket-extension-jsse-reference-guide.html}} \emph{hic sunt dracones}:

%%%%%%%%%%%%%%%%%%%%%%%%%%%%%%%%%%%%%%%%%%%%%%%%%%%%%%%%%%%%%%%%

\subsubsection{Encapsulating \code{SSLEngine} complexity for non-casual users}

TLS is initiated by sending a \ClientHello\ message. 
\ServerHello\ and \EncryptedExtensions\ messages are then received,
or a \HelloRetryRequest\ message is. 
Upon receiving a first message, either receive a further message, or send one. 
State is already non-trivial, and, as can be inferred from Figure~\ref{fig:handshake}, 
things only get more complex, without even considering further features 
such as \NewSessionTicket\ and \KeyUpdate\ messages, nor closure and error alerts.
Such states are represented by \code{SSLEngineResult}: 
%
\begin{itemize}
\item
  Data must be sent before handshaking can continue (\code{NEED_WRAP}), 
  e.g., after calling \code{SSLEngine.beginHandshake()} to produce a \ClientHello\ message.
\item
  Data must be received before handshaking can continue (\code{NEED_UNWRAP}), 
  e.g., having called \code{SSLEngine.wrap()} to buffer the aforementioned \ClientHello.
\item
  Blocking operation is required before handshaking can continue (\code{NEED_TASK}),
  e.g., to perform certificate validation.
\item
  Handshaking has just complete (\code{FINISHED}), 
  e.g., upon receipt of a \Finished\ message.
\item
  No handshaking in progress (\code{NOT_HANDSHAKING}),
  e.g., after completion, when application data can be sent.
\end{itemize}
%
The separation of TLS functionality from I/O requires guiding \code{SSLEngine} through those states;
herein, \code{SSLEngineHost} is introduced to navigate states.

\lstset{widthgobble=0*0} 

\lstinputlisting[
  frame=none,
  nolol=true,
  commentstyle=\normalfont,
  linerange={
    12-41
  }
]{code/SSLEngineHost.java}

\noindent
We buffer outbound \& inbound application data,

\lstinputlisting[
  frame=none,
  nolol=true,
  commentstyle=\ttfamily,
  inputencoding = utf8,
  extendedchars = true,
  literate = {Λ}{$\land$}1 {V}{$\vee$}1,
  linerange={
    43-44
  }
]{code/SSLEngineHost.java}

\noindent
encode outbound application data to TLS records 
\& decode TLS records to inbound application data 
(with methods \code{SSLEngine.wrap()} \& \code{SSLEngine.unwrap()}, respectively),

\lstinputlisting[
  frame=none,
  nolol=true,
  commentstyle=\ttfamily,
  inputencoding = utf8,
  extendedchars = true,
  literate = {Λ}{$\land$}1 {V}{$\vee$}1,
  linerange={
    45-61
  }
]{code/SSLEngineHost.java}

\noindent
and buffer outbound and inbound TLS records:

\lstinputlisting[
  frame=none,
  nolol=true,
  linerange={
    62-66
  }
]{code/SSLEngineHost.java}

\noindent
We don't concern ourselves with (transport layer) communication of outbound 
TLS records, nor consumption of inbound application data; such operations
are handed-off to a listener:

\lstinputlisting[
  frame=none,
  nolol=true,
  linerange={
    69-70
  }
]{code/SSLEngineHost.java}

\noindent
TLS operations are performed on a separate thread, we block operations 
until required, 

\lstinputlisting[
  frame=none,
  nolol=true,
  linerange={
    72-73
  }
]{code/SSLEngineHost.java}

\noindent
using locks to avoid concurrent manipulation of buffers \code{outAppBuffer} and \code{inPacketBuffer}:

\lstinputlisting[
  frame=none,
  nolol=true,
  linerange={
    75-78
  }
]{code/SSLEngineHost.java}

\noindent
Our constructor takes an \code{SSLEngine} as input, initiates buffers, a dummy listener,
and a thread for TLS operations (blocked until required):

\lstinputlisting[
  frame=none,
  nolol=true,
  linerange={
    80-95
  }
]{code/SSLEngineHost.java}

\noindent
(We resize our outbound application-data buffer and inbound TLS-record buffer
when needed.) 

%%%%%%%%%%%%%%%%%%%%%%%%%%%%%%%%%%%%%%%%%%%%%%%%%%%%%%%%%%%%%%%%

\subsubsection*{Navigating TLS state}

We execute until the underlying \code{SSLEngine} is shutdown, 

\lstinputlisting[
  frame=none,
  nolol=true,
  linerange={
    97-98
  }
]{code/SSLEngineHost.java}

\noindent
blocking until operation is required:

\lstinputlisting[
  frame=none,
  nolol=true,
  linerange={
    99-99
  }
]{code/SSLEngineHost.java}

\noindent
Upon demand for a TLS operation,

\lstinputlisting[
  frame=none,
  nolol=true,
  linerange={
    101-104
  }
]{code/SSLEngineHost.java}

\noindent
we handshake, if that's what \code{SSLEngine} is ready for.

\lstinputlisting[
  frame=none,
  nolol=true,
  linerange={
    105-106
  }
]{code/SSLEngineHost.java}

\noindent
Encode application data (by overloading \code{HandshakeStatus.NEED_WRAP} to include
such encoding), if there's outbound application data (i.e., data has been written 
to buffer \code{outAppBuffer}, advancing the buffer's position---at which further 
data can be written---beyond zero).

\lstinputlisting[
  frame=none,
  nolol=true,
  linerange={
    107-108
  }
]{code/SSLEngineHost.java}

\noindent
Decode TLS records (by overloading \code{HandshakeStatus.NEED_UNWRAP} to include
such decoding), if there's inbound TLS records.

\lstinputlisting[
  frame=none,
  nolol=true,
  linerange={
    109-116
  }
]{code/SSLEngineHost.java}

\noindent
Initiate closure, if method \code{closeOutbound()} has been called (setting boolean 
\code{closeBound}).

\lstinputlisting[
  frame=none,
  nolol=true,
  linerange={
    117-121
  }
]{code/SSLEngineHost.java}

\noindent
We iterate over the above until: Handshaking has complete, there's no 
outbound application data nor inbound TLS records buffered, and any closure request 
has been processed. The outer loop terminates when \code{SSLEngine} is shutdown, 
otherwise the inner-loop is blocked until operation is required.

\lstinputlisting[
  frame=none,
  nolol=true,
  linerange={
    122-123
  }
]{code/SSLEngineHost.java}

%%%%%%%%%%%%%%%%%%%%%%%%%%%%%%%%%%%%%%%%%%%%%%%%%%%%%%%%%%%%%%%%

\subsubsection*{Encoding outbound}

Outbound TLS records are encoded by \code{SSLEngine.wrap()}:

\lstinputlisting[
  frame=none,
  nolol=true,
  linerange={
    125-140
  }
]{code/SSLEngineHost.java}

\noindent
Method \code{wrap()} reads from outbound application-data buffer \code{outAppBuffer}
and writes to TLS-record buffer \code{outPacketBuffer}. The latter is first 
prepared for writing---method \code{clear()} sets the buffer's \emph{position} 
(at which a byte can be written, or read) to zero and \emph{limit} (beyond which should 
not be written, nor read) to the buffer's capacity. Outbound application buffer 
\code{outAppBuffer} is ready to be written to, hence, we \emph{flip} the contents (set 
the buffer's limit to the current position and then set the position to zero). 
Method \code{wrap()} will consume as much outbound application data as possible 
(governed by internal state), writing data to TLS record buffer \code{outPacketBuffer}, 
updating buffer positions to reflect consumption and production. A non-zero positioned 
application-data buffer will result from a partial read and \code{compact()} moves bytes 
(between a position and limit) to the beginning of a buffer. 
Method \code{SSLEngine.wrap()} returns a description of the operation's status coupled 
with the handshaking state resulting from encoding, which we process:

\lstinputlisting[
  frame=none,
  nolol=true,
  linerange={
    142-146 
  }
]{code/SSLEngineHost.java}

\noindent
Documentation is scant, \code{CLOSED} indicates: ``The operation just closed this 
side of the SSLEngine, or the operation could not be completed because it was 
already closed.'' Presumably the former is only produced by \code{SSLEngine.unwrap()} 
upon receipt of a closure alert.\footnote{%
  Documentation seems obsolete: Prior to TLS 1.3, a closure alert required the 
  receiver to immediately send a closure alert of their own---``clos[ing] this 
  side of the SSLEngine''---whereas only the inbound direction must be closed 
  in TLS 1.3 (to avoid truncating outbound messages). (Documentation of   
  \code{SSLEngine.isInboundDone()} seems similarly obsolete, ``Returns: true if the 
  SSLEngine will not consume anymore network data (and by implication, will not 
  produce any more application data.).'')
  \ifImplNotes \textcolor{red}{TO DO: Report.} \fi
}
The latter includes \code{SSLEngine.wrap()} partially completing, suggesting some 
(but not all) handshake messages are encoded. 
%(Respectively, suggesting \code{SSLEngine.unwrap()} decodes some but not all TLS records.) 
Upon inspection of source code (\code{SSLEngineImpl.java}, Lines~114--266), we 
discover \code{CLOSED} indicates either: no messages were encoded, or all messages 
were encoded but no further messages will be. Hence, we treat cases \code{OK} and
\code{CLOSED} almost the same---transmitting any TLS records produced by \code{wrap()}. 
(Application data buffered for wrapping cannot be transmitted once closed, 
case \code{CLOSED} additionally sets the buffer's position to zero, avoiding method 
\code{doTLS()} infinitely looping.)
TLS record buffer \code{outPacketBuffer} is handed-off to our listener:

\lstinputlisting[
  frame=none,
  nolol=true,
  linerange={
    147-151
  }
]{code/SSLEngineHost.java}

\noindent
Some housekeeping:

\lstinputlisting[
  frame=none,
  nolol=true,
  linerange={
    153-171
  }
]{code/SSLEngineHost.java}

%%%%%%%%%%%%%%%%%%%%%%%%%%%%%%%%%%%%%%%%%%%%%%%%%%%%%%%%%%%%%%%%

\subsubsection*{Decoding inbound}

Inbound application data---decoded by \code{SSLEngine.unwrap()}---is handed-off to 
our listener; the following code is similar to the above, with method \code{unwrap()}
reading from inbound TLS-record buffer \code{inPacketBuffer} and writing to application 
buffer \code{inAppBuffer}.


\lstinputlisting[
  frame=none,
  nolol=true,
  literate = {\ \ \ \ \ \ \ \ \ \ //TO\ DO:\ Be\ intelligent}{\ }1,
  linerange={
    173-217
  }
]{code/SSLEngineHost.java}

%%%%%%%%%%%%%%%%%%%%%%%%%%%%%%%%%%%%%%%%%%%%%%%%%%%%%%%%%%%%%%%%

\subsubsection*{Delegated tasks \& housekeeping}

Operations requiring time-consuming or blocking tasks are processed on separate threads:

\lstinputlisting[
  frame=none,
  nolol=true,
  linerange={
    219-223
  }
]{code/SSLEngineHost.java}

\noindent
For completeness, some housekeeping:

\lstinputlisting[
  frame=none,
  nolol=true,
  linerange={
    225-235
  }
]{code/SSLEngineHost.java}

\noindent
(Documentation asserts \code{FINISHED} is only produced by \code{SSLEngineResult.getHandshakeStatus()}, 
i.e., after calling \code{SSLEngine.wrap()} or \code{SSLEngine.unwrap()}, and 
\code{FINISHED} is never produced by \code{SSLEngine.getHandshakeStatus()}.\footnotemark\ 
Documentation also asserts \code{NEED_UNWRAP_AGAIN} is only relevant for DTLS.)

\footnotetext{%
 \code{SSLEngine} and TLS parlance deviate: \code{HandshakeStatus.FINISHED} 
  reports \code{SSLEngine}'s willingness to encode application data, as opposed 
  to whether TLS permits application-data encoding---indeed, TLS permits a 
  server to encode application data after unwrapping a (client) \Finished\ message, 
  but OpenJDK defers reporting completion until wrapping message \NewSessionTicket,
  which seems non-standard. Other implementations may vary. (A OpenJDK client reports
  completion upon wrapping a \Finished\ message, as expected.)
}

%%%%%%%%%%%%%%%%%%%%%%%%%%%%%%%%%%%%%%%%%%%%%%%%%%%%%%%%%%%%%%%%

\subsubsection*{Buffering outbound application data and inbound TLS records}

Outbound application data is buffered:

\lstinputlisting[
  frame=none,
  nolol=true,
  linerange={
    237-246
  }
]{code/SSLEngineHost.java}

\noindent
As are inbound TLS records:

\lstinputlisting[
  frame=none,
  nolol=true,
  linerange={
    248-257
  }
]{code/SSLEngineHost.java}

\noindent
Wherein:

\lstinputlisting[
  frame=none,
  nolol=true,
  linerange={
    259-263
  }
]{code/SSLEngineHost.java}

\noindent
And \code{semaphore.release()} is invoked to unblock our \code{while}-loop.

%%%%%%%%%%%%%%%%%%%%%%%%%%%%%%%%%%%%%%%%%%%%%%%%%%%%%%%%%%%%%%%%

\subsubsection*{Initiate handshake}

\lstinputlisting[
  frame=none,
  nolol=true,
  linerange={
    265-276
  }
]{code/SSLEngineHost.java}

%%%%%%%%%%%%%%%%%%%%%%%%%%%%%%%%%%%%%%%%%%%%%%%%%%%%%%%%%%%%%%%%

\subsubsection*{Closure}

\lstinputlisting[
  frame=none,
  nolol=true,
  linerange={
    278-300
  }
]{code/SSLEngineHost.java}

%%%%%%%%%%%%%%%%%%%%%%%%%%%%%%%%%%%%%%%%%%%%%%%%%%%%%%%%%%%%%%%%

\subsubsection*{Offloading I/O to \code{SSLEngineHostListener}}

Resulting outbound TLS records and inbound application data are handled by a listener, 
instantiated by the following method.

\lstinputlisting[
  frame=none,
  nolol=true,
  linerange={
    302-304
  }
]{code/SSLEngineHost.java}

\noindent
Our dummy listener is straightforward:

\lstinputlisting[
  frame=none,
  nolol=true,
  firstline=306
]{code/SSLEngineHost.java}

\noindent
(Subclasses add functionality.)


\lstset{widthgobble=1*1}

%%%%%%%%%%%%%%%%%%%%%%%%%%%%%%%%%%%%%%%%%%%%%%%%%%%%%%%%%%%%%%%%
%%%%%%%%%%%%%%%%%%%%%%%%%%%%%%%%%%%%%%%%%%%%%%%%%%%%%%%%%%%%%%%%
%%%%%%%%%%%%%%%%%%%%%%%%%%%%%%%%%%%%%%%%%%%%%%%%%%%%%%%%%%%%%%%%

\subsubsection{Simplified non-blocking I/O for the masses}

\textcolor{red}{Give exemplar client \& server here}

\subsubsection*{Client-side}

\subsubsection*{Server-side}

%%%%%%%%%%%%%%%%%%%%%%%%%%%%%%%%%%%%%%%%%%%%%%%%%%%%%%%%%%%%%%%%
%%%%%%%%%%%%%%%%%%%%%%%%%%%%%%%%%%%%%%%%%%%%%%%%%%%%%%%%%%%%%%%%
%%%%%%%%%%%%%%%%%%%%%%%%%%%%%%%%%%%%%%%%%%%%%%%%%%%%%%%%%%%%%%%%

\subsubsection*{We're not alone.}

Alex Karnezis's \href{https://github.com/alkarn/sslengine.example}{sslengine.example} makes a 
similar exposition of \code{SSLEngine}. 
%
Kashif Razzaqui's \href{https://github.com/kashifrazzaqui/sslfacade/}{sslfacade} provides a 
higher-level API, with outgoing TLS records and incoming application data being passed to 
a user-defined listener.
%
Mariano Barrios's \href{https://github.com/marianobarrios/tls-channel}{TLS Channel} implements
interface \code{ByteChannel} over \code{SSLEngine}, providing the same interface as 
\code{SocketChannel}, the non-blocking counterpart to \code{Socket} (of which \code{SSLSocket}
is a subclass).
%
These projects separate handshaking from transmission of application data, we do not.
