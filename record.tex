\section{Record protocol}\label{sec:record}

Handshake messages are encapsulated into one or more \TLSPlaintext\ records (\S\ref{sec:TLSPlaintext}), 
which, for \ClientHello, \ServerHello\ and \HelloRetryRequest\ messages, are 
immediately written to the transport layer, otherwise, \TLSPlaintext\ records are 
translated to \TLSCiphertext\ records (\S\ref{sec:TLSCiphertext}), which add protection 
prior to writing to the transport layer. (Alerts are similarly encapsulated and when 
appropriate protected. Application data is always encapsulated and protected.)

\subsection{\TLSPlaintext}\label{sec:TLSPlaintext}

Handshake messages are fragmented and each fragment is encapsulated into a 
\TLSPlaintext\ record, comprising the following fields:

\begin{description}

\item \TLStype: Constant 0x22 (\TLShandshake). (Other constants are used for records 
  encapsulating data other than handshake messages, e.g., alerts and application 
  data.)

\item \TLSlegacyRecordVersion: Constant 0x0303, except for an initial \ClientHello\
  message, which may use constant 0x0301. 


\item \TLSlength: The byte length of the following field (namely, \TLSfragment), which must not exceed
  $2^{14}$ bytes.

\item \TLSfragment: A handshake message fragment.

\end{description}

\noindent
An endpoint that receives a \TLSPlaintext\ record with field \TLSlength\ set 
greater than $2^{14}$ must abort with a \TLSrecordOverflow\ alert.

\subsection{\TLSCiphertext}\label{sec:TLSCiphertext}

For protection, a \TLSPlaintext\ record is transformed into 
a \TLSCiphertext\ record, comprising of the following fields:

\begin{description}

\item \TLSopaqueType: Constant 0x23. 

\item \TLSlegacyRecordVersion: Constant 0x0303.

\item \TLSlength: The byte length of the following field (namely, \TLSencryptedRecord), which must not 
  exceed $2^{14} + 256$ bytes.

\item \TLSencryptedRecord:  Encrypted data.

\end{description}

\noindent
Encrypted data is computed, using the negotiated AEAD algorithm,
as 
\[
  %\TLSAEADEncrypted =
    \AEADEnc(\textit{write\_key}, \textit{nonce}, \textit{additional_data}, \textit{plaintext}),
\]
where \textit{write\_key} is either \TLSclientWriteKey\ or \TLSserverWriteKey;
\textit{nonce} is derived from a sequence number XORed with \TLSclientWriteIV\ or 
\TLSserverWriteIV, respectively; \textit{additional_data} is the \TLSCiphertext\ 
record header, i.e., 
$\textit{additional_data} = 
            \TLSCiphertext.\TLSopaqueType \parallel 
            \TLSCiphertext.\TLSlegacyRecordVersion \parallel 
            \TLSCiphertext.\TLSlength$;
and \textit{plaintext} comprises of \TLSPlaintext.\TLSfragment\ appended with type
\TLSPlaintext.\TLStype\ and field \TLSzeros, which contains an arbitrary-length run 
of zero-valued bytes and is used to pad a TLS record (the resulting plaintext is known as 
record \TLSInnerPlaintext).

An endpoint that receives a \TLSCiphertext\ record with field \TLSlength\ set 
greater than $2^{14} + 256$ must abort with a \TLSrecordOverflow\ alert. Otherwise,
the endpoint computes 
\[
  \AEADDec(\textit{write\_key}, \textit{nonce},  \textit{additional_data}, %\TLSAEADEncrypted),
     \TLSCiphertext.\TLSencryptedRecord),
\]
which outputs a plaintext or terminates with an error. The endpoint aborts
with a \TLSbadRecordMac\ alert in the event of such an error.


\paragraph{Per-record nonce.}

The nonce used by the negotiated AEAD algorithm is derived from a 64-bit
sequence number, which is initialised as 0, incremented by one after 
reading or writing a record, and reset to 0 whenever the key is changed. 
That sequence number is XORed with \TLSclientWriteIV\ 
or \TLSserverWriteIV\ to derive the nonce.

\begin{tcolorbox}
Outgoing records are produced by class \code{SSLSocketOutputRecord} 
(Listing~\ref{lst:SSLSocketOutputRecord}) and parent \code{OutputRecord} 
(Listing~\ref{lst:OutputRecord}), using enum \code{SSLCipher} 
(Listing~\ref{lst:SSLCipher}) to protect outgoing records.
(Alternatively, outgoing records are constructed by class 
\code{SSLEngineOutputRecord}, which shares the same parent.) 
Incoming records are consumed by class \code{SSLSocketInputRecord} 
(or \code{SSLEngineInputRecord}) and parent \code{InputRecord}, which 
uses enum \code{SSLCipher} for record protection.
\end{tcolorbox}

\lstinputlisting[
  float=tbp,
  widthgobble=0*0,
  linerange={
    38-38,
    39-39,
    94-95,
    147-163,
    165-166,
    168-172,
    181-182,
    184-186,
    193-194,
    196-199,
    589-589
  },
  label=lst:SSLSocketOutputRecord,
  caption={[\code{SSLSocketOutputRecord.encodeHandshake} fragments outgoing handshake messages]
  Class \code{SSLSocketOutputRecord} defines method \code{encodeHandshake} to fragment 
  outgoing handshake messages and write fragments to (its parent's) buffer \code{buf} (Lines~159 \&~169), 
  using  method \code{ByteArrayOutputStream.write}, 
  which (if full) is processed by parent \code{OutputRecord} (Line~182) and delivered
  (Lines~185--186). \ifPresentationNotes\textcolor{red}{What special case is dealt with on Line~159? TO DO: 
  Give further explanation of the for-loop}\fi
  The class is also responsible for adding the encapsulated message to the transcript hash, 
  if appropriate (Lines~147--150).
}]{listings/SSLSocketOutputRecord.java}%

\lstinputlisting[
  float=tbp,
  widthgobble=0*0,
  linerange={
    42-43,
    %44-44,     %%Constructor
    %48-49,
    %66-66,
    %82-85,
    %87-87,
    %90-90,
    %248-256,   %%Used by SSLEngineOutputRecord
    %270-272,
    %279-279,
    %322-359,
    %
    %385-388,
    389-390,
    392-392,
    396-396,
    398-400,
    402-445  
  },
  label=lst:OutputRecord,
  caption={[\code{OutputRecord.t13Encrypt} produces records \TLSPlaintext\ or \TLSCiphertext]
  Class \code{OutputRecord} defines method \code{t13Encrypt} which appends constant 0x22
  (defined by variable \code{ContentType.HANDSHAKE.id}) and padding (defined by constant
  \code{zeros}) to buffer \code{buf} if outgoing data should be encrypted (Lines~404--408), 
  i.e., when producing record \TLSCiphertext, as opposed to \TLSPlaintext; encrypts the data 
  in that buffer (Lines~432--433), using a null cipher (\code{SSLCipher.NullReadCipherGenerator})
  %%%
  %%% Method SSLCipher.NullReadCipherGenerator.encrypt includes the following:
  %%%
  %%%   MAC signer = (MAC)authenticator;
  %%%   if (signer.macAlg().size != 0) {
  %%%      addMac(signer, bb, contentType);
  %%%   } else {
  %%%      authenticator.increaseSequenceNumber();
  %%%   }
  %%%
  %%% which suggests a MAC can be added. However, CipherSuite.MacAlg defines M_NULL
  %%% such that M_NULL.size = 0. So I suspect M_NULL is in use and no MAC is added.
  %%%
  if data should not be encrypted and a cipher in Galois/Counter Mode 
  (\code{SSLCipher.T13GcmWriteCipherGenerator}) otherwise; and 
  adds the header fields for record \TLSPlaintext\ or \TLSCiphertext\ (Lines~438--442), which 
  only differ on the first byte, in particular, the former uses constant 0x22 (which is input 
  by child \code{SSLSocketOutputRecord}, in the context of Listing~\ref{lst:SSLSocketOutputRecord}),
  whereas the latter uses constant 0x23 (Line~427).
}]{listings/OutputRecord.java}%

\lstinputlisting[
  float=tbp,
  widthgobble=0*0,
  linerange={
    1945-1946,
    1956-1973,
    1987-1987,
    1990-2005,
    2006-2006,
    2009-2009,
    2011-2018,
    %2025-2033,
    2035-2039,
    2043-2043,
    2054-2055,
    2082-2083
  },
  label=lst:SSLCipher,
  caption={[\code{SSLCipher.T13GcmWriteCipherGenerator} encrypts data in Galois/Counter Mode]
  Class \code{SSLCipher.T13GcmWriteCipherGenerator} defines method \code{encrypt} which 
  XORs the sequence number and initialisation vector (Lines~1992--1997); initialises a 
  cipher (Line~2003), using algorithm parameters that define the bit length of the 
  authentication tag and the initialisation vector (Lines~2000-2001); and encrypts the
  input data (Line~2036), appending the authentication tag (Lines~2013--2016), which 
  increments the sequence number as a side effect. (Method \code{Authenticator.TLS13Authenticator.acquireAuthenticationBytes} increments the 
  sequence number using method \code{Authenticator.increaseSequenceNumber}.)
}]{listings/SSLCipher.java}%
